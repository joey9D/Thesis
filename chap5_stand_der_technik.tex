In diesem Kapitel wird der aktuelle Stand der techwerden alternative Lösungen recherchiert und untersucht.
Ziel ist die Analyse bestehender Erkenntnisse und Theorien sowie die Identifizierung von Forschungslücken, die die Relevanz der vorliegenden Arbeit begründen.
Es wird geschaut, welche Ansätze verwendet werden, wie weit diese Ansätze den Anforderungen der eigenen Lösung entsprechen und ob diese in der eigenen Umsetzung der Lösung helfen.



\section{Recherche}
Welche Lösungen gibt es bereits?
Wie machen das andere; auch in Form kommerzieller Werkzeuge (cubeIDE, Espressif):
\begin{itemize}
	\item stm32CubeIDE
	\item Espressif IDE
	\item mcu-cpp
	\item modm
\end{itemize}

\section{Bewertung der Alternativlösungen}
\subsection{STM32CubeIDE}


\subsectionm{Espressif-IDE}


\subsection{mcu-cpp}
Pro:
Interface-/virtuelle Klasse, die als Basis dienen.

Verwendung von eigenen \texttt{namespaces} $\rightarrow$ sorgt für klaren, lesbaren Code in der Implementierung von Anwendungen.



Con:
Keine Verwendung der existierenden Treiber $\rightarrow$ z.b. HAL; statt dessen selbst geschriebene Treiber bzw. Wrapperklassen.

Führt dazu, dass für jede neue Hardware eine neue Wrapper geschrieben werden muss, was den Anfordungen widerspricht.

\subsection{modm}


\section{Abgrenzung des eigenen Ansatzes}
\subsection{modm}


