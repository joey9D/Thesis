\section{Motivation und Problemstellung}
% Unterschied von MCUs
Microcontroller (eng. \gls{mcu}) unterscheiden sich in vielerlei Hinsicht, unter anderem in ihrer Architektur, der verfügbaren Peripherie, dem Befehlssatz sowie in der Art und Weise, wie ihre Hardwarekomponenten über Register angesteuert werden.
% Fokus auf Register
Die Aufgabe dieser Register besteht in der Konfiguration und Steuerung grundlegender Funktionen, wie etwa der digitalen Ein- und Ausgänge, der Taktung, der Kommunikationsschnittstellen oder der Interrupt-Verwaltung.
Die konkrete Implementierung sowie die Adressierung und Bedeutung einzelner Bits und Bitfelder variieren jedoch von Hersteller zu Hersteller und sogar zwischen verschiedenen Serien desselben Herstellers erheblich.

% Kernproblem: Neu-Implementierung von Applikationen pro Hardware
Diese signifikante Varianz in Bezug auf die Hardware-Komponenten führt dazu, dass Softwarelösungen und Applikationen, für jede neue Plattform entweder vollständig neu entwickelt oder zumindest aufwendig angepasst werden müssen.
Obwohl Abstraktionschichten, sog. \gls{hal}, eine gewisse Erleichterung bei der Entwicklung bieten, resultieren daraus gleichzeitig starke Bindungen an die zugrunde liegende Hardwareplattform.

% TODO: chap1 Motivation: viele MCUs pro Projekt? Nochmal Michael fragen
% TODO: chap1: Quellen zu den Behauptungen finden/raussuchen. Dieses Wissen kommt nur von Michaels Erzählung
Insbesondere in Projekten, in denen mehrere Microcontroller-Plattformen parallel eingesetzt werden oder ein Wechsel der Zielplattform absehbar ist, steigt der Bedarf an portabler und modularer Software signifikant an. 
In der Praxis zeigt sich, dass das Fehlen von Abstraktion häufig zu redundantem Code, fehleranfälliger Portierung und ineffizienter Entwicklung führt.

\clearpage

\section{Ausgangssituation und Zielsetzung}
% Ausgangssituation in der Firma
Die Arbeit entsteht in der Produktentwicklung der Schaefer GmbH.
Hier werden die Produkte von Grund auf konzipiert, implementiert und getestet.
Dabei arbeiten Hardware- und Softwareentwicklung eng zusammen, denn es muss bekannt sein, welche Microcontroller verwendet und wie Displays und Taster angesprochen werden können, um den Anforderungen zu entsprechen.
% TODO: chap1: Ausgangssituation: Welche Produktlinien werden hauptsächlich verwendet?
Hauptsächliche kommen hier die STM32 Microcontroller der Firma ST zum Einsatz.
Diese bieten ein umfangreiches Portfolio, das von stromsparenden IoT-Bausteinen bis hin zu leistungsfähigen Controllern für grafikfähige Anwendungen reicht.
Die Auswahl und die damit verbundene Skalierbarkeit der Hardware ermöglichen eine flexible Anpassung an die unterschiedlichsten Leistungs- und Energieanforderungen.
Zusätzlich steht eine Vielzahl an integrierten Schnittstellen für Peripherie und Funktionsblöcken zur Verfügung, wodurch die Abdeckung verschiedenster Anwendungsbereiche ermöglicht wird.
Um die Software zu implementieren und mit der Hardware zu arbeiten, wird die STM32Cube-Umgebung bereitgestellt, auf die in \cref{sec:stm32cube} genauer eingegangen wird.
Außerdem kommen zum Teil eigens in C++ erstellte Klassen zum Einsatz, um über die HAL hinaus eine klarere Abstraktion und bessere Wiederverwendbarkeit zu erreichen.
Ein wesentliches Problem aktueller Entwicklungen besteht darin, dass Programme in vielen Fällen für jede Zielhardware neu implementiert oder stark angepasst werden müssen.
Innerhalb der STM32-Familie lässt sich dies vergleichsweise einfach durch Konfigurationen lösen. 
Beim Wechsel auf hardwarefremde Plattformen, wie etwa den ESP32, sind umfangreiche Anpassungen erforderlich. 
Diese reichen bis hin zu einer nahezu vollständigen Neuentwicklung des Codes.

% Ziel
Das Ziel dieser Arbeit besteht somit in der Entwicklung einer modularen, plattformunabhängigen und ressourceneffizienten Treiberbibliothek mit einer einheitlichen Schnittstelle. 
Diese soll eine nachhaltige, wartbare und flexible Softwarebasis schaffen, die den Herausforderungen der modernen Embedded-Entwicklung adäquat begegnen kann und eine  leicht Erweiterung um Funktionen und den evtl. Einsatz von \gls{rtos} ermöglicht.


\section{Aufbau der Arbeit}
Der Aufbau dieser Bachelorarbeit folgt einer klar strukturierten Herangehensweise, die im Folgenden erläutert wird. 

Aufbauend auf der in Kapitel 1 ''Einleitung'' beschriebenen Motivation und Problemstellung wird im Kapitel 2 ''Aufgabenstellung'' die konkrete Zielsetzung der Arbeit definiert.
Es werden die Anforderungen an die Entwicklung einer plattformunabhängigen Treiber-API beschrieben und die Rahmenbedingungen der Umsetzung definiert.
Darüber hinaus wird dargelegt, welche Werkzeuge im Entwicklungsprozess eingesetzt werden und anhand welcher Kriterien die Erfüllung der Aufgabe bewertet werden kann.

Darauf folgt das Kapitel 3 „Grundlagen”, das relevante technische Konzepte und Begriffe einführt. 
Im Fokus stehen hierbei insbesondere Aspekte der Microcontroller-Programmierung, wie die Verwendung von Ports und Registern, der Zugriff auf die Peripherie und grundlegende Prinzipien der hardwarenahen Softwareentwicklung. 
Dieses Kapitel schafft das notwendige Fachwissen, um die weiteren Inhalte der Arbeit in ihrem technischen Kontext zu verstehen, und bildet somit die Grundlage für das Verständnis der nachfolgenden Themengebiete.

Anschließend gibt das Kapitel 4 „Stand der Technik” einen Überblick über bestehende Lösungen zur plattformübergreifenden Ansteuerung von Microcontrollern und der Bereitstellung der plattformabhängigen Hardwaretreibern.
Dabei werden verschiedene Ansätze analysiert, verglichen und hinsichtlich ihrer Stärken und Schwächen bewertet. 
Ziel ist es, daraus Erkenntnisse für die eigene Umsetzung zu gewinnen und bewährte Konzepte zu identifizieren.

In Kapitel 5 ''Konzeption der API'' werden die einzelnen Schritt beschrieben, die behandelt werden müssen, damit die API erstellt werden kann.

Kapitel 6 ''Durchführung'' widmet sich der praktischen Umsetzung der API.
Auf Grundlage der zuvor erarbeiteten Anforderungen und Erkenntnisse wird eine eigene Architektur entworfen, die vorhandene Treiber integriert, hardwarespezifische Funktionen abstrahiert und eine flexible Erweiterbarkeit ermöglicht.
Dabei kommen etablierte Open-Source-Werkzeuge zum Einsatz.

Mit Hilfe dieser Werkzeuge wird eine modulare, portable und ressourcenschonende Lösung realisiert.






















