\let\ifdeutsch\iftrue
\input{pre_document.tex}
\documentclass[
	ngerman,
	a4paper,
%	twoside
	oneside
]{scrbook}	
% !TeX encoding = utf8
% -*- coding:utf-8 mod:LaTeX -*-\

% DE: In dieser Datei werden zuerst die benoetigten Pakete eingebunden und
%     danach diverse Optionen gesetzt. Achtung Reihenfolge ist entscheidend!

% DE: Styleguide:
%
% Ein sehr kleiner Styleguide. Packages werden in Blöcken organisiert.
% Zwischen zwei Blöcken sind 2 Leerzeilen!


% EN: Enable copy and paste of text from the PDF
%     Only required for pdflatex. It "just works" in the case of lualatex.
%     mmap enables mathematical symbols but does not work with the newtx font set
%     See: https://tex.stackexchange.com/a/64457/9075
%     Other solutions outlined at http://goemonx.blogspot.de/2012/01/pdflatex-ligaturen-und-copynpaste.html and http://tex.stackexchange.com/questions/4397/make-ligatures-in-linux-libertine-copyable-and-searchable
%     Troubleshooting outlined at https://tex.stackexchange.com/a/100618/9075

\ifluatex
\else
  \usepackage{cmap}
\fi

% DE: Codierung
%     Wir sind im 21 Jahrhundert, utf-8 löst so viele Probleme.
%
% Mit UTF-8 funktionieren folgende Pakete nicht mehr. Bitte beachten!
%   * fancyvrb mit §
%   * easylist -> http://www.ctan.org/tex-archive/macros/latex/contrib/easylist/
\ifluatex
  % EN: See https://tex.stackexchange.com/a/158517/9075
  %     Not required, because of usage of fontspec package
  %\usepackage[utf8]{luainputenc}
\else
  \usepackage[utf8]{inputenc}
\fi

% DE: Parallelbetrieb tex4ht und pdflatex

\makeatletter
\@ifpackageloaded{tex4ht}{
  \def\iftex4ht{\iftrue}
}{
  \def\iftex4ht{\iffalse}
}
\makeatother

% DE: Mathematik
%
% DE: Viele Mathematik-Sachen. Siehe https://texdoc.net/pkg/amsmath
%
% EN: Options must be passed this way, otherwise it does not work with glossaries
% DE: fleqn (=Gleichungen linksbündig platzieren) funktioniert nicht direkt. Es muss noch ein Patch gemacht werden:
\PassOptionsToPackage{fleqn,leqno}{amsmath}


%% EN: Fonts
%% DE: Schriften
%%
%% !!! If you change the font, be sure that words such as "workflow" can
%% !!! still be copied from the PDF. If this is not the case, you have
%% !!! to use glyphtounicode. See comment at cmap package


% EN: Times Roman for all text
\ifluatex
  \RequirePackage{amsmath}
  \RequirePackage{unicode-math}
  \setmainfont{TeX Gyre Termes}
  \setmathfont{texgyretermes-math.otf}
  \setsansfont[Scale=.9]{TeX Gyre Heros}
  \setmonofont[StylisticSet={1,3},Scale=.9]{inconsolata}
\else
  \RequirePackage{newtxtext}
  \RequirePackage{newtxmath}
  % EN: looks good with times, but no equivalent for lualatex found,
  %     therefore replaced with inconsolata
  %\RequirePackage[zerostyle=b,scaled=.9]{newtxtt}
  \RequirePackage[varl,scaled=.9]{inconsolata}

  % DE: Symbole
  % unicode-math scheint für die meisten schon etwas anzubieten
  %
  %\usepackage[geometry]{ifsym} % \BigSquare

  % EN: The euro sign
  % DE: Das Euro Zeichen
  %     Fuer Palatino (mathpazo.sty): richtiges Euro-Zeichen
  %     Alternative: \usepackage{eurosym}
  \newcommand{\EUR}{\ppleuro}
\fi


% EN: Alternative Font: Palantino. It is recommended by Prof. Ludewig for German texts
% DE: Alternative Schriftart: Palantino, Packageparamter [osf] = Minuskel-Ziffern
%     Bitte nur in deutschen Texten
\usepackage{mathpazo} %ftp://ftp.dante.de/tex-archive/fonts/mathpazo/ - Tipp aus DE-TEX-FAQ 8.2.1

% DE: Schriftart fuer Programmcode - ueberschreibt lmodern
%     Falls auskommentiert, wird die Standardschriftart lmodern genommen
%     Fuer schreibmaschinenartige Schluesselwoerter in den Listings - geht bei alten Installationen nicht, da einige Fontshapes (<>=) fehlen
%\usepackage[scaled=.92]{luximono}
%\usepackage{courier}
% DE: BeraMono als Typewriter-Schrift, Tipp von http://tex.stackexchange.com/a/71346/9075
%\usepackage[scaled=0.83]{beramono}

% EN: backticks (`) are rendered as such in verbatim environments.
%     See the following links for details:
%     - https://tex.stackexchange.com/a/341057/9075
%     - https://tex.stackexchange.com/a/47451/9075
%     - https://tex.stackexchange.com/a/166791/9075
\usepackage{upquote}

% EN: For \texttrademark{}
\usepackage{textcomp}

% EN: name-clashes von marvosym und mathabx vermeiden:
\def\delsym#1{%
  %  \expandafter\let\expandafter\origsym\expandafter=\csname#1\endcsname
  %  \expandafter\let\csname orig#1\endcsname=\origsym
  \expandafter\let\csname#1\endcsname=\relax
}

% EN: Modern font encoding
%     Has to be loaded AFTER any font packages. See https://tex.stackexchange.com/a/2869/9075.
\ifluatex
\else
  \usepackage[T1]{fontenc}
\fi

% EN: Character protrusion and font expansion. See http://www.ctan.org/tex-archive/macros/latex/contrib/microtype/
% DE: Optischer Randausgleich und Grauwertkorrektur
\usepackage[
  babel=true, % EN: Enable language-specific kerning. Take language-settings from the language of the current document (see Section 6 of microtype.pdf)
  expansion=alltext,
  protrusion=alltext-nott, % EN: Ensure that at listings, there is no change at the margin of the listing
  final % EN: Always enable microtype, even if in draft mode. This helps finding bad boxes quickly.
        %     In the standard configuration, this template is always in the final mode, so this option only makes a difference if "pros" use the draft mode
]{microtype}


% EN: \texttt{test -- test} keeps the "--" as "--" (and does not convert it to an en dash)
\DisableLigatures{encoding = T1, family = tt* }


% EN: For theorems, replacement for amsthm
\usepackage[amsmath,hyperref]{ntheorem}
\theorempreskipamount 2ex plus1ex minus0.5ex
\theorempostskipamount 2ex plus1ex minus0.5ex
\theoremstyle{break}
\newtheorem{definition}{Definition}[section]


% CTAN: https://ctan.org/pkg/lccaps
% Doc: http://texdoc.net/pkg/lccaps
%
% Required for DE/EN \initialism
\usepackage{lccaps}


% EN: Definition of colors. The "hyperref" argument is not used as we do not want to change the border colors of links: Links are not colored anymore.
% DE: Farbdefinitionen
\usepackage[dvipsnames]{xcolor}


% EN: Required for custom acronyms/glossaries style.
%     Left aligned Columns in tables with fixed width.
%     See http://tex.stackexchange.com/questions/91566/syntax-similar-to-centering-for-right-and-left
\usepackage{ragged2e}
 
% DE: Wichtig, ansonsten erscheint "No room for a new \write"
\usepackage{scrwfile}

% abstract-Umgebung für scrbook definieren
\newenvironment{abstract}{
  \cleardoublepage
  \thispagestyle{plain}
  \null\vfill
  \begin{center}
    \bfseries \abstractname
  \end{center}
  \itshape
}{
  \vfill\null
  \cleardoublepage
}

% DE: Neue deutsche Rechtschreibung und Literatur statt "Literature"
%     Die folgende Einstellung ist der Nachfolger von ngerman.sty
\ifdeutsch
  % DE: letzte Sprache ist default, Einbindung von "american" ermöglicht \begin{otherlanguage}{amercian}...\end{otherlanguage} oder \foreignlanguage{american}{Text in American}
  %     Siehe auch http://tex.stackexchange.com/a/50638/9075
  \usepackage[american,main=ngerman]{babel}
  % Ein "abstract" ist eine "Kurzfassung", keine "Zusammenfassung"
  \addto\captionsngerman{%
    \renewcommand\abstractname{Kurzfassung}%
  }
  \ifluatex
    % EN: conditionally disable ligatures. See https://github.com/latextemplates/scientific-thesis-template/issues/54
    %     for a discussion
    \usepackage[ngerman]{selnolig}
  \fi
\else
  % EN: Set English as the language and allow to write hyphenated"=words
  %     `american`, `english` and `USenglish` are synonyms for babel package (according to https://tex.stackexchange.com/questions/12775/babel-english-american-usenglish).
  %      "english" has to go last to set it as the default language
  \usepackage[ngerman,main=english]{babel}
  % EN: Hint by http://tex.stackexchange.com/a/321066/9075 -> enable "= as dashes
  \addto\extrasenglish{\languageshorthands{ngerman}\useshorthands{"}}
  \ifluatex
    % EN: conditionally disable ligatures. See https://github.com/latextemplates/scientific-thesis-template/issues/54
    %     for a discussion
    \usepackage[english]{selnolig}
  \fi
\fi
%

% DE: Anführungszeichen
%     Zitate in \enquote{...} setzen, dann werden automatisch die richtigen Anführungszeichen verwendet.
%     Dieses package erzeugt eine Warnung, wenn es vor minted (genauer fvextra) geladen wird.
\usepackage{csquotes}


% EN: For even easier quotations: \qq{text}.
%     Is not smart in the case of nesting, but good enough for most cases
\usepackage{textcmds}
\ifdeutsch
  % EN: German quotes are different. So do not use the English quotes, but the ones provided by the csquotes package.
  \renewcommand{\qq}[1]{\enquote{#1}}
\fi


% DE: erweitertes Enumerate
\usepackage{paralist}


% DE: Gestaltung der Kopf- und Fußteilen
\usepackage[automark]{scrlayer-scrpage}

\automark[section]{chapter}
\setkomafont{pageheadfoot}{\normalfont\sffamily}
\setkomafont{pagenumber}{\normalfont\sffamily}


% DE: Intelligentes Leerzeichen um hinter Abkürzungen die richtigen Abstände zu erhalten, auch leere.
%     Siehe commands.tex \gq{}
\usepackage{xspace}
% DE: Macht \xspace und \enquote kompatibel
\makeatletter
\xspaceaddexceptions{\grqq \grq \csq@qclose@i \} }
\makeatother

\newcommand{\eg}{e.\,g.,\ }
\newcommand{\ie}{i.\,e.,\ }

% EN: introduce \powerset - hint by http://matheplanet.com/matheplanet/nuke/html/viewtopic.php?topic=136492&post_id=997377
\DeclareFontFamily{U}{MnSymbolC}{}
\DeclareSymbolFont{MnSyC}{U}{MnSymbolC}{m}{n}
\DeclareFontShape{U}{MnSymbolC}{m}{n}{
  <-6>    MnSymbolC5
  <6-7>   MnSymbolC6
  <7-8>   MnSymbolC7
  <8-9>   MnSymbolC8
  <9-10>  MnSymbolC9
  <10-12> MnSymbolC10
  <12->   MnSymbolC12%
}{}
\DeclareMathSymbol{\powerset}{\mathord}{MnSyC}{180}

% DE: Anhang
\usepackage{appendix}
%[toc,page,title,header]
%

% DE: Grafikeinbindungen
%
% EN: The parameter "pdftex" is not required
\usepackage{graphicx}
\graphicspath{ {Pics/} }
%\newcommand{\getgraphicspath}{ Pics/ }


% EN: Enables inclusion of SVG graphics - 1:1 approach
%    This is NOT the approach of https://ctan.org/pkg/svg-inkscape,
%     which allows text in SVG to be typeset using LaTeX
%     We just include the SVG as is.
\usepackage{epstopdf}
\epstopdfDeclareGraphicsRule{.svg}{pdf}{.pdf}{%
  inkscape -z -D --file=#1 --export-pdf=\OutputFile
}


% EN: Enables inclusion of SVG graphics - text-rendered-with-LaTeX-approach
%     This is the approach of https://ctan.org/pkg/svg-inkscape,
\newcommand{\executeiffilenewer}[3]{%
  \IfFileExists{#2}
  {
    %\message{file #2 exists}
    \ifnum\pdfstrcmp{\pdffilemoddate{#1}}%
      {\pdffilemoddate{#2}}>0%
      {\immediate\write18{#3}}
    \else
      {%\message{file up to date #2}
      }
    \fi%
  }{
    %\message{file #2 doesn't exist}
    %\message{argument: #3}
    %\immediate\write18{echo "test" > xoutput.txt}
    \immediate\write18{#3}
  }
}
\newcommand{\includesvg}[1]{%
  \executeiffilenewer{#1.svg}{#1.pdf}%
  {
    inkscape -z -D --file=\getgraphicspath#1.svg %
    --export-pdf=\getgraphicspath#1.pdf --export-latex}%
  \input{\getgraphicspath#1.pdf_tex}%
}


% EN: Enable typesetting values with SI units.
\ifdeutsch
  \usepackage[mode=text,group-minimum-digits=4]{siunitx}
  \sisetup{locale=DE}
\else
  \usepackage[mode=text,group-minimum-digits=4,group-separator={,}]{siunitx}
  \sisetup{locale=US}
\fi


% DE: Tabellenerweiterungen
\usepackage{array} %increases tex's buffer size and enables ``>'' in tablespecs
\usepackage{longtable}
\usepackage{dcolumn} %Aligning numbers by decimal points in table columns
\ifdeutsch
  \newcolumntype{d}[1]{D{.}{,}{#1}}
\else
  \newcolumntype{d}[1]{D{.}{.}{#1}}
\fi
\setlength{\extrarowheight}{1pt}


% DE: Eine Zelle, die sich über mehrere Zeilen erstreckt.
%     Siehe Beispieltabelle in Kapitel 2
\usepackage{multirow}


% DE: Fuer Tabellen mit Variablen Spaltenbreiten
%\usepackage{tabularx}
%\usepackage{tabulary}


% EN: Links behave as they should. Enables "\url{...}" for URL typesettings.
%     Allow URL breaks also at a hyphen, even though it might be confusing: Is the "-" part of the address or just a hyphen?
%     See https://tex.stackexchange.com/a/3034/9075.
% DE: Links verhalten sich so, wie sie sollen
%     Zeilenumbrüche bei URLs auch bei Bindestrichen erlauben, auch wenn es verwirrend sein könnte: Gehört der Bindestrich zur URL oder ist es ein Trennstrich?
%     Siehe https://tex.stackexchange.com/a/3034/9075.
\usepackage[hyphens]{url}
%
%  EN: When activated, use text font as URL font, not the monospaced one.
%      For all options see https://tex.stackexchange.com/a/261435/9075.
% \urlstyle{same}
%
% EN: Hint by http://tex.stackexchange.com/a/10419/9075.
\makeatletter
\g@addto@macro{\UrlBreaks}{\UrlOrds}
\makeatother


% DE: Logik für TeX
%     FÜr if-then-else @ commands.tex
\usepackage{ifthen}


% Colors
% DE: Farben
% Benutzerdefinierte Farben
\definecolor{HSBlue}{RGB}{0, 35, 128}
\definecolor{HSGray}{RGB}{102, 102, 102}
\definecolor{dkgreen}{rgb}{0,0.6,0}
\definecolor{gray}{rgb}{0.5,0.5,0.5}
\definecolor{mauve}{rgb}{0.58,0,0.82}
\definecolor{backcolour}{rgb}{0.95, 0.95, 0.92}


% DE: Listings
\usepackage{listings}
\lstset{language=XML,
  showstringspaces=false,
  extendedchars=true,
  basicstyle=\footnotesize\ttfamily,
  commentstyle=\slshape,
  % DE: Original: \rmfamily, damit werden die Strings im Quellcode hervorgehoben. Zusaetzlich evtl.: \scshape oder \rmfamily durch \ttfamily ersetzen. Dann sieht's aus, wie bei fancyvrb
  stringstyle=\ttfamily,
  breaklines=true,
  breakatwhitespace=true,
  % EN: alternative: fixed
  columns=flexible,
  numbers=left,
  numberstyle=\tiny,
  basewidth=.5em,
  xleftmargin=.5cm,
  % aboveskip=0mm, %DE: deaktivieren, falls man lstlistings direkt als floating object benutzt (\begin{lstlisting}[float,...])
  % belowskip=0mm, %DE: deaktivieren, falls man lstlistings direkt als floating object benutzt (\begin{lstlisting}[float,...])
  captionpos=b,  
  % colors
  backgroundcolor = \color{backcolour},
  numberstyle=\tiny\color{gray},
  keywordstyle=\color{blue},
  stringstyle=\color{dkgreen}
}

\ifluatex
\else
  % EN: Enable UTF-8 support - see https://tex.stackexchange.com/q/419327/9075
  \usepackage{listingsutf8}
  \lstset{inputencoding=utf8/latin1}
\fi

\ifdeutsch
%  \renewcommand{\lstlistlistingname}{Verzeichnis der Listings}
	\renewcommand{\lstlistlistingname}{Codeverzeichnis}
	\renewcommand{\lstlistingname}{Code}
\fi

% DE: Alternative zu Listings ist fancyvrb. Kann auch beides gleichzeitig benutzt werden.
\usepackage{fancyvrb}
%
% DE: Groesse fuer den Fliesstext. Falls deaktiviert: \normalsize
%\fvset{fontsize=\small}
%
% EN: Shrink font size of listings
\RecustomVerbatimEnvironment{Verbatim}{Verbatim}{fontsize=\footnotesize}
\RecustomVerbatimCommand{\VerbatimInput}{VerbatimInput}{fontsize=\footnotesize}
%
% EN: Hack for fancyvrb based on http://newsgroups.derkeiler.com/Archive/Comp/comp.text.tex/2008-12/msg00075.html
%     Change of the solution: \Vref somehow collided with cleveref/varioref as the output of \Vref{} was "Abschnitt 4.3 auf Seite 85"; therefore changed to \myVref -- so completely removed
%     See https://tex.stackexchange.com/q/132420/9075 for more information.
\newcommand{\Vlabel}[1]{\label[line]{#1}\hypertarget{#1}{}}
\newcommand{\lref}[1]{\hyperlink{#1}{\FancyVerbLineautorefname~\ref*{#1}}}

% DE: Bildunterschriften bei floats genauso formatieren wie bei Listings
%     Anpassung wird unten bei den newfloat-Deklarationen vorgenommen
%     https://www.ctan.org/pkg/caption2 is superseeded by this package.
\usepackage{caption}
%\captionsetup[table]{position=bottom}

% DE: Ermoeglicht es, Abbildungen um 90 Grad zu drehen
%     Alternatives Paket: rotating Allerdings wird hier nur das Bild gedreht, während bei lscape auch die PDF-Seite gedreht wird.
%     Das Paket lscape dreht die Seite auch nicht
\usepackage{pdflscape}


% EN: Required for proper environments of fancyvrb and lstlistings
%    There is also the newfloat package (recommended by minted), but we currently have no experience with that
% DE: Wird für fancyvrb und für lstlistings verwendet
\usepackage{float}

% EN: See http://www.tex.ac.uk/cgi-bin/texfaq2html?label=floats
% DE: floats IMMER nach einer Referenzierung platzieren
%\usepackage{flafter}


% EN: Put footnotes below floats
%     Source: https://tex.stackexchange.com/a/32993/9075
\usepackage{stfloats}
\fnbelowfloat

% DE: Fuer Abbildungen innerhalb von Abbildungen
%     Ersetzt die Pakete subfigure und subfig - siehe https://tex.stackexchange.com/a/13778/9075
\usepackage[hypcap=true]{subcaption}


% DE: Fußnoten
%\usepackage{dblfnote}  %Zweispaltige Fußnoten
%
% Keine hochgestellten Ziffern in der Fußnote (KOMA-Script-spezifisch):
%\deffootnote[1.5em]{0pt}{1em}{\makebox[1.5em][l]{\bfseries\thefootnotemark}}
%
% Abstand zwischen Fußnoten vergrößern:
%\setlength{\footnotesep}{.85\baselineskip}
%
% DE: Folgendes Kommando deaktiviert die Trennlinie zur Fußnote
%\renewcommand{\footnoterule}{}
%
\addtolength{\skip\footins}{\baselineskip} % Abstand Text <-> Fußnote
%
% Fußnoten immer ganz unten auf einer \raggedbottom-Seite
% fnpos kommt aus dem yafoot package
\usepackage{fnpos}
\makeFNbelow
\makeFNbottom

% DE: Variable Seitenhöhen zulassen
\raggedbottom

% DE: Falls die Seitenzahl bei einer Referenz auf eine Abbildung nur dann angegeben werden soll,
%     falls sich die Abbildung nicht auf der selben Seite befindet...
\iftex4ht
  %tex4ht does not work well with vref, therefore we emulate vref behavior
  \newcommand{\vref}[1]{\ref{#1}}
\else
  \ifdeutsch
    \usepackage[ngerman]{varioref}
  \else
    \usepackage{varioref}
  \fi
\fi


% EN: More beautiful tables if one uses \toprule, \midrule, \bottomrule
% DE: Noch schoenere Tabellen als mit booktabs mit http://www.zvisionwelt.de/downloads.html
\usepackage{booktabs}
%
%\usepackage[section]{placeins}


% EN: Graphs and Automata
%
% TODO: Since version 3.0 (2013-10-01), it supports pdflatex via the auto-pst-pdf package
%       Requires -shell-escape
%\usepackage{gastex}

%\usepackage{multicol}

% DE: kollidiert mit diplomarbeit.sty
%\usepackage{setspace}


% DE: biblatex statt bibtex
\usepackage[
  backend       = biber, %biber does not work with 64x versions alternative: bibtex8
  %minalphanames only works with biber backend
  sortcites     = true,
  bibstyle      = alphabetic,
  citestyle     = alphabetic,
  giveninits    = true,
  useprefix     = false, %"von, van, etc." will be printed, too. See below.
  minnames      = 1,
  minalphanames = 3,
  maxalphanames = 4,
  maxbibnames   = 99,
  maxcitenames  = 2,
  natbib        = true,
  eprint        = true,
  url           = true,
  doi           = true,
  isbn          = true,
  backref       = true]{biblatex}

% enable more breaks at URLs. See https://tex.stackexchange.com/a/134281.
\setcounter{biburllcpenalty}{7000}
\setcounter{biburlucpenalty}{8000}

\bibliography{bibliography}
%\addbibresource[datatype=bibtex]{bibliography.bib}

%Do not put "vd" in the label, but put it at "\citeauthor"
%Source: http://tex.stackexchange.com/a/30277/9075
\makeatletter
\AtBeginDocument{\toggletrue{blx@useprefix}}
\AtBeginBibliography{\togglefalse{blx@useprefix}}
\makeatother

%Thin spaces between initials
%http://tex.stackexchange.com/a/11083/9075
\renewrobustcmd*{\bibinitdelim}{\,}

%Keep first and last name together in the bibliography
%http://tex.stackexchange.com/a/196192/9075
\renewcommand*\bibnamedelimc{\addnbspace}
\renewcommand*\bibnamedelimd{\addnbspace}

%Replace last "and" with a comma in bibliography
%See http://tex.stackexchange.com/a/41532/9075
\AtBeginBibliography{%
  \renewcommand*{\finalnamedelim}{\addcomma\space}%
}

\DefineBibliographyStrings{ngerman}{
  backrefpage  = {zitiert auf S\adddot},
  backrefpages = {zitiert auf S\adddot},
  andothers    = {et\ \addabbrvspace al\adddot},
  %Tipp von http://www.mrunix.de/forums/showthread.php?64665-biblatex-Kann-%DCberschrift-vom-Inhaltsverzeichnis-nicht-%E4ndern&p=293656&viewfull=1#post293656
  bibliography = {Literaturverzeichnis}
}

% EN: enable hyperlinked author names when using \citeauthor
%     source: http://tex.stackexchange.com/a/75916/9075
\DeclareCiteCommand{\citeauthor}
{\boolfalse{citetracker}%
  \boolfalse{pagetracker}%
  \usebibmacro{prenote}}
{\ifciteindex
  {\indexnames{labelname}}
  {}%
  \printtext[bibhyperref]{\printnames{labelname}}}
{\multicitedelim}
{\usebibmacro{postnote}}

% EN: natbib compatibility
%\newcommand{\citep}[1]{\cite{#1}}
%\newcommand{\citet}[1]{\citeauthor{#1} \cite{#1}}
% EN: Beginning of sentence - analogous to cleveref - important for names such as "zur Muehlen"
%\newcommand{\Citep}[1]{\cite{#1}}
%\newcommand{\Citet}[1]{\Citeauthor{#1} \cite{#1}}

% DE: Blindtext. Paket "blindtext" ist fortgeschritterner als "lipsum" 
%     Wird verwendet, um etwas Text zu erzeugen, um eine volle Seite wegen Layout zu sehen.
\usepackage[math]{blindtext}


% DE: Neue Pakete bitte VOR hyperref einbinden. Insbesondere bei Verwendung des
%     Pakets "index" wichtig, da sonst die Referenzierung nicht funktioniert.
%     Für die Indizierung selbst ist unter http://xindy.sourceforge.net
%     ein gutes Tool zu erhalten.

%     Hier also neue packages einbinden.


% DE: Erlaubt Hyperlinks im Dokument.
%     Option "unicode" fixes umlauts in the PDF bookmarks - see https://tex.stackexchange.com/a/338770/9075
%     Alle Optionen nach \hypersetup verschoben, sonst crash
%     Siehe auch: "Praktisches LaTeX" - www.itp.uni-hannover.de/~kreutzm
\usepackage[unicode]{hyperref}
\addto\extrasngerman{
	  \renewcommand{\figurename}{Abb.}
	  \renewcommand{\figureautorefname}{Abb.}
  }
% DE: Da es mit KOMA 3 und xcolor zu Problemen mit den global Options kommt MÜSSEN die Optionen so gesetzt werden.
%     Eigene Farbdefinitionen ohne die Namen des xcolor packages
\definecolor{darkblue}{rgb}{0,0,.5}
\definecolor{black}{rgb}{0,0,0}


% EN: Define the color of links and more
\hypersetup{
  % have both title and number hyperlinking to content
  linktoc=all,
  bookmarksnumbered=true,
  bookmarksopen=true,
  bookmarksopenlevel=1,
  breaklinks=true,
  colorlinks=true,
  pdfstartview=Fit,
  pdfpagelayout=SinglePage, % DE: Alterntaive: TwoPageRight -- zweiseitige Darstellung: ungerade Seiten rechts im PDF-Viewer - siehe auch http://tex.stackexchange.com/a/21109/9075
  %pdfencoding=utf8, % EN: This is probably the same as passing the option "unicode" at \usepackage{hyperref}
  filecolor=darkblue,
  urlcolor=darkblue,
  linkcolor=black,
  citecolor=black
}

% DE: Abkürzungsverzeichnis - muss nach hyperref geladen werden
%
% DE: siehe http://www.dickimaw-books.com/cgi-bin/faq.cgi?action=view&categorylabel=glossaries#glsnewwriteexceeded
\usepackage[acronym,indexonlyfirst]{glossaries}
\ifdeutsch
  \addto\captionsngerman % DE: siehe https://tex.stackexchange.com/a/154566
  {%
    \renewcommand*{\acronymname}{Abkürzungsverzeichnis}
  }
\else
  \addto\captionsenglish{%
    \renewcommand*{\acronymname}{List of Abbreviations}%
  }
\fi
\renewcommand*{\glsgroupskip}{}
%
% EN: Removed Glossarie as a table as a quick fix to get the template working again
%     See http://tex.stackexchange.com/questions/145579/how-to-print-acronyms-of-glossaries-into-a-table
%
\makenoidxglossaries
%\makeglossaries


% EN: Extensions for references inside the document (\cref{fig:sample}, ...)
% DE: cleveref für cref statt autoref, da cleveref auch bei Definitionen funktioniert
\usepackage[capitalise,nameinlink,noabbrev]{cleveref}
\ifdeutsch
  \crefname{table}{Tabelle}{Tabellen}
  \Crefname{table}{Tabelle}{Tabellen}
  \crefname{figure}{\figurename}{\figurename}
  \Crefname{figure}{Abbildung}{Abbildungen}
  \crefname{equation}{Gleichung}{Gleichungen}
  \Crefname{equation}{Gleichung}{Gleichungen}
  \crefname{theorem}{Theorem}{Theoreme}
  \Crefname{theorem}{Theorem}{Theoreme}
  \crefname{listing}{\lstlistingname}{\lstlistingname}
  \Crefname{listing}{Listing}{Listings}
  \crefname{section}{Abschnitt}{Abschnitte}
  \Crefname{section}{Abschnitt}{Abschnitte}
  \crefname{paragraph}{Abschnitt}{Abschnitte}
  \Crefname{paragraph}{Abschnitt}{Abschnitte}
  \crefname{subparagraph}{Abschnitt}{Abschnitte}
  \Crefname{subparagraph}{Abschnitt}{Abschnitte}
\else
  \crefname{listing}{\lstlistingname}{\lstlistingname}
  \Crefname{listing}{Listing}{Listings}
\fi


% DE: Zur Darstellung von Algorithmen
%     Algorithm muss nach hyperref geladen werden
\usepackage[chapter]{algorithm}
\usepackage[]{algpseudocode}


% DE: Links auf Gleitumgebungen springen nicht zur Beschriftung,
%     Doc: http://mirror.ctan.org/tex-archive/macros/latex/contrib/oberdiek/hypcap.pdf
%     sondern zum Anfang der Gleitumgebung
\usepackage[all]{hypcap}


% DE: Deckblattstyle
%
\ifdeutsch
  \PassOptionsToPackage{language=german}{scientific-thesis-cover}
\else
  \PassOptionsToPackage{language=english}{scientific-thesis-cover}
\fi


% EN: Settings for captions of floats
% DE: Formatierung der Beschriftungen
%
\captionsetup{
  format=hang,
  labelfont=bf,
  justification=justified,
  %single line captions should be centered, multiline captions justified
  singlelinecheck=true,
  position=bottom
}


% EN: New float environments for listings and algorithms
%
% \floatstyle{ruled} % TODO: enabled or disabled causes no change - listings and algorithms are always ruled
%
\newfloat{Listing}{tbp}{code}[chapter]
\crefname{Listing}{Listing}{Listings}

\newfloat{Algorithmus}{tbp}{alg}[chapter]
\ifdeutsch
  \crefname{Algorithmus}{Algorithmus}{Algorithmus}
\else
  \crefname{Algorithmus}{Algorithm}{Algorithms}
  \floatname{Algorithmus}{Algorithm}
\fi



% EN: Various chapter styles
% DE: unterschiedliche Chapter-Styles
%     u.a. Paket fncychap

% Andere Kapitelueberschriften
% falls einem der Standard von KOMA nicht gefaellt...
% Falls man zurück zu KOMA moechte, dann muss jede der vier folgenden Moeglichkeiten deaktiviert sein.

%\usepackage[Sonny]{fncychap}

%\usepackage[Bjarne]{fncychap}

%\usepackage[Lenny]{fncychap}

%DE: Zur Aktivierung eines der folgenden Möglichkeiten ein Paar von "\iffalse" und "\fi" auskommentieren

\iffalse
  \usepackage[Bjarne]{fncychap}
  \ChNameVar{\Large\sf} \ChNumVar{\Huge} \ChTitleVar{\Large\sf}
  \ChRuleWidth{0.5pt} \ChNameUpperCase
\fi

\iffalse
  \usepackage[Rejne]{fncychap}
  \ChNameVar{\centering\Huge\rm\bfseries}
  \ChNumVar{\Huge}
  \ChTitleVar{\centering\Huge\rm}
  \ChNameUpperCase
  \ChTitleUpperCase
  \ChRuleWidth{1pt}
\fi

\iffalse
  \usepackage{fncychap}
  \ChNameUpperCase
  \ChTitleUpperCase
  \ChNameVar{\raggedright\normalsize} %\rm
  \ChNumVar{\bfseries\Large}
  \ChTitleVar{\raggedright\Huge}
  \ChRuleWidth{1pt}
\fi

\iffalse
  \usepackage[Bjornstrup]{fncychap}
  \ChNumVar{\fontsize{76}{80}\selectfont\sffamily\bfseries}
  \ChTitleVar{\raggedright\Large\sffamily\bfseries}
\fi

% EN: Complete different chapter style - self-made

% Innen drin kann man dann noch zwischen
%   * serifenloser Schriftart (eingestellt)
%   * serifenhafter Schriftart (wenn kein zusaetzliches Kommando aktiviert ist) und
%   * Kapitälchen wählen
\iffalse
  \makeatletter
  %\def\thickhrulefill{\leavevmode \leaders \hrule height 1ex \hfill \kern \z@}

  %Fuer Kapitel mit Kapitelnummer
  \def\@makechapterhead#1{%
    \vspace*{10\p@}%
    {\parindent \z@ \raggedright \reset@font
      %Default-Schrift: Serifenhaft (gut fuer englische Dokumente)
      %A) Fuer serifenlose Schrift:
      \fontfamily{phv}\selectfont
      %B) Fuer Kapitaelchen:
      %\fontseries{m}\fontshape{sc}\selectfont
      %C) Fuer ganz "normale" Schrift:
      %\normalfont
      %
      \Large \@chapapp{} \thechapter
      \par\nobreak\vspace*{10\p@}%
      \interlinepenalty\@M
      {\Huge\bfseries\baselineskip3ex
        %Fuer Kapitaelchen folgende Zeile aktivieren:
        %\fontseries{m}\fontshape{sc}\selectfont
        #1\par\nobreak}
      \vspace*{10\p@}%
      \makebox[\textwidth]{\hrulefill}%    \hrulefill alone does not work
      \par\nobreak
      \vskip 40\p@
    }}

  %Fuer Kapitel ohne Kapitelnummer (z.B. Inhaltsverzeichnis)
  \def\@makeschapterhead#1{%
    \vspace*{10\p@}%
    {\parindent \z@ \raggedright \reset@font
      \normalfont \vphantom{\@chapapp{} \thechapter}
      \par\nobreak\vspace*{10\p@}%
      \interlinepenalty\@M
      {\Huge \bfseries %
        %Default-Schrift: Serifenhaft (gut fuer englische Dokumente)
        %A) Fuer serifenlose Schrift folgende Zeile aktivieren:
        \fontfamily{phv}\selectfont
        %B) Fuer Kapitaelchen folgende Zeile aktivieren:
        %\fontseries{m}\fontshape{sc}\selectfont
        #1\par\nobreak}
      \vspace*{10\p@}%
      \makebox[\textwidth]{\hrulefill}%    \hrulefill does not work
      \par\nobreak
      \vskip 40\p@
    }}
  %
  \makeatother
\fi


% DE: Minitoc-Einstellungen
%\dominitoc
%\renewcommand{\mtctitle}{Inhaltsverzeichnis dieses Kapitels}


% EN: Nicer paragraph line placement:
%     - Disable single lines at the start of a paragraph (Schusterjungen)
%     - Disable single lines at the end of a paragraph (Hurenkinder)
%     Normally, this is clubpenalty and widowpenalty, but using a package, it feels more non-hacky
\usepackage[all,defaultlines=3]{nowidow}
%
\displaywidowpenalty = 10000


% EN: Try to get rid of "overfull hbox" things and let the text flow better
%     See also
%       - http://groups.google.de/group/de.comp.text.tex/browse_thread/thread/f97da71d90442816/f5da290593fd647e?lnk=st&q=tolerance+emergencystretch&rnum=5&hl=de#f5da290593fd647e
%       - http://www.tex.ac.uk/cgi-bin/texfaq2html?label=overfull
\tolerance=2000
%
% EN: This could be increased to 20pt
\setlength{\emergencystretch}{3pt}
%
% EN: Suppress hbox warnings if less than 1pt
\setlength{\hfuzz}{1pt}


% EN: Fix names for algorithms in German
% DE: fuer algorithm.sty: - falls Deutsch und nicht Englisch.
\ifdeutsch
  \floatname{algorithm}{Algorithmus}
  \renewcommand{\listalgorithmname}{Verzeichnis der Algorithmen}
\fi




% Float-placements - http://dcwww.camd.dtu.dk/~schiotz/comp/LatexTips/LatexTips.html#figplacement
% and http://people.cs.uu.nl/piet/floats/node1.html
\renewcommand{\topfraction}{0.85}
\renewcommand{\bottomfraction}{0.95}
\renewcommand{\textfraction}{0.1}
\renewcommand{\floatpagefraction}{0.75}
%\setcounter{totalnumber}{5}

% EN: ensure that floats covering a whole page are placed at the top of the page
%    see http://tex.stackexchange.com/a/28565/9075
\makeatletter
\setlength{\@fptop}{0pt}
\setlength{\@fpbot}{0pt plus 1fil}
\makeatother



% EN: Margins
% DE: Ränder
%     Viele Moeglichkeiten, die Raender im Dokument einzustellen.
%
%     Satzspiegel neu berechnen. Dokumentation dazu ist in "scrguide.pdf" von KOMA-Skript zu finden
%     Optionen werden bei \documentclass[] in ausarbeitung.tex mitgegeben.
% \typearea[current]{current} %neu berechnen, da neue Schrift eingebunden

%\usepackage{a4}
%\usepackage{a4wide}
%\areaset{170mm}{277mm} %a4:29,7hochx21mbreit

%Wer die Masse direkt eingeben moechte:
%Bei diesem Beispiel wird die Regel nicht beachtet, dass der innere Rand halb so gross wie der aussere Rand und der obere Rand halb so gross wie der untere Rand sein sollte
%\usepackage[inner=2.5cm, outer=2.5cm, includefoot, top=3cm, bottom=1.5cm]{geometry}

% EN: Package geometry to enlarge on page
%
%     Normally, geometry should not be used as the typearea package calculates the margins perfectly for printing
%     However, we want better screen-readable documents where the content does not "jump"
%     Thus, we fix the margins left and right to the same value
%
%     Source: http://www.howtotex.com/tips-tricks/change-margins-of-a-single-page/
%
\usepackage[
  left=3cm,right=3cm,top=2.5cm,bottom=2.5cm,
  headsep=18pt,
  footskip=30pt,
  includehead,
  includefoot
]{geometry}


% EN: Provides todo notes
% DE: schoene TODOs
\ifdeutsch
  \usepackage[colorinlistoftodos,ngerman]{todonotes}
\else
  \usepackage[colorinlistoftodos]{todonotes}
\fi
\setlength{\marginparwidth}{2,5cm}

\let\xtodo\todo
\renewcommand{\todo}[1]{\xtodo[inline,color=black!5]{#1}}
\newcommand{\utodo}[1]{\xtodo[inline,color=green!5]{#1}}
\newcommand{\itodo}[1]{\xtodo[inline]{#1}}


% EN: Enable footnotes in tables.
%     This package supersedes the 1997 package "footnote"
\usepackage{footnotehyper}
% TODO: The footnotehyper author recommends to enclose the respective area with \begin{savenotes} ... \end{savenotes}
\makesavenoteenv{tabular}
\makesavenoteenv{table}
% Reuse of footnotes, see http://tex.stackexchange.com/questions/10102/multiple-references-to-the-same-footnote-with-hyperref-support-is-there-a-bett
\crefformat{footnote}{#2\footnotemark[#1]#3}


% EN: pgfplots (optional if the package is installed)
%     PGFPlots draws high-qual­ity func­tion plots in nor­mal or log­a­rith­mic scal­ing
\IfFileExists{pgfplots.sty}{
  \usepackage{pgfplots}
  % EN: highest version supported by overleaf as of 2018-03-16
  \pgfplotsset{compat=1.14}
}{}


% EN: pgfplotstable (optional if the package is installed)
%     PGFPlots generates tables from CSV files
\IfFileExists{pgfplotstable.sty}{
  \usepackage{pgfplotstable}
}{}


% EN: Package for creating graphics programmatically
\usepackage{tikz}


% EN: Package for creating uml diagramms
\usepackage{tikz-uml}


% EN: Forest: apgf/TikZ-based package for drawing linguistic trees - https://ctan.org/pkg/forest
\usepackage{forest}


% EN: Enable PlantUML listings in the environment "plantuml"
\IfFileExists{plantuml.sty}{
  \usepackage[output=latex]{plantuml}
}{}


% EN: Layout: bottoms of pages not aligned with each other
% DE: Der untere Rand darf "flattern"
\raggedbottom


% DE: Wie tief wird das Inhaltsverzeichnis aufgeschlüsselt
% 0 --\chapter
% 1 --\section % fuer kuerzeres Inhaltsverzeichnis verwenden - oder minitoc benutzen
% 2 --\subsection
% 3 --\subsubsection
% 4 --\paragraph
\setcounter{tocdepth}{2}


% EN: Fixes wrong spacing in the TOC.
%     Source: https://tex.stackexchange.com/a/33842/9075 -> comment by esdd
\RedeclareSectionCommand[tocnumwidth=2.8em]{section}


% DE: Angaben in die PDF-Infos uebernehmen
\makeatletter
\hypersetup{
  pdftitle={}, %Titel der Arbeit
  pdfauthor={}, %Author
  pdfkeywords={}, % CR-Klassifikation und ggf. weitere Stichworte
  pdfsubject={}
}
\makeatother


% EN: Higher compression of the output PDF
\pdfcompresslevel=9


% EN: Required for a recent version of komascript, as some packages are not as compatible with KOMAScript as they should be
%     Has to be loaded at the *very* end, so we use "\AtEndPreamble" by etoolsbox
\usepackage{etoolbox}
\AtEndPreamble{\usepackage{scrhack}}


% EN: Provide tables over multiple pages
\usepackage{longtable}


% EN: Show LaTeX commands and their results in the document
%     Enables the command \PrintDemo
% See https://github.com/latextemplates/scientific-thesis-template/issues/82 for further discussion
\usepackage{latexdemo}


% DE: Fuer deutsche Texte: Weniger Silbentrennung, mehr Abstand zwischen den Woertern
\ifdeutsch
  \setlength{\emergencystretch}{3em} % Silbentrennung reduzieren durch mehr frei Raum zwischen den Worten
\fi



% DOCTITLE
\def\doctitle{Design und Implementierung einer Treiber-API für industrielle Kommunikation}

\title{\Huge\bfseries\doctitle\\[1em]\large Dokumentation}
\date{\today} %clear date
\def\docauthor{Jan Kristel}
\ifoot{\leftmark}
\cfoot{}
\ofoot{\pagemark}
\ohead{}


% {key}{Abkürzung}{Ausgeschrieben}
\newacronym{ram}{RAM}{Random Access Memory}
\newacronym{cpu}{CPU}{Central Processing Unit}
\newacronym{api}{API}{Application Programming Interface}
\newacronym{uart}{UART}{}
\newacronym{hal}{HAL}{Hardware Abstraction Layer}
\newacronym{mcu}{MCU}{Microcontroller Unit}



\begin{document}

\begin{titlepage}
%	\maketitle
	\begin{center}
	\vspace*{1cm}
		{\Huge\bfseries\doctitle\\[1em]\large Bachelorthesis}
	\vspace{1cm}
		\date{\today} %clear date

		\docauthor\\
		kristeja@hs-albsig.de / jan.kristel@ws-schaefer.com\\
		Matriktelnummer: 100662\\
		Hochschule Albstadt-Sigmaringen\\
		Technische Informatik (B. Eng.)\\
		
\vspace{8mm}

		Erstbetreunung: Prof. Dr. Joachim Gerlach\\
		gerlach@hs-albsig.de\\
		Hochschule Albstadt-Sigmaringen\\
		72458 Albstadt
		
\vspace{8mm}

		Zweitbetreuung: Michael Grathwohl (M.End.)\\
		michael.grathwohl@ws-schaefer.com\\
		Schaefer GmbH\\
		Winterlinger-Straße 4\\
		72488 Sigmaringen
	\end{center}

\vspace{8mm}

% TODO: Firmen Bild Schaefer
	\begin{tabular}{ ll }
		\includegraphics[width=0.5\textwidth]{logoHsAS.png}
		&
   		\includegraphics[width=0.5\textwidth]{logoHsAS.png}
	\end{tabular}

	

\end{titlepage}

\chapter*{Eigenständigkeiterklärung}
Hiermit erkläre ich, Jan Kristel, Matrikel-Nr. 100662, dass diese Bachelorthesis auf meinen eigenen Leistungen beruht. Insbesondere erkläre ich, dass:

\begin{itemize}
	\item ich diese Bachelorthesis selbstständig ohne unzulässige fremde Hilfe erstellt haben,
	\item ich die Verwendung aller Quellen klar und korrekt angegeben habe und aus anderen Quellen entnommene Zitate eindeutig als solche gekennzeichnet habe,
	\item ich aus anderen/quelle entnommene Gedanken, Ideen, Bilder,Zeichnungen und Algorithmen, entsprechend der wissenschaftlichen Praxis gekennzeichnet habe,
	\item ich außer den angegebenen Quellen und Hilfsmitteln keine weiteren Quellen und Hilfsmittel zur Erstellung dieses Berichts verwendet habe und
	\item ich diese Bachelorthesis bisher in gleicher oder ähnlicher Form keiner anderen Prüfungsbehörde vorgelegt oder veröffentlich habe.
\end{itemize}


\begin{flushleft}
	\makebox[.4\textwidth]{Sigmaringen-Laiz, den \today}\\
\vspace{5mm}
	\makebox[.4\textwidth]{\hrulefill}\hfill\\
	\makebox[.4\textwidth][l]{\MakeUppercase{Jan Kristel}}
\end{flushleft}

\begin{abstract}
	
	
	
	Microcontroller weisen hinsichtlich ihrer Architektur, des Befehlssatz,  der Taktfrequenz, des verfügbaren Speichers, der Peripherie und weitererEigenschaften relevante Unterschiede auf.
	Die Aufgabe des Embedded-Softwareentwicklers besteht demnach darin, Hardware auszuwählen, die die Rahmenbedingungen des geplanten Einsatzes erfüllt.
	Darüber hinaus ist die Bereitstellung geeigneter Treiber essenziell, um eine optimale Ansteuerung der Hardware zu gewährleisten.
	
	In der vorliegenden Arbeit werden bewährte Methoden (Best Practices) zur Erstellung einer plattformunabhängigen Treiber-API für Microcontroller, die in der Embedded-Softwareentwicklung wiederverwendbar ist, untersucht
	Es wird analysiert, mit welchen Techniken verschiedene Treiber und Bibliotheken integriert und wie diese unterschiedlichen Hardwarekonfigurationen bereitgestellt werden.
	Dabei wird auch betrachtet, welche Auswirkungen unterschiedliche Prozessorarchitekturen auf die Umsetzung einer eigenen Treiberbibliothek haben.
	Diese integriert vorhandene Treiber, ersetzt hardwarespezifische Funktionen durch abstrahierte Schnittstellen und ermöglicht dadurch die Wiederverwendbarkeit der Applikation auf verschiedenen Hardwareplattformen – ohne dass eine Neuimplementierung erforderlich ist.

	Der modulare Aufbau des Projekts, das durch die Verwendung von Open-Source-Tools realisiert wurde, erlaubt eine flexible Erweiterung und kontinuierliche Optimierung.
	
	
	
\end{abstract} 

\chapter*{Vorwort}
Die vorliegende Bachelorarbeit mit dem Titel "Deign und Implementierung einer Treiber-API für industrielle Kommunikation" wurde als Abschlussarbeit des Studiums der Technischen Informatik in den Schwerpunkten Cyber-Physical-Systems and Security und Application Development\\ (StuPO 22.2) verfasst.

Der Inhalt der Arbeit wurde in Zusammenarbeit mit der Firma Schaefer GmbH in Sigmaringen-Laiz erarbeitet und dokumentiert.
Ziel der Thesis war es, eine 

	\tableofcontents
	\listoffigures
	\addcontentsline{toc}{chapter}{Abbildungsverzeichnis}
	\listoftables
	\addcontentsline{toc}{chapter}{Tabellenverzeichnis}
	\lstlistoflistings
	\addcontentsline{toc}{chapter}{Codeverzeichnis}
%	\chapter*{Abkürzungsverzeichnis}
%	\printacronyms	
	\printnoidxglossary[type=\acronymtype]
	\addcontentsline{toc}{chapter}{Abkürzungsverzeichnis}

% TODO: Überall Quellenverweise einfügen. Dort woher ich das Wissen habe, auf die Quelle verweisen/verlinken.
\chapter{Einleitung}


In der heutigen digitalen Welt spielen Programmierschnittstellen eine zentrale Rolle bei der Entwicklung verteilter, modularer und skalierbarer Softwaresysteme. 

Diese APIs ermöglichen die strukturierte Kommunikation zwischen Softwarekomponenten über definierte Protokolle und Schnittstellen und abstrahieren dabei komplexe Funktionen hinter\\ einfachen Aufrufen.
Während die Nutzung von APIs im Web- und Cloud-Umfeld bereits als etablierter Standard betrachtet werden kann, gewinnt diese Technologie auch in der Embedded-Entwicklung zunehmend an Relevanz.
Insbesondere im Bereich der Microkontroller tragen APIs zur Wiederverwendbarkeit, Portabilität und Wartbarkeit von Software bei.
Die zunehmende Komplexität eingebetteter Systeme sowie die Anforderungen an die Zusammenarbeit mit anderen Systemen, die Echtzeitfähigkeit und  die Ressourceneffizienz machen eine strukturierte Schnittstellendefinition unerlässlich.
Anwendungsprogrammierschnittstellen arbeiten in diesem Kontext als Vermittler zwischen der modularen Struktur von Applikation und Anwendungslogik, und der hardwarenahen Programmierung.
Zu typischen Anwendungsfällen zählen sowohl abstrahierte Zugriffe auf Peripheriekomponenten, Sensordaten, Kommunikationsschnittstellen als auch Betriebssystemdienste in Echtzeitbetriebssystemen (\gls{rtos}).	
		
\section{Problemstellung}
Mikrocontroller unterscheiden sich in vielerlei Hinsicht, unter anderem in ihrer Architektur, der verfügbaren Peripherie, dem Befehlssatz sowie in der Art und Weise, wie ihre Hardwarekomponenten über Register angesteuert werden.
Die Aufgabe dieser Register besteht in der Konfiguration und Steuerung grundlegender Funktionen, wie etwa der digitalen Ein- und Ausgänge, der Taktung, der Kommunikationsschnittstellen oder der Interrupt-Verwaltung.
Die konkrete Implementierung sowie die Adressierung und Bedeutung einzelner Bits und Bitfelder variieren jedoch von Hersteller zu Hersteller und sogar zwischen verschiedenen Serien desselben Herstellers erheblich.

Die signifikante Varianz in Bezug auf die Hardware-Komponenten führt dazu, dass Softwarelösungen und Applikationen, für jede neue Plattform entweder vollständig neu entwickelt oder zumindest aufwendig angepasst werden müssen.
Obwohl \gls{hals} eine gewisse Erleichterung bei der Entwicklung bieten, resultieren daraus gleichzeitig starke Bindungen an die zugrunde liegende Hardwareplattform.

Die wiederholte Implementierung oder Integration plattformspezifischer Klassen, Module und Bibliotheken erfordert einen erheblichen Aufwand in Bezug auf Entwicklungszeit und Ressourcen.
Zusätzlich ist eine Zunahme an Komplexität bei der Wartung sowie eine Erschwernis bezüglich der Wiederverwendbarkeit von Softwarekomponenten über verschiedene Projekte hinweg zu beobachten.

 In Anbetracht dessen ist die Entwicklung einer abstrahierten, plattformübergreifenden Architektur erforderlich, die diese Herausforderungen adressiert und eine einheitliche, modulare Schnittstelle für die Treiberauswahl bereitstellt.



\section{Motivation}
In der Embedded-Entwicklung stellt die effiziente und wartbare Bereitstellung von Software für eine wachsende Zahl unterschiedlicher Mikrocontroller-Plattformen eine zunehmende Herausforderung dar.
Die Vielzahl verfügbarer Mikrocontroller mit unterschiedlichen Architekturen, Peripheriekomponenten und Entwicklungsumgebungen führt zu einem hohen Aufwand bei der Anpassung und Pflege von Software und Applikationen.



\section{Ablauf}
Im Kapitel Aufgabenstellung
	\begin{itemize}
		\item Klärung der genauen Aufgabenstellung,
		\item Anschauen welche Werkzeuge verwendet werden,
		\item Ab wann die Aufgabe erfüllt ist.
	\end{itemize}
	
Danach geht man über die Grundlagen
	\begin{itemize}
		\item Hardwareentwicklung
		\begin{itemize}
			\item Ports
			\item Register
			\item Funktionen
			\item . . .
		\end{itemize}
		\item 
	\end{itemize}

Mit dem Grundlagenwissen wird im nächsten Kapitel sich der aktuelle Stand der Technik angeschaut.
Dabei wird sich angeschaut welche relevanten Lösungen und Ansätze es bereits gibt und diese mit einander verglichen.
Warum sich für diesen Ansatz entschieden wurden?

Danach geht es im Hauptteil der Arbeit um die Umsetzung des API mit eigenen Anpassungen und Erweiterungen.









	
	
\chapter{Aufgabenstellung}
Das Ziel dieser Arbeit ist es die Basis einer \gls{api}-Bibliothek zu erstellen, mit der je nach Zielplattform die richtigen/passenden Treiber integriert werden können.
Dafür muss 
Diese soll erste grundlegenden Funktionen für GPIO Input und Output, SPI, CAN und UART enthalten. 
Auf diese Weise kann die Funktionsweise für generelles Lesen und Schreiben und die Kommunikation über Busse getestet werden.
\\

\textbf{Warum für diese Entschieden?}

CAN damit ein anderes Projekt direkt integriert werden kann.
GPIO für die simpelste Art für Lesen und Schreiben

SPI und UART um Kommunikation via Bus zu testen




\section{Rahmenbedingungen}
Werkzeuge:
\begin{itemize}
	\item STM32CuebeIDE $\rightarrow$ VSCode
	\item C $\rightarrow$ C++
	\item CMake
	\item Linker-Files
\end{itemize}

VSCode Erweiterungen:
\begin{itemize}
	\item by franneck94
	\begin{itemize}
		\item C/C++ Runner v9.4.10
		\item C/C++ Config v6.3.0
	\end{itemize}
	\item Microsoft
	\begin{itemize}
		\item C/C++ Extension Pack v1.3.1
		\item C/C++ v1.25.3
		\item CMake Tools v1.20.53
	\end{itemize}
\end{itemize}

Verwendete Microcontroller:
\begin{itemize}
	\item STM32C032C6
	\item STm32G071RB
	\item STM32G0B1RE
	\item ESP32-C6 DevKitC-1
\end{itemize}


\section{Eigenschaften}
%Welche (architektonischen) Eigenschaft sind wichtig/sollen umgesetzt werden?
%\begin{itemize}
%	\item keine/geringen Redundanz $\rightarrow$ z.B. Klassen sollen nicht immer neu implementiert werden.
%	\item einfache Benutzung $\rightarrow$ damit auch zukünftige neue Mitarbeiter einen schnellen Einstieg und Verständnis für die Umgebung bekommen.
%	\item Skalierbarkeit $\rightarrow$ soll auf möglichst viele MCUs/Hardwareboards funktionieren/kompatibel sein.
%	\item Portabilität $\rightarrow$ mit Blick auf unterschiedliche Betriebssysteme (hier: Windows, Linux und MacOS), sollt die erstellte Library auf möglichst vielen bekannten Betriebssystemen laufen. Die damit verbundene Installation der benötigten Tools sollte dementsprechend dokumentiert sein.
%	\item Erweiterbarkeit $\rightarrow$ Leistungsstärkere MCUs bringen oft weitere Funktionen mit. Es muss einfach sein, die implementierten Klassen um diese neuen Funktionen zu erweitern.
%	\item Modularität $\rightarrow$ das Strukturieren der Library in klare Module hilft nicht nur der Trennung von Funktionen und dem damit gewonnen Überblick, sondern dient auch der Wartbarkeit, indem sie es ermöglicht Fehlerquellen schneller zu lokalisieren und diese dann zu beheben.
%	\item Effizienz $\rightarrow$ die Ressourcen, die eine Microcontroller mit bringt sind sehr begrenzt. So muss darauf geachtet werden, dass die Applikation und ihre Abhängigkeiten, z.B. externe Libraries nicht  groß werden und den gesamte Speicher einnehmen.
%\end{itemize}

Die Entwicklung einer plattformunabhängigen, wiederverwendbaren Treiber-API für Mikrocontroller stellt hohe Anforderungen an die Architektur der Softwarebibliothek.
Das Ziel besteht darin, eine Lösung zu schaffen, die sich durch eine geringe Redundanz auszeichnet. 
Die Konzeption von Klassen und Funktionen sollte derart erfolgen, dass eine erneute Implementierung für jede neue Plattform oder Anwendung nicht erforderlich ist.
Die Wiederverwendbarkeit zentraler Komponenten führt zu einer Reduktion des Entwicklungsaufwands und einer Erhöhung der Konsistenz im Code.

Ein weiteres zentrales Anliegen ist die einfache Benutzbarkeit. 
Die API ist so zu gestalten, dass eine effiziente Nutzung gewährleistet ist. 
Dies fördert nicht nur die Effizienz in der Erstellung neuer Applikationen, sondern erleichtert auch langfristig die Wartung und Weiterentwicklung der Software.

Im Sinne der Skalierbarkeit wird angestrebt, die Lösung auf möglichst viele Mikrocontroller-Architekturen und Hardwareplattformen anwendbar zu machen.
Die Vielfalt verfügbarer MCUs erfordert eine abstrahierte und flexibel erweiterbare Struktur, die die Integration neuer Plattformen mit minimalem Aufwand ermöglicht.

Auch die Portabilität spielt eine wichtige Rolle.
Die Bibliothek sollte nicht nur hardware-, sondern auch betriebssystemunabhängig konzipiert werden.
Aus diesem Grund wird bei der Entwicklung der Lösung darauf geachtet, dass diese erst unter Windows, später auch unter Linux und macOS einsetzbar ist.
Die Installation und Konfiguration der dafür benötigten Werkzeuge wird nachvollziehbar dokumentiert, um den Einstieg für die Nutzer zu erleichtern.

Darüber hinaus ist die Erweiterbarkeit ein wesentliches Architekturprinzip
Der Einsatz von leistungsstärkeren Mikrocontrollern hängt in der Regel mit einer Erweiterung der Funktionalitäten zusammen, die in die bestehenden Treiber- und API-Strukturen integriert werden müssen.
Daher wird großer Wert auf eine modulare und offen gestaltete Architektur gelegt, die neue Features ohne grundlegende Umbauten aufnehmen kann.

Modularität trägt wesentlich zur Übersichtlichkeit und Wartbarkeit des Systems bei. 
Eine saubere Trennung funktionaler Einheiten ermöglicht eine schnellere Lokalisierung und Behebung von Fehlern, was wiederum die langfristige Pflege und Weiterentwicklung der Software erleichtert.

Schließlich ist auch die Effizienz ein kritischer Aspekt.
Da Mikrocontroller in der Regel nur über begrenzte Ressourcen verfügen, ist es essenziell, dass die Bibliothek möglichst kompakt und ressourcenschonend implementiert wird. 
Externe Abhängigkeiten werden bewusst auf ein Minimum reduziert, um Speicherplatz zu sparen und unnötige Komplexität zu vermeiden.

Diese architektonischen Prinzipien bilden die Grundlage für die Konzeption und Umsetzung der in dieser Arbeit vorgestellten Treiber-API.

Wie wird der jeweilige Punkt umgesetzt?

Welche Tools werden benutzt/eignen sich besonders für die Umsetzung?
Welche Tools eignen sich für welchen Arbeitsschritt?

Warum wird etwas gerade auf diese Weise umgesetzt?


\section{Anforderungen an die Lösung}
Erfolgreiche Implementierung der Grundfunktionen 
von GPIO, SPI, UART, CAN. Das beinhaltet die Kommunikation über diese Technologien, d.h. Lesen und Schreiben.





\chapter{Technische Grundlagen}

Die Informatik umfasst eine Vielzahl unterschiedlicher Fachgebiete mit teils stark variierenden Schwerpunkten. 
Dazu zählen unter anderem die Web- und Anwendungsentwicklung sowie der Bereich der IT-Sicherheit und viele weitere Disziplinen. 
Im Rahmen dieser Arbeit liegt der Fokus auf dem speziellen Teilbereich der Embedded-Softwareentwicklung.

In diesem Kapitel werden die grundlegenden fachlichen und technischen Konzepte vermittelt, die zum Verständnis der weiteren Inhalte erforderlich sind.
% Ablauf des Kapitels
Zu Beginn wird eine Einführung in das Themenfeld der Embedded-Systeme gegeben, um ein klares Verständnis dafür zu schaffen, welche Unterschiede diesen Bereich kennzeichnen und wie er sich von anderen Teilgebieten der Informatik unterscheidet.
Darauffolgend werden zentrale Begriffe und Konzepte erläutert, die in der Embedded-Entwicklung eine signifikante Rolle spielen, wie beispielsweise Register, Ports, Peripherieansteuerung und hardwarenahe Programmierung.
Darüber hinaus wird technisches Hintergrundwissen vermittelt, das für das Verständnis der späteren Implementierungsschritte und der Architekturentscheidungen von Relevanz ist.

%Das Ziel dieses Kapitels besteht darin, eine solide Wissensbasis zu schaffen, auf der die Analyse bestehender Lösungen sowie die Entwicklung einer eigenen Treiber-API aufbauen können.
\section{Hardware}
\subsection{Eingebettete Systeme}
Bevor auf die Entwicklung eingebetteter Systeme eingegangen werden kann, ist zunächst zu klären, worum es sich bei diesen Systemen handelt.
Der Begriff \emph{Embedded System} (deutsch: eingebettetes System) bezeichnet ein Computersystem, das Hardware \textbf{und} Software in sich kombiniert und fest in einen übergeordneten technischen Kontext integriert ist. 
Typischerweise handelt es sich dabei um Maschinen, Geräte oder Anlagen, in denen das eingebettete System spezifische Steuerungs-, Regelungs- oder Datenverarbeitungsaufgaben übernimmt.
Ein wesentliches Merkmal eingebetteter Systeme besteht darin, dass sie nicht als eigenständige Recheneinheiten agieren, sondern als integraler Bestandteil eines übergeordneten Gesamtsystems dienen.
In der Regel operieren sie im Hintergrund und sind nicht direkt mit den Benutzern verbunden. In einigen Fällen erfolgt die Interaktion automatisch, in anderen durch Eingaben des Nutzers.

%\paragraph{Definition:}
%Ein Embedded System ist ein spezialisiertes, in sich geschlossenes Computersystem, das für eine klar definierte Aufgabe innerhalb eines übergeordneten technischen Systems konzipiert wurde.
%
%\vspace{6 mm}

Die Entwicklung von Software für eingebettete Systeme ist mit besonderen Anforderungen verbunden, die sich signifikant von denen unterscheiden, die etwa in der Web- oder Anwendungsentwicklung üblich sind.
Es ist von besonderer Bedeutung, hardwarenahe Aspekte zu berücksichtigen, da die Software unmittelbar mit der zugrunde liegenden Microcontroller-Hardware interagiert.
Ein zentraler Aspekt dabei ist die Integration geeigneter Treiber für die jeweilige Microcontroller-Architektur.
\begin{figure}[H]
	\includegraphics[width=\textwidth]{Pics/embedded_layer_architecture.png}
	\caption{Allgemeine Darstellung der Schichtenarchitektur.\cite{RichardsFord2020}}
	\label{fig:embedded_layer_architecture}
\end{figure}
Die betreffenden Treiber beinhalten Funktionen, welche den Zugriff auf die Hardware mittels sogenannter Register erlauben.
Register sind spezifische Speicherbereiche innerhalb des Microcontrollers, welche eine unmittelbare Manipulation des Hardware-Verhaltens ermöglichen.
Durch das gezielte Setzen oder Auslesen einzelner Bits in diesen Registern ist es möglich, beispielsweise Sensorwerte zu erfassen (z. B. das Drücken eines Tasters) oder Ausgaben zu erzeugen (z. B. das Anzeigen eines Textes auf einem Display).
% Memory Mapped IO
Der Zugriff auf diese Register erfolgt typischerweise über den Mechanismus des \gls{mmio}. 
In diesem Prozess werden die Peripherieregister in denselben Adressraum eingebunden wie der Arbeitsspeicher (RAM). 
Für den Prozessor ist es somit irrelevant, ob er Daten im RAM oder in einem Peripherieregister liest oder schreibt – beide Zugriffe erfolgen über die gleichen Speicherbefehle. 
Der wesentliche Unterschied zwischen den beiden Verfahren liegt darin, dass ein Zugriff auf ein Register nicht nur die Daten verändert, sondern auch das Verhalten der Hardware steuert oder deren aktuellen Status zurückgibt.
\gls{mmio} zeichnet sich zum einen durch die direkte Hardwaresteuerung aus.
Jeder Registerzugriff löst eine konkrete Aktion aus.
Ein Nachteil besteht in den Nebenwirkungen, die beispielsweise das automatische Löschen von Statusbits beim Lesen mit sich bringen.
% Caching und Spekulationsmechanismen
Ein weiteres Problem ist das fehlende Caching, da Peripheriebereiche von Cache- und Spekulationsmechanismen ausgeschlossen werden müssen. 
Diese Peripheriebereiche liegen im \gls{mmio} Bereich.
Damit dürfen diese nicht über einen Cache gelesen oder beschrieben werden, da der Prozessor sonst mit veralteten oder falschen Werten arbeiten oder Schreibzugriffe nicht korrekt bei der Hardware ankommen würden.
Darüber hinaus setzen moderne Prozessoren häufig sogenannte Spekulationsmechanismen ein, bei denen Befehle vorab ausgeführt werden, um die Leistung zu steigern.
Im Falle eines spekulativen Zugriffs einer CPU auf Speicherzugriffe im MMIO-Bereich besteht die Möglichkeit einer ungewollten Veränderung der Registerzustände, beispielsweise durch das vorzeitige Löschen von Statusbits.
Um derartige Seiteneffekte zu unterbinden, werden Speicherbereiche für Peripherie explizit als \texttt{device memory} oder \texttt{strongly ordered} markiert, sodass spekulative Zugriffe auf diese Bereiche unterbunden werden.
% Keyword: volatile
Zudem besteht die Notwendigkeit, in höheren Programmiersprachen wie C oder C++ Registerzugriffe als \texttt{volatile} zu deklarieren, um unzulässige Compileroptimierungen zu verhindern.
Das Schlüsselwort \texttt{volatile} weist den Compiler explizit darauf hin, dass der Wert einer Variablen oder eines Speicherbereichs sich jederzeit außerhalb der Programmlogik ändern kann. Ein Beispiel für eine solche Änderung sind Hardwareereignisse oder Interrupts.
Dadurch wird verhindert, dass Lese- oder Schreibzugriffe wegoptimiert, zwischengespeichert oder in ihrer Reihenfolge verändert werden.
Insbesondere bei Zugriffen auf Speicherbereiche im Kontext von \gls{mmio} ist dies von essenzieller Bedeutung, da jeder Zugriff direkt mit der Hardware interagiert und somit zwingend ausgeführt werden muss, um den korrekten Zustand der Peripherie sicherzustellen.


\subsection{Microcontroller Unit (MCU)}
Ein Microcontroller ist ein vollständig auf einem einzigen Chip realisierter Mikrocomputer, der neben dem eigentlichen Prozessor (CPU) auch sämtliche für den Betrieb notwendigen Komponenten integriert. 
Zu den Komponenten eines solchen Systems zählen in der Regel Programmspeicher (Flash), Datenspeicher (\gls{ram}), digitale Ein- und Ausgänge (\gls{gpio}), Timer, Kommunikationsschnittstellen (wie \gls{uart}, \gls{spi}, \gls{i2c}, \gls{can}) sowie in vielen Fällen analoge Peripheriekomponenten wie Analog/Digital-Wandler oder Pulsweitenmodulation-Einheiten.

Microcontroller werden für spezifische Steuerungs- und Regelungsaufgaben konzipiert und finden typischerweise Anwendung in eingebetteten Systemen, wie beispielsweise Haushaltsgeräten, Fahrzeugsteuerungen, Industrieanlagen oder IoT-Geräten. 
Die Geräte zeichnen sich durch einen geringen Energieverbrauch, eine kompakte Bauform, niedrige Kosten und eine direkte Hardwareansteuerung aus. 
Im Vergleich zu Mikroprozessoren sind für den Grundbetrieb von Microcontrollern keine externen Komponenten erforderlich, was besonders kompakte und zuverlässige Systemlösungen ermöglicht.


\subsection{Peripherie}
Unter dem Begriff der \emph{Peripherie} versteht man im Kontext der Embedded-Softwareentwicklung sämtliche Ein- und Ausgabeschnittstellen, die eine Interaktion des Microcontrollers mit seiner Umwelt ermöglichen.
Peripheriegeräte stellen die Verbindung zwischen der digitalen Rechenlogik des Microcontrollers und der realen Welt her.
Sie ermöglichen die Erfassung, Verarbeitung und Ausgabe physikalischer Signale wie Temperatur, Licht oder der Betätigung eines Tasters.
Ein moderner Microcontroller, wie etwa ein STM32, ist bereits mit einer Vielzahl an integrierten Peripherieeinheiten ausgestattet, darunter digitale Ein-/Ausgänge (GPIOs), serielle Kommunikationsschnittstellen (UART, SPI, I2C, CAN), analoge Wandler (ADC, DAC), Timer oder PWM-Module (Pulsweitenmodulation). 
Die als \emph{On-Chip} bezeichneten Komponenten sind integraler Bestandteil des Microcontrollers und können über zugehörige Register programmiert und gesteuert werden.
Zusätzlich zur integrierten Peripherie besteht die Möglichkeit, über die physischen Pins des Microcontrollers auch externe Peripheriegeräte anzuschließen. 
Die Verbindung erfolgt in der Regel mittels Steckverbindungen, wie etwa Jumper-Kabeln, Steckbrücken, Pin-Headern oder speziellen Anschlussleisten auf Entwicklungsboards. 
In der Regel werden zu diesem Zweck Steckbretter (Breadboards) oder Lochrasterplatinen verwendet, um eine übersichtliche und flexible Verdrahtung zu gewährleisten. 
Externe Bauteile, wie etwa Sensoren (Temperatursensor), Aktoren (LED), Displays oder Speicherbausteine, werden über gängige Schnittstellen wie I2C, SPI, UART oder digitale GPIOs mit dem Microcontroller verbunden.
Die Kommunikation mit externen Geräten wird durch die Peripheriemodule des Microcontrollers realisiert. 
Für den zuverlässigen Betrieb sind in der Regel spezifische Softwaretreiber erforderlich, die die Initialisierung, Datenübertragung und gegebenenfalls die Fehlerbehandlung übernehmen.

% TODO: chap3 Peripherie: Parallel/Seriell; synchron/asynchron; für alle nochmal drüber gehen
\subsubsection*{General Purpose Input Output}
Der Begriff "General Purpose Input/Output" (GPIO) bezeichnet universelle, digitale Pins eines Microcontrollers, die flexibel als Eingang oder Ausgang konfiguriert werden können.
Sie stellen die grundlegendste Form der Interaktion mit der Außenwelt dar und gestatten die Erfassung externer digitaler Signale, z.B. von Tastern oder Sensoren, sowie die Erzeugung entsprechender Signale etwa zur Steuerung von LEDs oder Relais.
Grundsätzlich können GPIOs flexibel als Eingang oder Ausgang verwendet werden.
Typischerweise erfolgt die Konfiguration solcher Embedded-Systeme statisch während der Initialisierung, entweder automatisch durch Codegeneratoren wie STM32CubeMX oder manuell in der Startkonfiguration der Firmware.
Obwohl eine Änderung der GPIO-Funktionalität zur Laufzeit technisch möglich wäre, wird dies in der Praxis häufig vermieden, um ein deterministisches und stabiles Systemverhalten zu gewährleisten.
In der praktischen Anwendung bilden sie die Grundlage für einfache Steuerungs- und Überwachungsaufgaben und sind daher von zentraler Bedeutung für die hardwarenahe Embedded-Programmierung.

\subsubsection*{Serial Peripheral Interface}
Die Schnittstellen des \emph{Serial Peripheral Interface} (SPI) ist ein synchrones, serielles Kommunikationsprotokoll, das insbesondere für die schnelle und effiziente Datenübertragung über kurze Distanz zwischen einem Master- und einem oder mehreren Slave-Geräten eingesetzt wird. 
Die primäre Aufgabe des Protokolls besteht in der Verbindung von \gls{mcu}s mit integrierten oder externen Komponenten, zu denen unteranderem  Sensoren, Speicher, Aktoren sowie Displays zählen.
SPI arbeitet synchron, d.h. Sender und Empfänger teilen sich ein gemeinsames Taktsignal.
Der Master ist derjenige, der diesen Takt vorgibt und bereitgestellt.
Dadurch wird eine präzise, zeit-sensitive Übertragung ermöglicht. 
Die zentrale Eigenschaft von \gls{spi}, die das gleichzeitige Senden und Empfangen ermöglicht ist die Unterstützung der Voll-Duplex-Kommunikation.
% TODO: SPI: Quellenverweise einfügen.
Der \gls{spi}-Bus verwendet meistens vier physikalische Leitungen:
\begin{itemize}
	\item \gls{mosi} / \gls{copi} für die Kommunikation vom Master zu den Peripheriegeräten (Slaves).
	\item \gls{miso} / \gls{cipo} für die Kommunikation von den Peripheriegeräten zum Master.
	\item \gls{ss} / \gls{cs} für die Auswahl des gewünschten Peripheriegerätes.
	\item \gls{sclk} als Taktleitung, die den vom Master vorgegebenen Takt enthält.
\end{itemize}

In der Regel dient der Microcontroller als Master, der den Datenfluss steuert.
Mittels des Slave-Signals ist es der \gls{mcu} möglich, gezielt Slaves anzusprechen.
Dabei ist darauf zu achten, dass jeweils nur ein Slave die Kommunikation aktiv durchführen darf, um eine Kollision auf Bus zu vermeiden.

\gls{spi} zeichnet sich im Vergleich zu anderen seriellen Protokollen wie \gls{i2c} durch eine vereinfachte Implementierung und eine deutlich höhere Datenübertragungsrate aus. 
Allerdings fehlen eine standardisierte Adressierung und Fehlerprüfung, was den Einsatz auf kurze Distanzen und überschaubare Topologien begrenzt. 

% UART
% TODO: UART Löschen da nicht verwendet.
%\subsubsection*{Universal Asynchronous Receiver Transmitter}
%Der \emph{Universal Asynchronous Receiver Transmitter} (UART) ist ein asynchrones Kommunikationsprotokoll, das insbesondere für die serielle, asynchrone Punkt-zu-Punkt-Kommunikation zwischen zwei Geräten eingesetzt wird. 
%Das Protokoll eignet sich für verschiedene Anwendungsbereiche, darunter als Debugging-Schnittstelle, der Kommunikation von Sensoren, GPS-Modulen sowie die Kommunikation mit Computern über USB-zu-Seriell-Wandler. 
%Im Gegensatz zu synchronen Schnittstellen wie SPI ist für UART kein gemeinsames Taktsignal erforderlich.
%Die Datenübertragung passiert hier asynchron über zwei Leitungen: eine für das Senden (Transmitter TX) und eine für das Empfangen (Receiver RX).
%Die Synchronisation basiert auf einer zuvor festgelegten Baudrate (Bits pro Sekunde), die von beiden Geräten unabhängig voneinander eingehalten werden muss.
%Die Kommunikation, d.h. die Datenübertragung erfolgt in sogenannten Frames. Ein typisches \gls{uart}-Frame besteht aus:
%\begin{itemize}
%	\item \textbf{Startbit}, das den Begin eines Datenframes signalisiert,
%	\item \textbf{Datenframe}, bestehend aus fünf bis acht Bits,
%	\item \textbf{Paritätsbit}, das einer einfacheren Fehlererkennung dient und
%	\item \textbf{Stopbit}, das das Ende des Datenframes markiert.
%\end{itemize}
%
%In Abhängigkeit von der Implementierung unterstützt UART Simplex-, Halbduplex- und Vollduplex-Kommunikation. 
%In einer Vielzahl von Microcontrollern ist UART als Hardwaremodul integriert, wodurch die serielle Kommunikation effizient und mit minimalem Softwareaufwand realisiert werden kann. 
%Dennoch erfordert die korrekte Konfiguration – insbesondere die Wahl der Baudrate, des Paritätsmodus und der Anzahl von Stoppbits – besondere Sorgfalt, da Abweichungen zu Datenverlust oder Kommunikationsfehlern führen können.
%Ein weiterer Vorteil von UART ist seine Einfachheit in Aufbau und Handhabung: Es werden lediglich zwei Leitungen benötigt. 
%Die Kommunikation ist prinzipiell auf zwei Geräte beschränkt (Punkt-zu-Punkt-Verbindung), da UART keine native Unterstützung für Bussysteme mit mehreren Teilnehmern bietet.

\subsubsection*{Controller Area Network}
Das \emph{Controller Area Network} (CAN) ist ein robustes, serielles, asynchrones Bussystem, das insbesondere in der Automobilindustrie eine weite Verbreitung findet. 
Es ermöglicht eine zuverlässige Kommunikation zwischen mehreren Steuergeräten (Nodes), auch unter schwierigen elektromagnetischen Bedingungen. 
Der Einsatz von CAN in sicherheitskritischen Anwendungen beruht auf zwei wesentlichen Eigenschaften: 
\begin{itemize}
	\item der prioritätsbasierten Arbitrierung
	\item der integrierten Fehlererkennung
\end{itemize}
 
Diese Eigenschaften gewährleisten eine hohe Ausfallsicherheit.
Die CAN-Technologie basiert auf einem \emph{shared medium} mit Bus-Topologie, bei der alle Teilnehmer über zwei Leitungen mit einander verbunden sind.
Jede im CAN-Bus übertragene Nachricht ist durch eine eindeutige Identifikationsnummer gekennzeichnet, die sich auf die Art der übertragenen Information bezieht, beispielsweise Geschwindigkeit, Drehzahl oder Sensordaten. 
Für jede Identifikationsnummer ist ausschließlich ein Knoten zur Übertragung von Nachrichten mit dieser Kennung vorgesehen, um Konflikte im Bus zu vermeiden. 
Alle angeschlossenen Knoten sind prinzipiell in der Lage, diese Nachrichten zu empfangen, müssen dies jedoch nicht zwingend tun. 
Jeder Knoten entscheidet autonom, welche Nachrichten für ihn relevant sind, und verarbeitet ausschließlich diese. 
Ein Absender oder Empfänger ist in der Nachricht selbst nicht kodiert, sodass die Kommunikation vollständig nachrichtenbasiert und nicht adressbasiert erfolgt.
Es ist grundsätzlich möglich, dass mehrere Knoten auf dieselbe Nachricht reagieren.

Ein wesentliches Merkmal ist die prioritätsbasierte Arbitrierung. 
Jeder Knoten hat die Möglichkeit, eine Nachricht gleichzeitig zu senden. 
Das Protokoll verwendet ein bitweises Arbitrierungsverfahren. 
Nachrichten mit einer niedrigeren ID (höhere Priorität) durchdringen das System automatisch, ohne dass es zu Kollisionen oder Datenverlust kommt.
Dieses Verfahren zeichnet sich durch seine besondere Effizienz aus und ist in der Lage, Echtzeitanforderungen zu erfüllen.

Obwohl CAN asynchron arbeitet, d.h. jeder Knoten verfügt über einen eigenen Takt, erfolgt die Synchronisation der Kommunikation durch ein fein abgestimmtes Zeitraster innerhalb des CAN-Controllers. 
Die Bitzeit wird in sogenannte Zeitquanten (Time Quanta, TQ) unterteilt, deren Definition relativ zur eingestellten Bus-Baudrate erfolgt. 
Diese Zeitquanten lassen sich in mehrere Segmente unterteilen: 

\begin{itemize}
	\item Synchronisationssegment, das der Flankenerkennung dient, 
	\item das Propagationssegment, das die Signallaufzeiten berücksichtigt,
	\item die Segmente für die Phase 1 und Phase 2, die für die Feinjustierung des Abtastzeitpunkts verantwortlich sind.
\end{itemize}

Der Sample Point ist definiert als der Zeitpunkt, an dem der Wert des Bits für alle Knoten einheitlich abgetastet wird. 
Dieser befindet sich zwischen Phase 1 und Phase 2. 
Die interne Taktfrequenz der CPU beeinflusst dabei lediglich die Genauigkeit der Generierung der Zeitquanten, während die tatsächliche Bitdauer durch die konfigurierte \emph{Baudrate} des CAN-Busses bestimmt wird.


\section{Software}

\label{sec:architecture_design_pattern}
\subsection{Architektur- und Designmuster}

\subsubsection{Architekturmuster}
%Unter Architektur, speziell \emph{Softwarearchitektur} versteht man den Prozess eine allgemeine Struktur für ein Softwaresystem zu erstellen.
Helmut Balzert definiert den Begriff als ''eine strukturierte oder hierarchische Anordnung der Systemkomponenten sowie Beschreibung ihrer Beziehungen'' \cite{balzert2011softwaretechnik2}.
In diesem Prozess gilt es, die Komponenten in eine grobe (high-level) Gliederung zu bringen.
Im Kontext der Embedded Systeme und Entwicklung und speziell dieser Arbeit, wird sich primär mit der \emph{Schichtenarchitektur} befasst.

Bei diesem Architekturmuster wird das gesamte System in Schichten unterteilt, die  Handlungsbereiche darstellen.
Diese Schichten funktionieren so, dass sie nur mit der direkt anliegenden tieferen Schicht kommunizieren können.
Das bedeutet eine Schicht $n$ kann nur mit der Schicht $n-1$ kommunizieren und ist von dieser abhängig.
Schicht $n-1$ bietet dabei die entsprechenden Funktionen für Schicht $n$.
Umgekehrt gilt diese Abhängigkeit aber nicht.

Im Embedded Bereich lassen sich die Schichten wie folgt beschreiben:

\paragraph{Anwendungsschicht/Application Layer}
dient als oberste Schicht.
Diese besteht aus allen Dateien, Funktionen und Klassen, die nicht direkt mit den auf Hardwareebene liegenden Registern zu tun haben; so z.B. Hilfsfunktionen.
Stattdessen werden die Funktionen der nächst tieferen Schicht verwendet.

\paragraph{Mittelschicht/Middleware}
als optionale zweite Schicht, befasst sich mit möglichen Zusatzfunktionen wie USB, Netzwerkanschlüsse (WLAN), Bluetooth oder IoT (Internet of Things) oder \gls{api}-Funktionen.
Sie dient als verteilende Zwischenschicht zwischen der Programm und der Abstraktionsschicht der Hardware.

\paragraph{Betriebssystem} ist eine weitere optionale Schicht.
Optional in dem Sinn, das ein Embedded System nicht zwingend eine Betriebssystem benötigt.
Dann wird von Bare-Metal-Entwicklung gesprochen.
Ohne das Betriebssystem werden direkt die Pins, d.h. die Hardware angesprochen und programmiert; z.B. wenn in kleinem Schaltkreis nur ein Schalter, mit dem ein Signal ein oder ausgeschalten werden soll, und eine LED, die mit dem Schaltersignal leuchtet oder nicht, verbaut sind.
Das Programm befindet sich dann in einem sog. \textit{Superloop}, einer endlosen Schleife, in der alle Aufgaben des Systems sequentiell und wiederholt abgearbeitet werden, ohne dass ein Betriebssystem zur Ablaufsteuerung oder Taskverwaltung erforderlich ist.
Wird ein Betriebssystem eingesetzt bringt das funktionale Erweiterungen mit sich, wie Multitasking oder besseres Zeitmanagement.
Des weiteren muss bei mit einem OS (Operating System) auf die verfügbaren Ressourcen geachtet werden, da der Speicher bei Microcontrollern begrenzt ist.

\paragraph{Hardwareabstraktionsschicht} (\gls{hal}) befindet sich unter der Middleware bzw. unter dem Betriebssystem.
Gibt es keine zusätzlichen Funktionen in der Middlewareschicht und wird bare-metal entwickelt kann aus der Applikation direkt auf die hier gelagerten Funktionen zugreifen.
Wie dieser direkte Zugriff auf die Abstraktionsschicht aussieht ist in Code \cref{lst:stm32_mx_gpio_init}
zu sehen.
\clearpage

\begin{lstlisting}[language=C, caption={Funktion zur Initialisierung der GPIO-Pins aus einem STM32-Projekt.}, label={lst:stm32_mx_gpio_init}]
int main(void)
{
  HAL_Init();
  SystemClock_Config();
  MX_GPIO_Init();
  
  while (1){ ... } 
}

static void MX_GPIO_Init(void)
{
  GPIO_InitTypeDef GPIO_InitStruct = {0};

  /* GPIO Ports Clock Enable */
  __HAL_RCC_GPIOC_CLK_ENABLE();
  __HAL_RCC_GPIOF_CLK_ENABLE();
  __HAL_RCC_GPIOA_CLK_ENABLE();

  /*Configure GPIO pin Output Level */
  HAL_GPIO_WritePin(LEDextern_GPIO_Port, LEDextern_Pin, GPIO_PIN_RESET);

  /*Configure GPIO pin : LEDextern_Pin */
  GPIO_InitStruct.Pin = LEDextern_Pin;
  GPIO_InitStruct.Mode = GPIO_MODE_OUTPUT_PP;
  GPIO_InitStruct.Pull = GPIO_NOPULL;
  GPIO_InitStruct.Speed = GPIO_SPEED_FREQ_LOW;
  HAL_GPIO_Init(LEDextern_GPIO_Port, &GPIO_InitStruct);
}
\end{lstlisting}

Aus der \texttt{int main(void)}, der Hauptfunktion wird die Funktion \texttt{static MX\_GPIO\_Init(void)} zur Initialisierung der Pins aufgerufen.

In dieser Funktion wird erst eine Struktur \texttt{GPIO\_InitStruct} vom Typ \texttt{GPIO\_InitTypeDef} vorbereitet, indem alle Werte, die in der Struktur enthalten sind gleich $0$ gesetzt werden.
Danach werden für die verwendeten Ports die jeweiligen Clocks gestartet, damit jeder Port auch eine Takt hat;
gefolgt von dem Zurücksetzen des GPIO-Pins, damit dieser keine ungewünschten Werte ausgibt, die evtl. zuvor in dem Register standen.
Mit \texttt{ HAL\_GPIO\_WritePin(LEDextern\_GPIO\_Port, LEDextern\_Pin, GPIO\_PIN\_RESET);} wird sichergestellt, dass der Pin auf $0$ gesetzt ist.
Ab jetzt wird die vorbereitete Struktur mit Werten belegt.
Damit die Funktion \texttt{HAL\_GPIO\_Init()} weiss, welchen Pin sie initialisieren muss, bekommt die Struktur die einzelnen Eigenschaften des Pins übergeben.
So bestimmt 
\begin{itemize}
	\item \texttt{GPIO\_InitStruct.Pin} um welchen Pin es sich handelt, 
	\item \texttt{GPIO\_InitStruct.Mode} in welchem Modus der Pin operieren soll,
	\item \texttt{GPIO\_InitStruct.Pull} welche Art von internem Widerstande (Pull-Up, Pull-Down oder kein Pullwiderstand) verwendet werden soll,
	\item \texttt{GPIO\_InitStruct.Speed} mit welcher Schaltgeschwindigkeit der Pin arbeitet, d.h. wie schnell ein Flankenwechsel erfolgen darf und welche Anstiegszeit für die Signale zugelassen wird.
\end{itemize}

Am Ende dieser Funktion sieht man \texttt{HAL\_GPIO\_Init(GPIO\_Port, \&GPIO\_InitStruct)}.
Diese Funktion ist Teil der Hardwareabstraktionsschicht, auf die hier ohne weitere Zwischenschicht oder Betriebssystem zugegriffen wird.

\paragraph{Die Treiberschicht} 
ist die letzte Ebene vor der Hardwareschicht.
Diese Schicht arbeitet eng mit der Abstraktionsschicht zusammen.
Sie enthält neben den Low-Level-Treibern, die direkten Zugriff auf die Register haben, den in Assembler geschriebenen Startupcode und Initialisierungsroutinen.


\subsubsection{Designmuster} \label{chap3_2_1_designmuster}
% TODO: Designpattern: Nur für OOP?
Neben dem Architekturmuster, das für die Struktur des gesamten Projekts verantwortlich ist, stehen die \emph{Designmuster}.
Unter diesem Begriff versteht man das Designen von einzelnen Softwarekomponenten, wie diese aufgebaut sein sollen, wie sie miteinander kommunizieren, welche Eigenschaften sie haben und hilft dabei die Software zu implementieren.
Designmuster konzentrieren sich somit auf das Innenleben eines Projekts.\cite{gfg_DesignVsArchitecture}

Dabei wird unterschieden zwischen
\begin{itemize}
	\item \textbf{Erzeugungsmuster},
	\item \textbf{Strukturmuster} und
	\item \textbf{Verhaltensmuster}.
\end{itemize}

Erzeugungsmuster helfen dabei, die Art und Weise der Erzeugung von Objekten umzusetzen.
Sie sorgen dafür, dass der eigentliche Erzeugungsprozess nicht direkt sichtbar.
Der Fokus liegt auf der Trennung von der Erzeugung und Verwendung von Objekten, um Flexibilität, Wiederverwendbarkeit und Austauschbarkeit zu ermöglichen.
Beispiel, die im Laufe dieser Arbeit noch erscheinen sind das Factory-Pattern, das Singleton-Pattern oder das Builder-Pattern.
%% Factory
Das Factory-Pattern dient der Kapselung des Erzeugungsprozesses, indem die Instanziierung von Objekten an eine spezielle Fabrikklasse oder -methode ausgelagert wird. 
Es ist nicht erforderlich, dass dem aufrufenden Code der konkrete Typ des zurückgegebenen Objekts bekannt ist. 
Dies hat eine erhöhte Austauschbarkeit und Erweiterbarkeit zur Folge.

%% Singleton
Das sogenannte Singleton-Pattern stellt sicher, dass von einer bestimmten Klasse lediglich eine Instanz existiert und diese global zugänglich ist. 
Typischerweise findet es Anwendung, um zentrale Ressourcen wie Konfigurationsobjekte oder Schnittstellen konsistent bereitzustellen.

%% Builder
Das Builder-Pattern dient der schrittweisen und flexiblen Erzeugung komplexer Objekte.
Dabei werden die Eigenschaften dieser Objekte nacheinander gesetzt. 
Auf diese Weise wird eine klare Trennung zwischen dem Erstellungsprozess und der finalen Darstellung erreicht.

Strukturmuster helfen dabei, die erstellten Klassen und Objekte zu organisieren.
Der Fokus liegt auf dem Zusammenspiel unabhängig entwickelter Klassenbibliotheken sowie der Vereinfachung von Schnittstellen und dem modularen Aufbau von Systemen.
%% Facade
Das sogenannte Facade-Pattern dabei stellt eine Methode dar, um eine einheitliche und vereinfachte Schnittstelle zu einem komplexen Subsystem bereitzustellen. 
Dies hat den Vorteil, dass Implementierungsdetails verborgen werden und die Verwendung der Schnittstelle für den Anwender erheblich vereinfacht wird.

Verhaltensmuster definieren, wie Objekte miteinander interagieren, wie Zuständigkeiten aufgeteilt werden und wie der Kontrollfluss zwischen ihnen abläuft.
Der Fokus liegt nicht ausschließlich auf dem ''Was'' (z. B. ein Event), sondern auch auf dem ''Wie'', ''Wann'' und ''Wer''.
Ein Beispiel hierfür ist die Template Method. 
In Muster definiert eine Basisklasse einen Algorithmus, d.h. eine feststehende Reihenfolge von Befehlen oder Funktionen. 
Die Implementierung aller Schritte wird dabei nicht durch die Basisklasse selbst vorgenommen, sondern kann für die jeweiligen einzelnen Zwischenschritte, sogenannter Hooks, individuell durch Unterklassen erfolgen.
Im Rahmen der STM32-HAL wird dieses Muster bei Callback-Funktionen für Interrupts verwendet. 
Die Funktion \texttt{HAL\_GPIO\_EXTI\_IRQHandler(uint16\_t GPIO\_Pin)} fungiert als Template, während die Funktion \texttt{HAL\_GPIO\_EXTI\_Callback(uint16\_t GPIO\_Pin)} als Hook vom Entwickler selbst implementiert werden muss.

\subsection{Application Programming Interface}
Eine \emph{Anwendungsprogrammierschnittstelle} (\gls{api}) wird von IBM beschrieben ''eine Reihe von Regeln oder Protokollen, die es Softwareanwendungen ermöglichen, miteinander zu kommunizieren, um Daten, Funktionen und Funktionalitäten auszutauschen.''\cite{ibmAPI}.
Damit soll eine Vereinfachung und Effizienzsteigerung für die Softwareentwicklung erreicht werden.
\glspl{api} dienen als Zwischenschicht zwischen verschiedenen Softwarekomponenten oder Systemen.
Sie ermöglichen eine klare Abgrenzung der Zuständigkeiten und stellen eine Abstraktion komplexer interner Abläufe hinter einer standardisierten Schnittstelle bereit.
So können beispielsweise Anwendungen Datenformate automatisch anpassen oder Funktionen anderer Programme nutzen, ohne deren interne Implementierung kennen zu müssen.
Eine solche standardisierte Schnittstelle ermöglicht es die \gls{api}-Funktionen wieder zu verwenden, so dass Entwickler diese nicht immer wieder neu implementieren müssen.
Gleichzeitig wird zur allgemeinen Sicherheit beigetragen, da nur definierte Informationen nach außen weitergegeben werden und der Zugriff von außen gezielt eingeschränkt.

%Dies hat zur Folge, dass Geräte oder Server ihre Daten nicht vollständig aufdecken müssen.

\subsection{CMake}
CMake ist ein plattformübergreifendes Open-Source-Werkzeug zur Automatisierung des Buildprozesses in der Softwareentwicklung
Der sogenannte Metabuild-Generator (\autoref{fig:cmake_generators}) dient als eine Art universeller Konfigurator, der mithilfe Konfigurationsdateien, den \texttt{CMakeLists.txt}-Dateien, spezifische Build-Systeme für eine Vielzahl unterschiedlicher Plattformen und Entwicklungsumgebungen generiert.
Unter diesen Build-Systemen finden sich beispielsweise Makefiles für Unix/Linux, Projektdateien für Visual Studio oder Xcode.

\begin{figure}[H]
	\includegraphics[width=\textwidth]{cmake_generators.png}
	\caption{Ausschnitt einer Liste von verfügbaren Generatoren.}
	\label{fig:cmake_generators}
\end{figure}

Ein wesentlicher Vorteil von CMake liegt in der Trennung von Quell- und Build-Verzeichnissen, was sogenannte Out-of-Source-Builds ermöglicht.
Diese Vorgehensweise trägt zur Schaffung einer übersichtlichen Projektstruktur bei und vereinfacht die Verwaltung von Build-Artefakten.
Zusätlich fördert CMake die hierarchische Strukturierung von Projekten mittels der Implementierung von modularen CMakeLists.txt-Dateien in Unterverzeichnissen.
Dieser Ansatz steigert die Wartbarkeit und Skalierbarkeit komplexer Softwareprojekte.

\subsection{Make und Makefiles}
Make ist ein traditionelles Werkzeug zur Automatisierung von Build-Prozessen, das sogenannte Makefiles zur Steuerung dieser Prozesse einsetzt.
Die Makefiles definieren Regeln, mit deren Hilfe der Quellcode, abhängig davon ob sich etwas im Code geändert hat, kompiliert und verlinkt wird.
Make findet für gewöhnlich Anwendung in der direkten Steuerung von Kompilierungsprozessen.
 Es besteht jedoch auch die Möglichkeit, es zur Steuerung anderer Build-Systeme einzusetzen.
In einigen Projekten findet ein manuelles Makefile Verwendung, welches ausschließlich CMake mit spezifischen Parametern aufruft, um den eigentlichen Build-Prozess zu initialisieren.
In einem solchen Szenario fungiert Make als Wrapper über CMake und ersetzt nicht dessen eigentliche Build-Logik.

































\chapter{Stand der Technik}

Unter Portabilität von Software in eingebetteten Systemen wird im Kontext dieser Arbeit die Fähigkeit verstanden, einmal entwickelten Quellcode mit möglichst geringem Anpassungsaufwand auf unterschiedliche Mikrocontrollerarchitekturen und Hardwareplattformen zu übertragen. 
Dieses Ziel besitzt insbesondere im Hinblick auf die potenziellen Änderungen der Hardwareanforderungen im Produktlebenszyklus eine hohe Relevanz, da es zu einer Migration auf alternative Controllerfamilien kommen kann.
Von zentraler Bedeutung sind hierbei die Auswahl geeigneter Treiber sowie die Abstraktionsebene der Hardwarezugriffe, die die Entkopplung zwischen Anwendungscode und konkreter Zielhardware bestimmen.

In der Praxis existieren verschiedene Ansätze, die versuchen, mit jeweils eigene Stärken und Einschränkungen, dieses Problem zu lösen.

\section{Lightweight Operation Systems}
Die Abstraktion von Hardwarefunktionen kann durch den Einsatz schlanker Betriebssysteme wie FreeRTOS, Zephyr oder RIOT-OS erfolgen. 
Diese bieten einheitliche Schnittstellen für Multitasking, Synchronisation und Peripheriezugriffe und erleichtern dadurch die Übertragbarkeit von Anwendungen zwischen unterschiedlichen Mikrocontrollerarchitekturen. 
Gleichzeitig verursachen sie jedoch einen zusätzlichen Ressourcenverbrauch, der auf sehr leistungslimitierten Controllern problematisch sein kann.
 Darüber hinaus ist anzumerken, dass die Abhängigkeit von den jeweiligen Hardwareabstraktionsschichten (\gls{hal}) bestehen bleibt, sodass eine vollständige Unabhängigkeit von der Zielhardware nicht erreicht wird. \cite{freertos}\cite{zephyr}\cite{riot}.  


\section{Retargetierbare Compiler}
Ein alternativer Ansatz besteht im Einsatz retargetierbarer Compiler. 
Die Verwendung von Systemen wie GCC oder LLVM erlaubt die Übertragung identischer Quellcodes auf unterschiedliche Zielarchitekturen, wobei lediglich das Backend an die spezifische Plattform angepasst wird.
Dadurch wird eine hohe Flexibilität auf der Ebene der Codegenerierung gewährleistet. 
Allerdings erfolgt keine Abstraktion der hardwarenahen Zugriffe.
Die Funktion \texttt{HAL\_GPIO\_WritePin()} wie bei STM32 verwendet wird, kann somit nicht für ESP32 verwendet werden.
Hier kommt die Funktion \texttt{gpio\_set\_level()} zum Einsatz.
Ein retargetierbarer Compiler allein kann diese Unterschiede nicht kompensieren. 
Peripheriezugriffe und Registerkonfigurationen bleiben plattformspezifisch, sodass diese für jeden Mikrocontroller individuell implementiert werden müssen. 
Ohne eine zusätzliche Abstraktionsschicht muss der Quellcode für jede Plattform individuell angepasst werden.\cite{gccint}\cite{llvm}\cite{johnson1975}

\section{Arduino-Framework}
Das Arduino-Ökosystem zeichnet sich durch einen pragmatischen, anwendungsorientierten Ansatz aus.
Ursprünglich für AVR-Mikrocontroller konzipiert, existieren heute Erweiterungen sowohl für STM32- als auch für ESP32-Micorcontroller. 
Die Entwicklung und Portierung wird durch die Verwendung einer einheitlichen und stark vereinfachten API erleichtert, die die Hardwarezugriffe abstrahiert. 
Dadurch lassen sich schnelle Prototypen erstellen. 
Dieser Ansatz hat sich insbesondere in der Maker-Community etabliert, weist jedoch im industriellen Umfeld gewisse Einschränkungen auf. 
Die Abstraktionsebene weist eine vergleichsweise geringe Detailtiefe auf, die Effizienz ist in nicht allen Szenarien ausreichend und Aspekte wie Modularität oder langfristige Wartbarkeit sind nur eingeschränkt gewährleistet.\cite{stm32duino}\cite{arduinoesp32}. 

\vspace{0.5em}
\noindent\rule{\linewidth}{0.4pt}
\vspace{0.5em}

Die zuvor genannten Ansätze verdeutlichen, dass die Frage der Portabilität stets mit einem Konflikt zwischen Ressourcenverbrauch, Abstraktionsgrad und Wartbarkeit verbunden ist. 
Obwohl Betriebssysteme und Compiler vorrangig die Entwicklungsumgebung standardisieren und Arduino durch eine vereinfachte High-Level-API arbeitet, sind auch diese etablierten Ansätze nicht frei von Einschränkungen. 
In Anbetracht dessen verfolgt die vorliegende Arbeit einen alternativen Ansatz. 
Der Fokus liegt auf der Entwicklung einer modularen, ressourcenschonenden und plattformunabhängigen Treiberbibliothek in C++. 
Im Gegensatz zu Light OS erfolgt keine vollständige Einführung eines Betriebssystems, wodurch der Ressourcenverbrauch minimiert bleibt. 
Im Zuge dessen wird gegenüber Compiler-basierten Lösungen eine klare Abstraktionsebene geschaffen, die hardwarespezifische Implementierungen kapselt.
Im Unterschied zum Arduino-Framework liegt der Fokus nicht auf einer vereinfachten, sondern auf einer wohldefinierten und erweiterbaren Schnittstelle, die eine langfristige Softwarebasis für den industriellen Einsatz ermöglicht.


\chapter{Konzeption der API}
%Die Entwicklung einer benutzerfreundlichen und leistungsfähigen API-Library erfordert eine systematische Herangehensweise, die die einzelnen Phasen der Anforderungsanalyse, Architekturentwurf, Implementierung, Testing und Dokumentation integriert.
 
In diesem Teil der Arbeit wird ein Konzept der API erstellt.
Der Aufbau dieses Konzepts passiert in mehreren Schritten:
\begin{itemize}
	\item [1.] Anforderungsanalyse: \\In diesem Abschnitt werden die wichtigen Eigenschaften, die die API haben muss zusammengetragen. Daneben wird analysiert, wie die Funktionen, die enthalten sein sollen aufgebaut und implementiert werden können.
	\item [2.] Architekturentwurf: \\Hier werden Architekturmuster für das gesamte System der API und Designmuster für mögliche Module ausgewählt.
\end{itemize}

Anhand dieses Entwurfs wird ein neue Zwischenschicht implementiert.
%Die Zwischenschicht wählt zur Kompilierzeit die richtige Hardware aus, damit das nicht zur Runtime geschehen muss (da dies Performanceeinbusen mit sich bringen würde) und bekommt so die richtigen Treiber mit.
%Die Zwischenschicht soll eine Art Default-Klasse für die jeweilige Funktion bereitstellen.
%Mit der ausgewählten Hardware können die Default-Klassen die richtigen Treiber ansprechen.

\section{Anforderungsanalyse}
Um eine benutzerfreundlichen und leistungsfähigen API-Bibliothek entwickeln zu können, ist es wichtig die grundlegenden Funktionen klar zu definieren.
So gilt es als erstes die Fragen zu klären: \emph{Was muss die API können?} und \emph{Welche Eigenschaften soll die API haben?}

Es ist das übergeordnete Ziel, plattformunabhängigen Code schreiben zu können.
Das bedeutet, es muss möglich sein ein Programm, das z.B. für eine Hardware mit einem STM32-MCU geschrieben wurde, auch für Hardware mit einer ESP32-MCU funktionsfähig zu haben.
Die spezifische Konfiguration der Hardware und der Pins, wie sie beispielsweise mit STM32CubeMX gemacht werden kann, muss dennoch für jede Hardware, je nach Projekt, neu erstellt werden.
Dies liegt unter anderem an den unterschiedlichen Prozessorarchitekturen, der Anzahl an Pins und deren Zuordnung zu spezifischen Funktionen oder der Registerkonfiguration.
Um eine Pinkonfiguration mit Code zu lösen und von der graphischen Oberfläche wegzukommen, liegt der Gedanke, Objekte zu verwenden nahe.
Besonders im Kontext der Verwendung von C++.
Solche Objekt werden mittels eines Konstruktors, der die Werte für die Attribute der Pins übergeben bekommt, erstellt.
Bevor eine Erstellung dieser Pin-Objekte stattfinden kann, Aufgrund der angesprochen Unterschiede, muss erst die Hardware ausgewählt werden.

Damit die Pin-Objekte auch verwendet werden können, muss vorher die Hardware ausgewählt und initialisiert werden.
Beginnend mit der Auswahl, muss die API in der Lage sein nach einer Art der Definition, welche Hardware real zur Verfügung steht, die passenden Treiber auszuwählen, eine Instanz der Hardware zu erstellen, mit der im Programm gearbeitet werden kann und aus diesem heraus die Hardware über allgemein definierte Funktionen mit den richtigen Treibern zu initialisieren.
Diese Definition kann beispielsweise über ein \texttt{\#define}, dass den Namen der Hardware beinhaltet gelöst werden.
Mit Blick auf zukünftige Veränderungen sollte es auch so einfach wie möglich sein, weitere Hardware der API hinzu zu fügen, um die Auswahl zu erweitern.
Diese Veränderungen und Erweiterungen würden auch die jeweiligen Peripheriefunktionen betreffen.
Um einen klaren Überblick über diese Funktionen zu behalten, ist der Gedanke an Module zu betrachten.
So könnte für jede Peripheriefunktion (\gls{gpio}, \gls{spi}, \gls{uart}, \gls{can}) ein eigenes Modul implementiert werden.
Auf diese Weise hat man neben dem Überblick auch eine klare Struktur, die Fehlersuchen und Wartungen der Software wiederum vereinfacht.



\section{Architekturentwurf}

\begin{itemize}
%	\item Was muss die API können?
	\item Welche Grundarchitektur liegt vor?
	\item Welches Architekturmuster eignet sich dafür?
	\item Welche Architektur eignet sich hier für? $\rightarrow$ Aufbau des gesamten Systems
	\item Desgin Patterns $\rightarrow$ Aufbau der einzelnen Klassen
	\begin{itemize}
		\item Welche werden sonst eingesetzt; welches eignet sich für diesen Zweck
		\item Wie sollen die Module der jeweiligen Peripherie aufgebaut sein?
	\end{itemize}
	\item C++ $\rightarrow$ Objekt-orientiert
	\item Erstellung von Objekten?
	\item Auswahl der Funktionen
	\item 
\end{itemize}

\begin{itemize}
	\item Globales Interface
	\item Factory Architektur zur Erstellung von Objekten
	\item Peripheriefunktionen (GPIO, SPI, etc.) als eigenständigen Klassen
%	\item Verwendung von \texttt{namespaces}
	\item HardwareInterface mit allen:
	\begin{itemize}
		\item vllt. Core: ClockInit, Delay, GetTick
		\item Gpio
		\item SPI
		\item UART
		\item CAN
	\end{itemize}
	\item Interface ruft Factory auf, die MCU spezifischen Treiber inkludiert und eine Instanz des HW-Objektes zurückgibt. Mit dieser kann gearbeitet werden.
	\item 
\end{itemize}


%\section{Einstellungen pro MCU}

%TODO: informations in reference manual for each mcu
% CONFIG Abschnitt im cmake
%\begin{lstlisting}
%add_compile_options(
%	-mcpu=
%	-mfloat-abi=
%	-mfpu=
%	-mthumb
%	-ffunction-sections
%	-fdata-sections
%	$<$<COMPILE_LANGUAGE:CXX>:-fno-exceptions>
%	$<$<COMPILE_LANGUAGE:CXX>:-fno-rtti>
%	$<$<COMPILE_LANGUAGE:CXX>:-fno-threadsafe-statics>
%	$<$<COMPILE_LANGUAGE:CXX>:-fno-use-cxa-atexit>
%)
%\end{lstlisting}



\section{Architektonische Eigenschaften der Treiber-API}
Moderen Softwarelösungen bestehen meist aus vielen, großen Dateien, die untereinander von einander abhängig sind.
Um bei solch großen Projekten den Überblick zu behalten, werden/sollten diese Softwarelösungen nach gewissen Eigenschaften erstellt werden.
Diese \emph{architektonischen Eigenschaften} lassen sich (grob) in drei Teilbereiche unterteilen: Betriebsrelevante, Strukturelle und Bereichsübergreifende, wie in Tabelle~\ref{tab:architektonische_eigenschaften} aufgeführt. % Eigenschaften. 

\begin{table}[H]
	\begin{center}
		\begin{tabular}{ c | c | c }
			\textbf{Betriebsrelevante} & \textbf{Strukturelle} & \textbf{Bereichsübergreifende}\\
			%\midrule
			\hline
			Verfügbarkeit & Erweiterbarkeit & Sicherheit\\
			Performance & Modularität & Rechtliches\\
			Skalierbarkeit & Wartbarkeit & Usability\\
			$\cdots$ & $\cdots$ & $\cdots$\\
		\end{tabular}
		\caption{Teilbereiche architektonischer Eigenschaften}
	    \label{tab:architektonische_eigenschaften}
	\end{center}
\end{table}

Aus diesen Eigenschaften gilt es, die wichtigsten für die Treiber-API zu identifizieren. 
Mit diesem Hintergrund lässt sich ein Struktur für das Projekt bilden.

Die Entwicklung einer plattformunabhängigen, wiederverwendbaren Treiber-API für Mikro-controller stellt hohe Anforderungen an die Architektur der Softwarebibliothek.

% Welche (architektonischen) Eigenschaft sind wichtig/sollen umgesetzt werden?
% geringe Redundanz
Das Ziel besteht darin, eine Lösung zu schaffen, die sich durch eine geringe Redundanz auszeichnet. 
Die Konzeption von Klassen und Funktionen sollte derart erfolgen, dass eine erneute Implementierung der Applikation für jede neue Plattform nicht erforderlich ist.
Die Wiederverwendbarkeit zentraler Komponenten führt zu einer Reduktion des Entwicklungsaufwands und einer Erhöhung der Konsistenz im Code.

% Usability - Benutzerfreundlichkeit
Ein weiteres zentrales Anliegen ist die einfache Benutzbarkeit. 
Die API ist so zu gestalten, dass eine effiziente Nutzung gewährleistet ist. 
Dies fördert nicht nur die Effizienz in der Erstellung neuer Applikationen, sondern erleichtert auch langfristig die Wartung und Weiterentwicklung der Software.

% Skalierbarkeit
Im Sinne der Skalierbarkeit wird angestrebt, die Lösung auf möglichst viele Mikrocontroller-Architekturen und Hardwareplattformen anwendbar zu machen.
Die Vielfalt verfügbarer MCUs erfordert eine abstrahierte und flexibel erweiterbare Struktur, die die Integration neuer Plattformen mit minimalem Aufwand ermöglicht.

% Portabilität - nochmal anpassen; OS Bezug passt nicht richtig
Auch die Portabilität spielt eine wichtige Rolle.
Die Bibliothek sollte nicht nur hardware-, sondern auch betriebssystemunabhängig konzipiert werden.
Aus diesem Grund wird bei der Entwicklung der Lösung darauf geachtet, dass diese erst unter Windows, später auch unter Linux und macOS einsetzbar ist.
Die Installation und Konfiguration der dafür benötigten Werkzeuge wird nachvollziehbar dokumentiert, um den Einstieg für die Nutzer zu erleichtern.

% Erweiterbarkeit
Darüber hinaus ist die Erweiterbarkeit ein wesentliches Architekturprinzip
Der Einsatz von leistungsstärkeren Mikrocontrollern hängt in der Regel mit einer Erweiterung der Funktionalitäten zusammen, die in die bestehenden Treiber- und API-Strukturen integriert werden müssen.
Daher wird großer Wert auf eine modulare und offen gestaltete Architektur gelegt, die neue Features ohne grundlegende Umbauten aufnehmen kann.

% Modularität
Modularität trägt wesentlich zur Übersichtlichkeit und Wartbarkeit des Systems bei. 
Eine saubere Trennung funktionaler Einheiten ermöglicht eine schnellere Lokalisierung und Behebung von Fehlern, was wiederum die langfristige Pflege und Weiterentwicklung der Software erleichtert.

% (Ressourcen-)Effizienz
Schließlich ist auch die Effizienz ein kritischer Aspekt.
Da Mikrocontroller in der Regel nur über begrenzte Ressourcen verfügen, ist es essenziell, dass die Bibliothek möglichst kompakt und ressourcenschonend implementiert wird. 
Externe Abhängigkeiten werden bewusst auf ein Minimum reduziert, um Speicherplatz zu sparen und unnötige Komplexität zu vermeiden.

Diese architektonischen Prinzipien bilden die Grundlage für die Konzeption und Umsetzung der in dieser Arbeit vorgestellten Treiber-API.

% Fragen
Wie wird der jeweilige Punkt umgesetzt?

Welche Tools werden benutzt/eignen sich besonders für die Umsetzung?
Welche Tools eignen sich für welchen Arbeitsschritt?

Warum wird etwas gerade auf diese Weise umgesetzt?
































\chapter{Implementierung}











	\printbibliography

\end{document}