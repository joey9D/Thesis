Die Umsetzung dieser Arbeit besteht aus mehreren Phase.

Die Entwicklung einer benutzerfreundlichen und leistungsfähigen API-Library erfordert eine systematische Herangehensweise, die die einzelnen Phasen der Anforderungsanalyse, Architekturentwurf, Implementierung, Testing und Dokumentation integriert.
 


Um dieses Problem und die in vorherigem Abschnitt angesprochenen Probleme anzugehen, soll eine neue/weitere Zwischenschicht implementiert werden. 
Die Zwischenschicht wählt zur Kompilierzeit die richtige Hardware aus, damit das nicht zur Runtime geschieht; und bekommt so die richtigen Treiber mit.
Die Zwischenschicht soll eine Art default-Klasse für die jeweilige Funktion bereitstellen.
Mit der ausgewählten Hardware können die Default-Klassen die richtigen Treiber ansprechen.

\section{Analyse}
\begin{itemize}
	\item Was muss die API können?
	\item Welche Architektur eignet sich hier für?
	\item C++ $\rightarrow$ Objekt-orientiert
	\item Erstellung von Objekten?
	\item Auswahl der Funktionen
	\item 
\end{itemize}

\section{Ansatz}
\begin{itemize}
	\item Globales Interface
	\item Factory Architektur zur Erstellung von Objekten
	\item Peripheriefunktionen (GPIO, SPI, etc.) als eigenständigen Klassen
%	\item Verwendung von \texttt{namespaces}
	\item HardwareInterface mit allen:
	\begin{itemize}
		\item vllt. Core: ClockInit, Delay, GetTick
		\item Gpio
		\item SPI
		\item UART
		\item CAN
	\end{itemize}
	\item Interface ruft Factory auf, die MCU spezifischen Treiber inkludiert und eine Instanz des HW-Objektes zurückgibt. Mit dieser kann gearbeitet werden.
	\item 
\end{itemize}


%\section{Einstellungen pro MCU}

%TODO: informations in reference manual for each mcu
% CONFIG Abschnitt im cmake
%\begin{lstlisting}
%add_compile_options(
%	-mcpu=
%	-mfloat-abi=
%	-mfpu=
%	-mthumb
%	-ffunction-sections
%	-fdata-sections
%	$<$<COMPILE_LANGUAGE:CXX>:-fno-exceptions>
%	$<$<COMPILE_LANGUAGE:CXX>:-fno-rtti>
%	$<$<COMPILE_LANGUAGE:CXX>:-fno-threadsafe-statics>
%	$<$<COMPILE_LANGUAGE:CXX>:-fno-use-cxa-atexit>
%)
%\end{lstlisting}



\section{Architektonische Eigenschaften der Treiber-API}
Moderen Softwarelösungen bestehen meist aus vielen, großen Dateien, die untereinander von einander abhängig sind.
Um bei solch großen Projekten den Überblick zu behalten, werden/sollten diese Softwarelösungen nach gewissen Eigenschaften erstellt werden.
Diese \emph{architektonischen Eigenschaften} lassen sich (grob) in drei Teilbereiche unterteilen: Betriebsrelevante, Strukturelle und Bereichsübergreifende, wie in Tabelle~\ref{tab:architektonische_eigenschaften} aufgeführt. % Eigenschaften. 

\begin{table}[H]
	\begin{center}
		\begin{tabular}{ c | c | c }
			\textbf{Betriebsrelevante} & \textbf{Strukturelle} & \textbf{Bereichsübergreifende}\\
			%\midrule
			\hline
			Verfügbarkeit & Erweiterbarkeit & Sicherheit\\
			Performance & Modularität & Rechtliches\\
			Skalierbarkeit & Wartbarkeit & Usability\\
			$\cdots$ & $\cdots$ & $\cdots$\\
		\end{tabular}
		\caption{Teilbereiche architektonischer Eigenschaften}
	    \label{tab:architektonische_eigenschaften}
	\end{center}
\end{table}

Aus diesen Eigenschaften gilt es, die wichtigsten für die Treiber-API zu identifizieren. 
Mit diesem Hintergrund lässt sich ein Struktur für das Projekt bilden.

Die Entwicklung einer plattformunabhängigen, wiederverwendbaren Treiber-API für Mikro-controller stellt hohe Anforderungen an die Architektur der Softwarebibliothek.

% Welche (architektonischen) Eigenschaft sind wichtig/sollen umgesetzt werden?
% geringe Redundanz
Das Ziel besteht darin, eine Lösung zu schaffen, die sich durch eine geringe Redundanz auszeichnet. 
Die Konzeption von Klassen und Funktionen sollte derart erfolgen, dass eine erneute Implementierung der Applikation für jede neue Plattform nicht erforderlich ist.
Die Wiederverwendbarkeit zentraler Komponenten führt zu einer Reduktion des Entwicklungsaufwands und einer Erhöhung der Konsistenz im Code.

% Usability - Benutzerfreundlichkeit
Ein weiteres zentrales Anliegen ist die einfache Benutzbarkeit. 
Die API ist so zu gestalten, dass eine effiziente Nutzung gewährleistet ist. 
Dies fördert nicht nur die Effizienz in der Erstellung neuer Applikationen, sondern erleichtert auch langfristig die Wartung und Weiterentwicklung der Software.

% Skalierbarkeit
Im Sinne der Skalierbarkeit wird angestrebt, die Lösung auf möglichst viele Mikrocontroller-Architekturen und Hardwareplattformen anwendbar zu machen.
Die Vielfalt verfügbarer MCUs erfordert eine abstrahierte und flexibel erweiterbare Struktur, die die Integration neuer Plattformen mit minimalem Aufwand ermöglicht.

% Portabilität - nochmal anpassen; OS Bezug passt nicht richtig
Auch die Portabilität spielt eine wichtige Rolle.
Die Bibliothek sollte nicht nur hardware-, sondern auch betriebssystemunabhängig konzipiert werden.
Aus diesem Grund wird bei der Entwicklung der Lösung darauf geachtet, dass diese erst unter Windows, später auch unter Linux und macOS einsetzbar ist.
Die Installation und Konfiguration der dafür benötigten Werkzeuge wird nachvollziehbar dokumentiert, um den Einstieg für die Nutzer zu erleichtern.

% Erweiterbarkeit
Darüber hinaus ist die Erweiterbarkeit ein wesentliches Architekturprinzip
Der Einsatz von leistungsstärkeren Mikrocontrollern hängt in der Regel mit einer Erweiterung der Funktionalitäten zusammen, die in die bestehenden Treiber- und API-Strukturen integriert werden müssen.
Daher wird großer Wert auf eine modulare und offen gestaltete Architektur gelegt, die neue Features ohne grundlegende Umbauten aufnehmen kann.

% Modularität
Modularität trägt wesentlich zur Übersichtlichkeit und Wartbarkeit des Systems bei. 
Eine saubere Trennung funktionaler Einheiten ermöglicht eine schnellere Lokalisierung und Behebung von Fehlern, was wiederum die langfristige Pflege und Weiterentwicklung der Software erleichtert.

% (Ressourcen-)Effizienz
Schließlich ist auch die Effizienz ein kritischer Aspekt.
Da Mikrocontroller in der Regel nur über begrenzte Ressourcen verfügen, ist es essenziell, dass die Bibliothek möglichst kompakt und ressourcenschonend implementiert wird. 
Externe Abhängigkeiten werden bewusst auf ein Minimum reduziert, um Speicherplatz zu sparen und unnötige Komplexität zu vermeiden.

Diese architektonischen Prinzipien bilden die Grundlage für die Konzeption und Umsetzung der in dieser Arbeit vorgestellten Treiber-API.

% Fragen
Wie wird der jeweilige Punkt umgesetzt?

Welche Tools werden benutzt/eignen sich besonders für die Umsetzung?
Welche Tools eignen sich für welchen Arbeitsschritt?

Warum wird etwas gerade auf diese Weise umgesetzt?





























