% TODO: Abstract: anpassen am Beispiel von Tati

\begin{abstract}
	
	
	
	Microcontroller unterscheiden sich hinsichtlich ihrer Architektur, ihres Befehlssatz, der Taktfrequenz, des verfügbaren Speichers, der Peripherie und weiterer Eigenschaften teils erheblich.
	Die Aufgabe des Embedded-Softwareentwicklers besteht demnach darin, Hardware auszuwählen, die die Rahmenbedingungen des geplanten Einsatzes erfüllt.
	Darüber hinaus ist die Bereitstellung geeigneter Treiber essenziell, um eine optimale Ansteuerung der Hardware zu gewährleisten.
	
	Die vorliegende Bachelorthesis untersucht bewährte Methoden %(Best Practices) 
	zur Entwicklung einer plattformunabhängigen Treiber-API für Microcontroller, mit dem Ziel, die Wiederverwendbarkeit von Applikationen und Softwarelösungen in der Embedded-Softwareentwicklung zu fördern und eine einfache Nutzung zu ermöglichen. 
	Es wird analysiert, mit welchen Techniken verschiedene Treiber und Bibliotheken integriert und wie diese unterschiedlichen Hardwarekonfigurationen bereitgestellt werden.
	Dabei wird auch betrachtet, welche Auswirkungen unterschiedliche Prozessorarchitekturen auf die Umsetzung einer eigenen Treiberbibliothek haben.
	Diese integriert vorhandene Treiber, ersetzt hardwarespezifische Funktionen durch abstrahierte Schnittstellen und ermöglicht dadurch die Wiederverwendbarkeit der Applikation auf verschiedenen Hardwareplattformen, ohne dass eine Neuimplementierung erforderlich ist.

	Der modulare Aufbau des Projekts, das durch die Verwendung von Open-Source-Tools realisiert wurde, erlaubt eine flexible Erweiterung und kontinuierliche Optimierung.
	
\end{abstract}  
