Das Ziel dieser Arbeit ist es die Basis eine Treiber-API zu erstellen, mit der je nach Zielhardware die passenden Treiber integriert werden können.
Dafür muss eine Programmstruktur entwickelt werden, die es ermöglicht erstellte Softwarelösungen und Applikationen auf verschiedenen Microkontrollern anwenden zu können, indem die spezifischen Hardwaretreiber, nach geringer Konfiguration, automatisch in den Buildprozess mit integriert werden.
Die Struktur soll erste grundlegenden Funktionen für GPIO, SPI, CAN und UART enthalten. 
Damit können die Funktionen für generelles Lesen, Schreiben und die Kommunikation über Busse getestet werden.
\\
%\textbf{Warum für diese Entschieden?}
%GPIO für die simpelste Art für Lesen und Schreiben
%
%CAN damit ein anderes Projekt direkt integriert werden kann.
%
%
%SPI und UART um Kommunikation via Bus zu testen




\section{Rahmenbedingungen} \label{chap2_rahmenbedingungen}
Die Arbeit wird in der Schaefer GmbH erstellt. 
Die Firma sorgt mit ihren Custom-Designs von Aufzugkontrollpanels dafür, dass jeder Kunde seinen spezifischen Wunsch erfüllt bekommt.
In diesen Kontrollpanels kommen unterschiedliche \gls{mcu}s zum Einsatz um die jeweiligen Softwarelösungen umzusetzen.
Als Entwicklungsumgebung (\gls{ide}) wird primär die STM32CubeIDE verwendet.
Mit den bereits integrierten Tools zum Bauen (Builden) und Debuggen von hardware-orientierter Software, eignet sich diese \gls{ide} besonders gut um einen erleichterten Einstieg zu erhalten.
Da die Lösung plattformübergreifend funktionieren soll, wird auch Visual Studio Code (VSCode) verwendet.
Diese \gls{ide} wird weltweit genutzt und kann durch Extension für den gewünschten Gebrauch angepasst werden kann. 
Außerdem ist VSCode auf den gängigen Betriebssytemen lauffähig.

Damit die API direkt auf einem etablierten Stand ist, soll sie in C++ dem Standard 17 nach, programmiert werden.
Um über VSCode für das Arbeiten mit C++ vorzubereiten und anzupassen, empfiehlt es sich Erweiterungen (Extensions) zu installieren.
Im Rahmen dieser Arbeit wird das Paket von frannekXX verwendet.
Dieses beinhaltet alle für die moderne C++-Entwicklung relevanten Paket:
\begin{itemize}
	\item C/C++ Extension Pack  v1.3.1 by Microsoft
	\begin{itemize}
		\item C/C++  v1.25.3  by Microsoft		
		\item CMake Tools  v1.20.53  by Microsoft
		\item C/C++ Themes  v2.0.0 by Microsoft 
	\end{itemize}
	\item C/C++ Runner  v9.4.10 by franneckXX
	\item C/C++ Config  v6.3.0 by franneckXX
	\item CMake  v0.0.17 by twxs
	\item Doxygen  v1.0.0 by Baptist BENOIST
	\item Doxygen Documentation Generator  v1.4.0 by Christopher Schlosser
	\item CodeLLDB  v1.11.4 by Vadim Chugunov
	\item Better C++ Syntax  v1.27.1  by Jeff Hyklin
	\item x86 and x86\_64 Assembly  v3.1.5 by 13xforever
	\item cmake-format  v0.6.11 by cheshirekow
\end{itemize}

Dabei handelt es sich bei jeder Extension um die aktuellste Version.
%Diese dreizehn Erweiterungen lassen sich zusammenfassen zu C/C++ relevant, Buildsystem, Dokumentation und Formatierung \& Optik.
Für die Nutzung von VSCode mit den verwendeten MCUs gibt es ebenfalls entsprechende Extensions.
Um STM32-\gls{mcu}s zu programmieren gibt es offizielle Extensions von\\STMicroelectronics.
Zu installieren ist hier \emph{STM32Cube for Visual Studio Code}.
Zusätzlich empfiehlt es sich zu den bereits genannt \gls{ide}s, STM32CubeIDE und Espressif-IDE, auch deren Umgebungen mit zu installieren.
Für ST-Hardware sind das STM32CubeMX um die \gls{mcu}s zu konfigurieren, STM32Programmer um die Hardware zu Programmieren % STM32-Finder um die richtigen \gls{mcu}s auswählen zu können.

Um eine erstellte API testen zu können wird im Rahmen dieser Arbeit auf folgende Hardware der Firmen STMicroelectronics und Espressif Systems zurückgegriffen:
\begin{itemize}
	\item STM32C032C6
	\item STM32G071RB
	\item STM32G0B1RE
	\item ESP32-C6 DevKitC-1
\end{itemize}


\section{Anforderungen an die Lösung}
Im Rahmen dieser Arbeit sollen die grundlegenden Funktionen wie Lesen und Schreiben der folgenden Kommunikationsprotokolle implementiert werden:
\begin{itemize}
	\item \gls{gpio}
	\item \gls{spi}
	\item \gls{uart}
	\item \gls{can}
\end{itemize}

In einen Schaltkreis sind eine LED und ein Taster verbaut. 
Um die Kommunikation über \gls{gpio} zu testen soll das Betätigen des Tasters die LED zum leuchten bringen.
Auf diese Weise kann das Lesen, der Input, des Tasters und das Schreiben, der Output über das Leuchten der LED getestet werden.

% TODO: chap3: Bedingungen für die Bussysteme nochmal überarbeiten; wenn es dann so weit ist.
% 		[ ] SPI
% 		[ ] UART
% 		[ ] CAN
% siehe auch AE1 Skript
Damit nachvollzogen werden kann, ob die Kommunikation über den SPI-Bus funktioniert, 
wird über ein Oszilloskop der Datenverkehr des Masters beobachtet. 
Bei erfolgreicher Signaleübertragung zeigt das Oszilloskop die Signalveränderung.

Ähnlich zu SPI kann der Datenverkehr auch bei UART und CAN mit passenden Software beobachtet und überprüft werden.
Für den UART-Bus wird HTerm verwendet. % Datentransfer am PC beobachten.

Für CAN kommt ein Ixxat-Dongel zum Einsatz. % Ixxat Dongel, CAN Mini (oder so ähnlich) :)



























