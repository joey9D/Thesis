Das Ziel dieser Arbeit ist es die Basis eine Treiber-API zu erstellen, mit der je nach Zielhardware die passenden Treiber integriert werden können.
Dafür muss 
Diese soll erste grundlegenden Funktionen für GPIO, SPI, CAN und UART enthalten. 
Auf diese Weise kann die Funktionsweise für generelles Lesen und Schreiben und die Kommunikation über Busse getestet werden.
\\

\textbf{Warum für diese Entschieden?}
GPIO für die simpelste Art für Lesen und Schreiben

CAN damit ein anderes Projekt direkt integriert werden kann.


SPI und UART um Kommunikation via Bus zu testen




\section{Rahmenbedingungen}
Werkzeuge:
\begin{itemize}
	\item STM32CuebeIDE $\rightarrow$ VSCode
	\item C $\rightarrow$ C++
	\item CMake
	\item Linker-Files
\end{itemize}

VSCode Erweiterungen:
\begin{itemize}
	\item by franneck94
	\begin{itemize}
		\item C/C++ Runner v9.4.10
		\item C/C++ Config v6.3.0
	\end{itemize}
	\item Microsoft
	\begin{itemize}
		\item C/C++ Extension Pack v1.3.1
		\item C/C++ v1.25.3
		\item CMake Tools v1.20.53
	\end{itemize}
\end{itemize}

Verwendete Microcontroller:
\begin{itemize}
	\item STM32C032C6
	\item STm32G071RB
	\item STM32G0B1RE
	\item ESP32-C6 DevKitC-1
\end{itemize}


\section{Welche Eigenschaften muss die Treiber-API haben?}
%Welche (architektonischen) Eigenschaft sind wichtig/sollen umgesetzt werden?
%\begin{itemize}
%	\item keine/geringen Redundanz $\rightarrow$ z.B. Klassen sollen nicht immer neu implementiert werden.
%	\item einfache Benutzung $\rightarrow$ damit auch zukünftige neue Mitarbeiter einen schnellen Einstieg und Verständnis für die Umgebung bekommen.
%	\item Skalierbarkeit $\rightarrow$ soll auf möglichst viele MCUs/Hardwareboards funktionieren/kompatibel sein.
%	\item Portabilität $\rightarrow$ mit Blick auf unterschiedliche Betriebssysteme (hier: Windows, Linux und MacOS), sollt die erstellte Library auf möglichst vielen bekannten Betriebssystemen laufen. Die damit verbundene Installation der benötigten Tools sollte dementsprechend dokumentiert sein.
%	\item Erweiterbarkeit $\rightarrow$ Leistungsstärkere MCUs bringen oft weitere Funktionen mit. Es muss einfach sein, die implementierten Klassen um diese neuen Funktionen zu erweitern.
%	\item Modularität $\rightarrow$ das Strukturieren der Library in klare Module hilft nicht nur der Trennung von Funktionen und dem damit gewonnen Überblick, sondern dient auch der Wartbarkeit, indem sie es ermöglicht Fehlerquellen schneller zu lokalisieren und diese dann zu beheben.
%	\item Effizienz $\rightarrow$ die Ressourcen, die eine Microcontroller mit bringt sind sehr begrenzt. So muss darauf geachtet werden, dass die Applikation und ihre Abhängigkeiten, z.B. externe Libraries nicht  groß werden und den gesamte Speicher einnehmen.
%\end{itemize}

Die Entwicklung einer plattformunabhängigen, wiederverwendbaren Treiber-API für Mikro-controller stellt hohe Anforderungen an die Architektur der Softwarebibliothek.

% geringe Redundanz
Das Ziel besteht darin, eine Lösung zu schaffen, die sich durch eine geringe Redundanz auszeichnet. 
Die Konzeption von Klassen und Funktionen sollte derart erfolgen, dass eine erneute Implementierung der Applikation für jede neue Plattform nicht erforderlich ist.
Die Wiederverwendbarkeit zentraler Komponenten führt zu einer Reduktion des Entwicklungsaufwands und einer Erhöhung der Konsistenz im Code.

% Usability - Benutzerfreundlichkeit
Ein weiteres zentrales Anliegen ist die einfache Benutzbarkeit. 
Die API ist so zu gestalten, dass eine effiziente Nutzung gewährleistet ist. 
Dies fördert nicht nur die Effizienz in der Erstellung neuer Applikationen, sondern erleichtert auch langfristig die Wartung und Weiterentwicklung der Software.

% Skalierbarkeit
Im Sinne der Skalierbarkeit wird angestrebt, die Lösung auf möglichst viele Mikrocontroller-Architekturen und Hardwareplattformen anwendbar zu machen.
Die Vielfalt verfügbarer MCUs erfordert eine abstrahierte und flexibel erweiterbare Struktur, die die Integration neuer Plattformen mit minimalem Aufwand ermöglicht.

% Portabilität - nochmal anpassen; OS Bezug passt nicht richtig
Auch die Portabilität spielt eine wichtige Rolle.
Die Bibliothek sollte nicht nur hardware-, sondern auch betriebssystemunabhängig konzipiert werden.
Aus diesem Grund wird bei der Entwicklung der Lösung darauf geachtet, dass diese erst unter Windows, später auch unter Linux und macOS einsetzbar ist.
Die Installation und Konfiguration der dafür benötigten Werkzeuge wird nachvollziehbar dokumentiert, um den Einstieg für die Nutzer zu erleichtern.

% Erweiterbarkeit
Darüber hinaus ist die Erweiterbarkeit ein wesentliches Architekturprinzip
Der Einsatz von leistungsstärkeren Mikrocontrollern hängt in der Regel mit einer Erweiterung der Funktionalitäten zusammen, die in die bestehenden Treiber- und API-Strukturen integriert werden müssen.
Daher wird großer Wert auf eine modulare und offen gestaltete Architektur gelegt, die neue Features ohne grundlegende Umbauten aufnehmen kann.

% Modularität
Modularität trägt wesentlich zur Übersichtlichkeit und Wartbarkeit des Systems bei. 
Eine saubere Trennung funktionaler Einheiten ermöglicht eine schnellere Lokalisierung und Behebung von Fehlern, was wiederum die langfristige Pflege und Weiterentwicklung der Software erleichtert.

% (Ressourcen-)Effizienz
Schließlich ist auch die Effizienz ein kritischer Aspekt.
Da Mikrocontroller in der Regel nur über begrenzte Ressourcen verfügen, ist es essenziell, dass die Bibliothek möglichst kompakt und ressourcenschonend implementiert wird. 
Externe Abhängigkeiten werden bewusst auf ein Minimum reduziert, um Speicherplatz zu sparen und unnötige Komplexität zu vermeiden.

Diese architektonischen Prinzipien bilden die Grundlage für die Konzeption und Umsetzung der in dieser Arbeit vorgestellten Treiber-API.
 % Fragen
Wie wird der jeweilige Punkt umgesetzt?

Welche Tools werden benutzt/eignen sich besonders für die Umsetzung?
Welche Tools eignen sich für welchen Arbeitsschritt?

Warum wird etwas gerade auf diese Weise umgesetzt?


\section{Anforderungen an die Lösung}
Erfolgreiche Implementierung der Grundfunktionen 
von GPIO, SPI, UART, CAN. Das beinhaltet die Kommunikation über diese Technologien, d.h. Lesen und Schreiben.


