

In der heutigen digitalen Welt spielen Programmierschnittstellen eine zentrale Rolle bei der Entwicklung verteilter, modularer und skalierbarer Softwaresysteme. 

Diese Anwendungsprogrammierschnittstellen (\gls{api}) ermöglichen die strukturierte Kommunikation zwischen Softwarekomponenten über definierte Protokolle und Schnittstellen und abstrahieren dabei komplexe Funktionen hinter einfachen Aufrufen.
Während die Nutzung von \gls{api}s im Web- und Cloud-Umfeld bereits als etablierter Standard betrachtet werden kann, gewinnt diese Technologie auch in der Embedded-Entwicklung zunehmend an Relevanz.
Insbesondere im Bereich der Microcontroller tragen APIs zur Wiederverwendbarkeit, Portabilität und Wartbarkeit von Software bei.
Die zunehmende Komplexität eingebetteter Systeme sowie die Anforderungen an die Zusammenarbeit mit anderen Systemen, die Echtzeitfähigkeit und  die Ressourceneffizienz machen eine strukturierte Schnittstellendefinition unerlässlich.
%Anwendungsprogrammierschnittstellen
\gls{api}s arbeiten in diesem Kontext als Vermittler zwischen der modularen Struktur von Applikation und Anwendungslogik, und der hardwarenahen Programmierung.
Zu typischen Anwendungsfällen zählen sowohl abstrahierte Zugriffe auf Peripheriekomponenten, Sensordaten, Kommunikationsschnittstellen als auch Betriebssystemdienste in Echtzeitbetriebssystemen (\gls{rtos}).