Die vorliegende Bachelorarbeit mit dem Titel "Deign und Implementierung einer Treiber-API für industrielle Kommunikation" wurde als Abschlussarbeit des Studiums der Technischen Informatik in den Schwerpunkten Cyber-Physical-Systems and Security und Application Development\\ (StuPO 22.2) verfasst.

Der Inhalt der Arbeit wurde in Zusammenarbeit mit der Firma Schaefer GmbH in Sigmaringen-Laiz erarbeitet und dokumentiert.
% TODO: Vorwort: Zwischenschicht, so lassen?
Ziel der Thesis war es, eine Basis einer Zwischenschicht (API) zu entwickeln, die es ermöglicht, einmal erstellt Programme für unterschiedliche Hardware, d.h. Mikrokontroller, wiederverwendbar zu machen. in dem je nach Hardware, die richtigen Treiber automatisch ausgewählt und verwendet werden.

Die Schaefer GmbH ist ein mittelständisches Unternehmen, das sich durch ein breites Portfolio an Bedienelementen sowie langjähriger Expertise einen festen Platz in der internationalen Aufzugsbranche erarbeitet hat. 
Das Unternehmen zählt heute zu den führenden Anbietern von anwender-, design- und technologisch orientierten Komplettlösungen im Aufzugbau.
Das Sortiment umfasst eine Vielzahl von Bedien- und Anzeigeelementen, Kabinen- und Ruftableaus sowie individuell gestaltete Komponenten in diversen Formen, Farben, Materialien und Oberflächen. 
Mit der Entwicklung, Produktion und dem Vertrieb elektrischer und elektrotechnischer Geräte und Systeme sowie die dazugehörigen Softwarelösungen werden ganzheitliche Produkte und Leistungen angeboten.
Das Resultat sind maßgeschneiderte Lösungen, die nicht nur funktionale, sondern auch ästhetische Anforderungen erfüllen. 

Zusammen mit Michael Grathwohl, M.Eng, meinem Betreuer bei der Schaefer GmbH, wurde das Thema der Thesis und der Umfang der praktischen Umsetzung festgelegt.
Die Mitarbeiter der Produktentwicklung verfolgten den Fortschritt mit großem Interesse, um Sachverhalte und Zusammenhänge der Arbeit mit der aktuellen Umgebung zu verbinden.
Besonders im Hinblick auf zukünftige Einsätze und Erweiterungen der Zwischenschicht.
% TODO: Vorwort: Danksagung
%Für den Erfolg dieser Arbeit 
