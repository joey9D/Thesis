Das STM32Cube-Ecosystem \cite{stm32cube_ecosystem} der Firma STMicroelectronics bietet ein gesamtes System, von der Auswahl und der Konfiguration der Hardware bis hin zu einer IDE zur Softwareentwicklung und einer Software um den internen Speicher der MCUs zu programmieren.
Die Kernprogramme sind dabei:

\paragraph{STM32CubeMX} 
	dient der Konfiguration der Hardware, d.h. Benennung und Funktionszuweisung der Pins, Aktivieren oder Deaktivieren von Registern und Protokollen, Konfiguration der internen Frequenzen über ein graphische Oberfläche.
	Nach der Konfiguration kann der Code für das Projekt generiert werden.
	In diesem Schritt werden die notwendigen Pakete, Treiber (\gls{hal}, CMSIS % TODO: CMSIS Glossareintrag
	) und Firmware für die ausgewählte Hardware geladen. \cite{stm32cubemx}

\paragraph{STM32CubeIDE}
	dient der Softwareentwicklung für die MCUs zu entwickeln und implementieren.
	Die Entwicklungsumgebung, basierend auf Eclipse, bietet neben dem Codeeditor ein eigenes Buildsystem, das mit Make und der \texttt{arm-none-ebai-gcc}-Toolchain arbeitet und einen Debugger hat, mit dem nicht nur Code sondern auch das Verhalten der Hardware beobachtet werden kann um Fehler zu erkennen. \cite{stm32cubeide}
\\
\\
Wird ein neues Projekt über STM32CubeMX gestartet werden automatisch die benötigten Hardwaretreiber und Firmware heruntergeladen und der Projektstruktur hinzugefügt, gleichzeitig wird ein Coderahmen in C generiert. 
In der Hauptheaderdatei (main.h) befinden sich neben den eingebundenen Headerdateien der Treiber auch die Pindefinitionen, die zuvor konfiguriert wurden.
In der Hauptprogrammdatei (main.c) sind generierte Funktionen für die Hardwareinitialisierung und die Pininitialisierung zu finden.
% Schichtenarchitektur
Untersucht man den Aufbau des gesamten Projekts von der Hauptdatei ausgehend soweit bis man die Register in den Funktionen der \gls{hal} findet, lassen sich Schichten erkennen.
% Anwendungsschicht
Als oberste Schicht wird die Anwendungsschicht gesehen.
Diese besteht aus allen Dateien, Funktionen und Klassen, die nicht direkt mit Registern zu tun haben.
Stattdessen werden die Funktionen der nächst tieferen Schicht verwendet.
\\
\\
Allerdings ist das auch ein Aspekt, der bedacht werden muss. 
Das Softwarepaket funktioniert nur mit der STM32-Hardware.
Der Einsatz mit MCUs anderer Hersteller ist damit nicht vorgesehen.
Für allgemeine Projekte bzw. st-fremde Hardware besteht die Möglichkeit, in der STM32CubeIDE leere CMake-Projekte zu erstellen.
Hier müssen dann die benötigten Pakete und Treiber selber inkludiert werden.
Ein Buildsystem müsste selber eingebunden und mit eigenen CMake-Dateien implementiert werden.


