Das Open-Source-Projekt \emph{mcu-cpp} verwendet einen eigenen \texttt{namespace} um die einzelnen Funktionen und Klassen zu gruppieren.
\emph{Namespaces} sind eine Möglichkeit in C++ um Variablen, Klasse und Funktionen zu gruppieren, damit Konflikte bei der Benennung solcher Identifizierer zu vermeiden.
Die ermöglicht einen sauber-strukturierten und lesbaren Applikationscode zu schreiben, in dem man nachvollziehen kann, wer was aufruft.
Basierend auf den virtuellen Klassen, implementieren die jeweiligen MCUs die Methoden damit diese für sich funktionieren.
Um innerhalb einer Produktfamilie, z.B. STM32F0 MCUs, die richtigen bzw. alle notwendigen Ports zu aktivieren, gibt es eine zusätzliche Datei \texttt{gpio\_hw\_mapping.hpp}.
In dieser werden einzelne Ports, die nicht auf jeder MCU verfügbar sind, durch bedingte Kompilierung aktiviert oder nicht.
Die Information, welche Hardware verwendet wird, muss entweder in der \texttt{CMakeLists.txt} oder im Code mit \texttt{\#define} angegeben sein.
Zusätzlich werden die CMSIS-Treiber verwendet, die die Startdateien bereit stellen.
Als RTOS wird aktuell FreeRTOS verwendet.
%Dies ermöglicht
Allerdings fehlen hier die offizielle \emph{Hardware-Abstracition-Layer} (HAL), die bereits vorgefertigte Strukturen und Funktionen für die einzeln Hardwarefunktionen implementiert haben.
Stattdessen werden diese durch die Implementierung der virtuellen Klassen ersetzt.
Das sorgt im weiteren Verlauf dafür, dass die Funktionen auf Basis der virtuellen Klassen für jede neue MCU-Familie neu implementiert werden muss, was einen für wiederholten Aufwand sorgt und den Anforderungen an die Lösung widerspricht.

