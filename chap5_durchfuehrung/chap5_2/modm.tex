% TODO: chap4 modm: Quellen-Verlinkungß
Das Open-Source-Projekt \emph{modm} dient als Baukasten um zugeschnittene und anpassbare Bibliotheken für Mikrocontroller zu generieren.
Dadurch ist es möglich, dass eine Bibliothek nur aus den Teilen besteht, die tatsächlich in der Applikation und im Code verwendet werden müssen, ohne das es einen unnötig großen Overhead gibt.
Um das zu bewerkstelligen wird eine Kombination aus \texttt{Jinja2}-Template-Dateien, \texttt{lbuild}-Pyhton-Skripte und eigenen Moduldefinitionen verwendet, mit der der Code für die Bibliotheken generiert wird.
Die Templatedateien enthalten Platzhalter.
Die Werte kommen aus \texttt{YAML} und \texttt{JSON}-Dateien, die von den \texttt{lbuild}-Pyhtonskripten gelesen und in die entsprechenden Positionen der Platzhalter, während des Buildprozesses, eingefügt werden.
 
Um eine Bibliothek zu erstellen, muss ein Prozess über die Konsole gestartet werden.
modm hat bereits vordefinierten Konfigurationen für eine große Auswahl an MCUs.
Mit diesen kann die Bibliothek für ein Projekt erstellt/gebuildet werden.

Will man aber Module verwenden, die in der vordefinierten Konfiguration nicht enthalt sind, kann man Module einzeln zu der \texttt{project.xml} hinzufügen.
Um sehen zu können welche Module zur Verfügung stehen muss folgende Zeile in der Konsole ausgeführt werden:

\vspace{3mm}
\begin{lstlisting}[language=bash, caption={Konsolenbefehl um verf\"ugbare Module aufgelistet zu bekommen; hier f\"ur den STM32C031C6T6 Mikrokontroller.}, label={lst:modm_lbild_discover}]
\modm\app\project>
	lbuild --option modm:target=stm32c031c6t6 discover
\end{lstlisting}

Sobald die gewünschten Module hinzugefügt wurde, beginnt der Installations- bzw. der Generierungsprozess der Library. 
Gibt man nun \texttt{lbuild builld} in der Konsole ein wenn man sich im \texttt{app/project}-Verzeichnis kann die Bibliothek erstellt werden.
Nach erfolgreichem Build erscheint in dem Projektverzeichnis ein neuer Ordner \emph{modm}.
Dieser enthält die generierten Dateien der ausgewählten Module.

Positiv hervorzuheben ist hier das (vordergründige) simple Hinzufügen von Modulen.
Da das Projekt aktuell bereits sehr umfangreich ist und sehr viele Mikrokontroller und Optionen unterstützt, bietet es eine große Auswahl an Modulen, die beliebig zu einem Projekt hinzugefügt werden können.

Allerdings ist zu beachten, dass falls man zukünftig neue Module oder Mikrokontroller hinzufügen will, müssen diese an die bestehende Struktur angepasst und in das Zusammenspiel von Python, Jinja2 und den YAML/JSON-Dateien integriert werden.
Dies ist mit einem sehr hohen Aufwand verbunden.
