\section{Betrachtung bestehender Lösungen}
In diesem Kapitel erfolgt eine Untersuchung des aktuellen Stands der Technik im Bereich der hardwarenahen Softwareentwicklung für Mikrocontroller.
Das Ziel besteht darin, gemeinsame Eigenschaften heraus zu arbeiten, verwendete Architekturmuster zu identifizieren und bestehende Ansätze und Konzepte zu analysieren, die das Problem der Treiberauswahl und -abstraktion lösen – insbesondere im Hinblick auf Portabilität und Wiederverwendbarkeit. 

Die Analyse dient zudem der Identifikation möglicher Lücken oder Einschränkungen bestehender Lösungen und trägt somit zur Begründung der Relevanz und Zielsetzung dieser Arbeit bei.


\subsection{Stand der Technik}
Im Rahmen der Untersuchung wurden neben Onlinerecherchen speziell zwei praxisnahe Quellen herangezogen. 
Zu den praxisnahen Quellen zählen technische Dokumentationen, Open-Source-Projekte und Herstellerdokumentationen.
Der Fokus der Recherche lag auf bestehenden Lösungen für die plattformübergreifende Auswahl von Hardwaretreibern für Mikrocontroller.
Die im Rahmen der Untersuchung verwendeten relevanten Schlüsselbegriffe umfassten unter anderem \textit{Hardware Abstraction Layer, Embedded Driver Portability, CMSIS, Arduino Core, Zephyr RTOS,C++ Hardware API Design}.

Auf diese Weise wurden verschiedene Ansätze zur Hardwareabstraktion und Treiberbereitstellung gefunden.
Die \emph{Common Microcontroller Software Interface Standard} (CMSIS)-Bibliothek ist eine von ARM entwickelte Schnittstelle, die eine weit verbreitete Anwendung findet. 
Sie bietet eine einheitliche Zugriffsebene für Cortex-M-Prozessoren. 
Herstellerbezogene Entwicklungsumgebungen wie die STM32CubeIDE von STMicroelectronics und die Espressif-IDE bieten umfangreiche Hardware-Abstraktionsbibliotheken, die gezielt auf ihre jeweiligen Mikrocontroller-Familien zugeschnitten sind.

Darüber hinaus wurden zwei Open-Source-Projekte auf GitHub analysiert: mcu-cpp und modm. 
Die Zielsetzung beider Ansätze besteht in der Modularisierung der Treiberentwicklung in C++ sowie der Bereitstellung portabler, wiederverwendbarer Hardware-APIs. 
Die Projekte zeigen eine Reihe unterschiedlicher Herangehensweisen in Bezug auf Abstraktionslevel, Architektur und Hardwareunterstützung, was wertvolle Erkenntnisse für die eigene Lösungsentwicklung bietet.
\\
\\
In den folgenden Absätzen werden die einzelnen Plattformen bewertet und potentiellen Vor- und Nachteile benannt; auch in Bezug auf die Anforderungen der eigenen Lösung. 
%Wie machen das andere; auch in Form kommerzieller Werkzeuge (cubeIDE, Espressif):
%\begin{itemize}
% 	\item CMSIS
%	\item Espressif IDE
%\end{itemize}


\subsection{STM32Cube}
Das STM32Cube-Ecosystem \cite{stm32cube_ecosystem} der Firma STMicroelectronics bietet ein gesamtes System, von der Auswahl und der Konfiguration der Hardware bis hin zu einer IDE zur Softwareentwicklung und einer Software um den internen Speicher der MCUs zu programmieren.
Die Kernprogramme sind dabei:

\paragraph{STM32CubeMX} 
	dient der Konfiguration der Hardware, d.h. Benennung und Funktionszuweisung der Pins, Aktivieren oder Deaktivieren von Registern und Protokollen, Konfiguration der internen Frequenzen über ein graphische Oberfläche.
	Nach der Konfiguration kann der Code für das Projekt generiert werden.
	In diesem Schritt werden die notwendigen Pakete, Treiber (\gls{hal}, CMSIS % TODO: CMSIS Glossareintrag
	) und Firmware für die ausgewählte Hardware geladen. \cite{stm32cubemx}

\paragraph{STM32CubeIDE}
	dient der Softwareentwicklung für die MCUs zu entwickeln und implementieren.
	Die Entwicklungsumgebung, basierend auf Eclipse, bietet neben dem Codeeditor ein eigenes Buildsystem, das mit Make und der \texttt{arm-none-ebai-gcc}-Toolchain arbeitet und einen Debugger hat, mit dem nicht nur Code sondern auch das Verhalten der Hardware beobachtet werden kann um Fehler zu erkennen. \cite{stm32cubeide}
\\
\\
Wird ein neues Projekt über STM32CubeMX gestartet werden automatisch die benötigten Hardwaretreiber und Firmware heruntergeladen und der Projektstruktur hinzugefügt, gleichzeitig wird ein Coderahmen in C generiert. (Code \ref{lst:stm32_mx_gpio_init} ist Teil des generierten Coderahmens.)
%In der Hauptheaderdatei (main.h) befinden sich neben den eingebundenen Headerdateien der Treiber auch die Pindefinitionen, die zuvor konfiguriert wurden.
%In der Hauptprogrammdatei (main.c) sind generierte Funktionen für die Hardwareinitialisierung und die Pininitialisierung zu finden.

Dies Funktioniert im Kosmos der STM32Cube-Plattform sehr gut, allerdings ist dies auch Aspekt der beachtet werden muss:\\
Das Softwarepaket funktioniert nur mit der STM32-Hardware, der Einsatz mit \gls{mcu}s anderer Hersteller ist nicht vorgesehen.
Für allgemeine Projekte bzw. st-fremde Hardware besteht die Möglichkeit, in der STM32CubeIDE leere CMake-Projekte zu erstellen.
Die benötigten Pakete und Treiber, sowie ein Buildsystem müssen dann selber inkludiert und mit eigenen CMake-Dateien implementiert werden.


% Schichtenarchitektur
Untersucht man den Aufbau des gesamten Projekts von der Hauptdatei ausgehend soweit bis die Register in den Funktionen der \gls{hal} erreicht sind, lassen sich Schichten erkennen.
% Anwendungsschicht
Die Anwendungsschicht beinhaltet das Hautprogramm inklusive des Hauptheaders.
% Middleware & RTOS
Ein explizite Middleware und Betriebssystemschicht fehlen in einem blanken Projekt, wenn man diese während des Konfigurationsprozesses nicht explizit hinzugefügt hat.
% CMSIS & HAL
In der Treiber- und Abstraktionsschicht finden sich \gls{hal} und CMSIS-Treiber, mit allen benötigten Funktionen und Definitionen um auf Register zuzugreifen und Pins steuern zu können. 

% Designpattern
Sucht man den Code nach Designmuster lassen sich für alle drei Kategorien Exemplare finden.
Für Erzeugungsmuster lassen sich Vergleiche zu Singleton und Builder finden.
Die \texttt{GPIO\_InitStruct}, die man bereits in Code \ref{lst:stm32_mx_gpio_init} sehen kann, zeigt Ähnlichkeiten zu dem Builder-Muster.
Die Struktur wird hier ebenfalls Option für Option aufgebaut und erweitert.
Wird \gls{spi} kommt eine globale \texttt{SPI\_HandleTypeDef} Instanz dazu, ähnlich dem Singleton-Pattern.

Sucht man nach Strukturmuster lässt sich das Facade-Pattern erkennen.
Um Pins zu initialisieren wird die Funktion \texttt{HAL\_GPIO\_Init(GPIO\_TypeDef  *GPIOx, const GPIO\_InitTypeDef *pGPIO\_Init)} verwendet, die eine Pointer auf den Port und die Adresse auf die Struktur, die die Pininformationen enthält, übergeben bekommt.
Diese Funktion kann beliebig vom Entwickler verwendet werden, ohne dass dieser Wissen muss, wie diese Funktion intern die übergebenen Informationen verarbeitet.

Im Bereich Verhaltensmuster findet man die Template Method.
Bei diesem Muster definiert eine Basisklasse ein Algorithmus, d.h. eine feste Reihenfolge von Befehlen oder Funktionen; sie implementiert aber nicht alle Befehle selber.
Einige Zwischenschritte, sog. Hooks, können von Unterklassen implementiert werden.
Im Fall der STM32-\gls{hal} findet man dieses Pattern bei den Callback-Funktionen für Interrupts.
Hier ist der \texttt{void HAL\_GPIO\_EXTI\_IRQHandler(uint316\_t GPIO\_Pin)} das Template, die \texttt{void HAL\_GPIO\_EXTI\_Callback(uint16\_t GPIO\_Pin)} die Hook-Funktion, die vom Entwickler selber implementiert werden kann.

%Zusammenfassend wird sich hier in einer Schichtenarchitektur bewegt.






\subsection{Espressif-IDF}
% TODO: chap4 Espressif-IDE
%Aufbau / Ebenene
%\begin{itemize}
%	\item main
%	\item Funktionen
%	\item Funktionen $\rightarrow$ HAL-Funktion
%	\item HAL-Funktion $\rightarrow$ LL-Funktion
%\end{itemize}

Das ESP-IDF (Espressif IoT Development Framework) stellt ein offizielles Entwicklungsframework für die Microcontroller der Firma Espressif dar, wie etwa das ESP32 und dessen Varianten. 
Es stellt ein umfangreiches Ökosystem bereit, das sowohl die Auswahl und Konfiguration der Hardware als auch die Entwicklung, das Flashen und Debugging von Software einschließt.

Anders als bei der STM32Cube-Umgebung gibt es hier ein primär Paket, das für die Entwicklung installiert werden muss.
Im Rahmen dieser Installation werden die erforderlichen Softwarekomponenten automatische mit integriert.
Zu diesen Komponenten zählen:

\paragraph{Toolchain}
, die die passenden Compiler und die erforderlichen Werkzeuge zum Übersetzen des Quellcodes für die jeweilige ESP32-Plattform mit sich bringt. 
Diese beinhalten die Xtensa GCC Toolchain (\texttt{xtensa-esp32-elf-gcc}) für ältere Modelle wie  ESP32-, ESP32-S2- und ESP32-S3-Modelle.
Für neuere Modelle wie den ESP32-C3 und ESP32-C6, die auf RISC-V basieren, wird die RISC-V GCC Toolchain (\texttt{riscv32-esp-elf-gcc}) verwendet.

\paragraph{Build-Tools} 
bestehen aus \texttt{CMake} und \texttt{Ninja} als Generator. 
CMake übernimmt die Konfiguration und Verwaltung des Projektes sowie die Generierung der entsprechenden Build-Files. 
Ninja sorgt für eine schnelle und effiziente Ausführung des eigentlichen Buildprozesses.

\paragraph{Python Skripte}
übernehmen Aufgaben wie die Verwaltung und Konfiguration der Entwicklungsumgebung, das Bauen von Projekten, das Flashen der Firmware auf die Zielhardware sowie die Automatisierung von häufigen Arbeitsabläufen. 
Diese Skripte verwalten im Hintergrund das Framework, sodass der Entwickler selber wenig bis garnicht mit diesen in Kontakt kommt.
Viele Befehle, wie das Kompilieren oder Hochladen, werden über diese Skripte im IDF-Terminal ausgeführt und erleichtern so die Entwicklung und den Workflow mit ESP-IDF erheblich.

\paragraph{Debug-Tools}
wie beispielsweise \texttt{OpenOCD} werden mit installiert.
Diese Werkzeuge ermöglichen neben dem Flashen der Firmware auf die Zielhardware, auch das Setzen von Breakpoints sowie das Debugging direkt auf dem Microcontroller. 
Sie unterstützen verschiedene Schnittstellen (z.\,B. JTAG oder USB) und lassen sich mit gängigen IDEs und Entwicklungsumgebungen integrieren.

\vspace{0.5cm}

Wird ein neues Projekt mit dem ESP-IDF Framework gestartet, erfolgt die Einrichtung der Projektstruktur und der benötigten Komponenten ebenfalls weitgehend automatisiert.
Die Generierung eines neuen Projekts kann über die Kommandozeile des IDF-Terminal oder entsprechende Assistenten wie der ESP-IDF Erweiterung in VSCode erfolgen.
In diesem Prozess generiert das Framework die zugehörige Ordnerstruktur, den Beispielcode sowie die Konfigurationsdateien.
Die erforderlichen Hardwaretreiber, Bibliotheken und Tools wurden bereits mit der Installation des Frameworks bereitgestellt, sodass ein weiterer Download nicht mehr notwendig ist.


Der grundlegende Aufbau eines Projekts im ESP-IDF ist durch eine hierarchische Struktur gekennzeichnet, bei der die einzelnen Ebenen klar voneinander getrennt sind.
Auf oberster Ebene befindet sich die main-Funktion, die den Einstiegspunkt des Programms darstellt.
Von diesem Punkt aus werden die zentralen Initialisierungen ausgeführt und die Steuerung des weiteren Programmablaufs initiiert.
Aus der \texttt{main}-Funktion erfolgt der Aufruf spezifischer Anwendungsfunktionen.
In der Regel erfolgt der Zugriff auf diese Funktionen durch die Verwendung der sog. \gls{hal}-Funktionen.
Das Framework stellt damit einen standardisierten Zugriff auf die zugrunde liegende Hardware bereit.
Die \gls{hal}-Funktionen selbst basieren wiederum auf Low-Level-(LL)-Funktionen, die den direkten Zugriff auf Register und Peripherie des ESP32 ermöglichen.
Diese Schichtung resultiert in einer klare Abstraktion, da die Anwendung hardwareunabhängig entwickelt werden kann, während der Zugriff auf die Peripherie über wohldefinierte Schnittstellen erfolgt.
Zudem besteht bei Bedarf die Möglichkeit, über die LL-Ebene direkt in die Hardware einzugreifen.
Das mehrstufige Konzept zielt darauf ab, sowohl die Portabilität als auch die Wartbarkeit der Software innerhalb des ESP-IDF-Frameworks zu fördern.











































% TODO: chap4 mcu-cpp & modm: Tiefer Analyse: Wie ist der Code aufgebaut -> Architektur-Muster

\subsubsection{mcu-cpp}
Das Open-Source-Projekt \emph{mcu-cpp} verwendet einen eigenen \texttt{namespace} um die einzelnen Funktionen und Klassen zu gruppieren.
\emph{Namespaces} sind eine Möglichkeit in C++ um Variablen, Klasse und Funktionen zu gruppieren, damit Konflikte bei der Benennung solcher Identifizierer zu vermeiden.
Die ermöglicht einen sauber-strukturierten und lesbaren Applikationscode zu schreiben, in dem man nachvollziehen kann, wer was aufruft.
Basierend auf den virtuellen Klassen, implementieren die jeweiligen MCUs die Methoden damit diese für sich funktionieren.
Um innerhalb einer Produktfamilie, z.B. STM32F0 MCUs, die richtigen bzw. alle notwendigen Ports zu aktivieren, gibt es eine zusätzliche Datei \texttt{gpio\_hw\_mapping.hpp}.
In dieser werden einzelne Ports, die nicht auf jeder MCU verfügbar sind, durch bedingte Kompilierung aktiviert oder nicht.
Die Information, welche Hardware verwendet wird, muss entweder in der \texttt{CMakeLists.txt} oder im Code mit \texttt{\#define} angegeben sein.
Zusätzlich werden die CMSIS-Treiber verwendet, die die Startdateien bereit stellen.
Als RTOS wird aktuell FreeRTOS verwendet.
%Dies ermöglicht
Allerdings fehlen hier die offizielle \emph{Hardware-Abstracition-Layer} (HAL), die bereits vorgefertigte Strukturen und Funktionen für die einzeln Hardwarefunktionen implementiert haben.
Stattdessen werden diese durch die Implementierung der virtuellen Klassen ersetzt.
Das sorgt im weiteren Verlauf dafür, dass die Funktionen auf Basis der virtuellen Klassen für jede neue MCU-Familie neu implementiert werden muss, was einen für wiederholten Aufwand sorgt und den Anforderungen an die Lösung widerspricht.


% TODO: chap4 mcu-cpp: Quellen-Verlinkung
Das Open-Source-Projekt \emph{mcu-cpp} verwendet einen eigenen \texttt{namespace} um die einzelnen Funktionen und Klassen zu gruppieren.
\emph{Namespaces} sind eine Möglichkeit in C++ um Variablen, Klasse und Funktionen zu gruppieren, damit Konflikte bei der Benennung solcher Identifizierer zu vermeiden.
Die ermöglicht einen sauber-strukturierten und lesbaren Applikationscode zu schreiben, in dem man nachvollziehen kann, wer was aufruft.
Basierend auf den virtuellen Klassen, implementieren die jeweiligen MCUs die Methoden damit diese für sich funktionieren.
Um innerhalb einer Produktfamilie, z.B. STM32F0 MCUs, die richtigen bzw. alle notwendigen Ports zu aktivieren, gibt es eine zusätzliche Datei \texttt{gpio\_hw\_mapping.hpp}.
In dieser werden einzelne Ports, die nicht auf jeder MCU verfügbar sind, durch bedingte Kompilierung aktiviert oder nicht.
Die Information, welche Hardware verwendet wird, muss entweder in der \texttt{CMakeLists.txt} oder im Code mit \texttt{\#define} angegeben sein.
Zusätzlich werden die CMSIS-Treiber verwendet, die die Startdateien bereit stellen.
Als RTOS wird aktuell FreeRTOS verwendet.
%Dies ermöglicht
Allerdings fehlen hier die offizielle \emph{Hardware-Abstracition-Layer} (HAL), die bereits vorgefertigte Strukturen und Funktionen für die einzeln Hardwarefunktionen implementiert haben.
Stattdessen werden diese durch die Implementierung der virtuellen Klassen ersetzt.
Das sorgt im weiteren Verlauf dafür, dass die Funktionen auf Basis der virtuellen Klassen für jede neue MCU-Familie neu implementiert werden muss, was einen für wiederholten Aufwand sorgt und den Anforderungen an die Lösung widerspricht.



\subsubsection{modm}
% TODO: chap4 modm: Quellen-Verlinkungß
Das Open-Source-Projekt \emph{modm} dient als Baukasten um zugeschnittene und anpassbare Bibliotheken für Mikrocontroller zu generieren.
Dadurch ist es möglich, dass eine Bibliothek nur aus den Teilen besteht, die tatsächlich in der Applikation und im Code verwendet werden müssen, ohne das es einen unnötig großen Overhead gibt.
Um das zu bewerkstelligen wird eine Kombination aus \texttt{Jinja2}-Template-Dateien, \texttt{lbuild}-Pyhton-Skripte und eigenen Moduldefinitionen verwendet, mit der der Code für die Bibliotheken generiert wird.
Die Templatedateien enthalten Platzhalter.
Die Werte kommen aus \texttt{YAML} und \texttt{JSON}-Dateien, die von den \texttt{lbuild}-Pyhtonskripten gelesen und in die entsprechenden Positionen der Platzhalter, während des Buildprozesses, eingefügt werden.
 
Um eine Bibliothek zu erstellen, muss ein Prozess über die Konsole gestartet werden.
modm hat bereits vordefinierten Konfigurationen für eine große Auswahl an MCUs.
Mit diesen kann die Bibliothek für ein Projekt erstellt/gebuildet werden.

Will man aber Module verwenden, die in der vordefinierten Konfiguration nicht enthalt sind, kann man Module einzeln zu der \texttt{project.xml} hinzufügen.
Um sehen zu können welche Module zur Verfügung stehen muss folgende Zeile in der Konsole ausgeführt werden:

\vspace{3mm}
\begin{lstlisting}[language=bash, caption={Konsolenbefehl um verf\"ugbare Module aufgelistet zu bekommen; hier f\"ur den STM32C031C6T6 Mikrokontroller.}, label={lst:modm_lbild_discover}]
\modm\app\project>
	lbuild --option modm:target=stm32c031c6t6 discover
\end{lstlisting}

Sobald die gewünschten Module hinzugefügt wurde, beginnt der Installations- bzw. der Generierungsprozess der Library. 
Gibt man nun \texttt{lbuild builld} in der Konsole ein wenn man sich im \texttt{app/project}-Verzeichnis kann die Bibliothek erstellt werden.
Nach erfolgreichem Build erscheint in dem Projektverzeichnis ein neuer Ordner \emph{modm}.
Dieser enthält die generierten Dateien der ausgewählten Module.

Positiv hervorzuheben ist hier das (vordergründige) simple Hinzufügen von Modulen.
Da das Projekt aktuell bereits sehr umfangreich ist und sehr viele Mikrokontroller und Optionen unterstützt, bietet es eine große Auswahl an Modulen, die beliebig zu einem Projekt hinzugefügt werden können.

Allerdings ist zu beachten, dass falls man zukünftig neue Module oder Mikrokontroller hinzufügen will, müssen diese an die bestehende Struktur angepasst und in das Zusammenspiel von Python, Jinja2 und den YAML/JSON-Dateien integriert werden.
Dies ist mit einem sehr hohen Aufwand verbunden.




%\vspace{3mm}
%\begin{figure}[H]
%	\includegraphics[width=\textwidth]{stm32_c031c6_clean_registers_setPin.PNG}
%	\caption{Register beim setzen des Pin.}
%	\label{fig:stm32_register_setPin}
%\end{figure}
%
%\vspace{3mm}
%\begin{figure}[H]
%	\includegraphics[width=\textwidth]{stm32_c031c6_clean_registers_resetPin.PNG}
%	\caption{Register beim zurücksetzen des Pin.}
%	\label{fig:stm32_register_resetPin}
%\end{figure}
%
%In den Abbildungen \autoref{fig:stm32_register_setPin} und \autoref{fig:stm32_register_resetPin} ist der Wert von Pin 15 (ID15 \& OD15) zu beobachten. Bei \texttt{0x0} wird der Pin zurückgesetzt und die LED leuchtet nicht mehr. Bei \texttt{0x1} wird der Wert auf $1$ gesetzt und die LED beginnt zu leuchten.
%
%Die Funktionen \texttt{HAL\_GPIO\_WritePin(GPIO\_TypeDef \*GPIOx, uint16\_t GPIO\_Pin, GPIO\_PinState PinState)} steuern nicht die in \autoref{fig:stm32_register_setPin} und \autoref{fig:stm32_register_resetPin} gezeigten Register IDR und ODR an, sondern die Set- und Reset-Register BSRR und BRR. 
%
%Diese Register sind \emph{write only}, d.h. sie können nicht ausgelesen werden.
%Wird die Funktion korrekt ausgeführt, kann dass Verhalten an den Registern IDR und ODR beobachtet werden.



























