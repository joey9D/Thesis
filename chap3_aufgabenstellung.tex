Das Ziel dieser Arbeit ist es die Basis eine Treiber-API zu erstellen, mit der je nach Zielhardware die passenden Treiber integriert werden können.

Dafür muss 

Diese soll erste grundlegenden Funktionen für GPIO, SPI, CAN und UART enthalten. 
Auf diese Weise kann die Funktionsweise für generelles Lesen und Schreiben und die Kommunikation über Busse getestet werden.
\\

\textbf{Warum für diese Entschieden?}
GPIO für die simpelste Art für Lesen und Schreiben

CAN damit ein anderes Projekt direkt integriert werden kann.


SPI und UART um Kommunikation via Bus zu testen




\section{Rahmenbedingungen}
Die Arbeit wird in der Schaefer GmbH erstellt. 
Die Firma sorgt mit ihren Custom-Designs von Aufzugkontrollpanels dafür, dass jeder Kunde seinen spezifischen Wunsch erfüllt bekommt.
In diesen Kontrollpanels kommen unterschiedliche \gls{mcu}s zum Einsatz um die jeweiligen Softwarelösungen umzusetzen.
Da die Lösung plattformübergreifend funktionieren soll, wird als \gls{ide} VSCode verwendet, dass weltweit genutzt wird und durch Extension für den gewünschten Gebrauch/Projekt angepasst werden kann. 
Außerdem ist VSCode auf den großen Betriebssytemen lauffähig.
Um einen leichteren Einstieg in die Umgebungen der einzelnen \gls{mcu}s zu haben, werden neben VSCode auch die \gls{ide}s der \gls{mcu}s verwendet.
\begin{itemize}
	\item STM32CubeIDE für STM32 \gls{mcu}s
	\item Espressif IDE und ESP-IDF für ESP32 \gls{mcu}s
\end{itemize}

Damit die \gls{api} direkt auf dem aktuellen Stand ist, soll sie in C++ implementiert werden.
Um über VSCode für das Arbeiten mit C++ vorzubereiten und anzupassen empfiehlt es sich Extension zu installieren.
Der User franneck94 hat dafür bereits ein Paket gebunden, dass die wichtigsten Extensions beinhaltet.
In dem Paket enthalten sind:
\begin{itemize}
	\item C/C++ Extension Pack  v1.3.1 by Microsoft
	\begin{itemize}
		\item C/C++  v1.25.3  by Microsoft		
		\item CMake Tools  v1.20.53  by Microsoft
		\item C/C++ Themes  v2.0.0 by Microsoft 
	\end{itemize}
	\item C/C++ Runner  v9.4.10 by franneck94
	\item C/C++ Config  v6.3.0 by franneck94
	\item CMake  v0.0.17 by twxs
	\item Doxygen  v1.0.0 by Baptist BENOIST
	\item Doxygen Documentation Generator  v1.4.0 by Christopher Schlosser
	\item CodeLLDB  v1.11.4 by Vadim Chugunov
	\item Better C++ Syntax  v1.27.1  by Jeff Hyklin
	\item x86 and x86\_64 Assembly  v3.1.5 by 13xforever
	\item cmake-format  v0.6.11 by cheshirekow
\end{itemize}

%\begin{table}[H]
%	\begin{center}
%		\begin{tabular}{ c | c | c }
%			\textbf{C/C++} & \textbf{Buildsystem} & \textbf{Formatierung/Optik}\\
%			\hline
%			C/C++ & CMake Tools & C/C++ Themes \\
%		\end{tabular}
%		\caption{Zuordnung der VSCode Extensions}
%		\label{tab:vscode_extensions}
%	\end{center}
%\end{table}
Diese 13 Extensions lassen sich so zusammenfassen
zu C/C++ relevant, Buildsystem, Dokumentation und Formatierung \& Optik.
Für die Nutzung mit den verwendeten MCUs gibt es entsprechende Extension.

Um über VSCode STM32-\gls{mcu}s zu programmieren gibt es offizielle Extensions von \\STMicroelectronics.
Zu installieren ist hier \emph{STM32Cube for Visual Studio Code}.

Verwendete Microcontroller:
\begin{itemize}
	\item STM32C032C6
	\item STm32G071RB
	\item STM32G0B1RE
	\item ESP32-C6 DevKitC-1
\end{itemize}


\section{Anforderungen an die Lösung}
Erfolgreiche Implementierung der Grundfunktionen 
von GPIO, SPI, UART, CAN. Das beinhaltet die Kommunikation über diese Technologien, d.h. Lesen und Schreiben.


