
Die Informatik umfasst eine Vielzahl unterschiedlicher Fachgebiete mit teils stark variierenden Schwerpunkten. 
Dazu zählen unter anderem die Web- und Anwendungsentwicklung sowie der Bereich der IT-Sicherheit und viele weitere Disziplinen. 
Im Rahmen dieser Arbeit liegt der Fokus auf dem speziellen Teilbereich der Embedded-Softwareentwicklung.

In diesem Kapitel werden die grundlegenden fachlichen und technischen Konzepte vermittelt, die zum Verständnis der weiteren Inhalte erforderlich sind
% Ablauf des Kapitels
Zu Beginn wird eine Einführung in das Themenfeld der Embedded-Softwareentwicklung gegeben, um ein klares Verständnis dafür zu schaffen, welche Unterschiede diesen Bereich kennzeichnen und wie er sich von anderen Teilgebieten der Informatik unterscheidet.
Darauffolgend werden zentrale Begriffe und Konzepte erläutert, die in der Embedded-Entwicklung eine signifikante Rolle spielen, wie beispielsweise Register, Ports, Peripherieansteuerung und hardwarenahe Programmierung.
Darüber hinaus wird technisches Hintergrundwissen vermittelt, das für das Verständnis der späteren Implementierungsschritte und der Architekturentscheidungen von Relevanz ist.

Das Ziel dieses Kapitels besteht darin, eine solide Wissensbasis zu schaffen, auf der die Analyse bestehender Lösungen sowie die Entwicklung einer eigenen Treiber-API aufbauen können.
\section{Embedded Systems}
Bevor man sich um die Entwicklung von Embedded Systems Gedanken machen kann, muss geklärt werden, um was es sich bei diesen Systemen handelt.
Sog. \emph{eingebettete Systeme} sind Computersysteme, die aus Hardware und Software bestehen und in komplexe technische Umgebungen eingebettet sind.
Diese Umgebungen sind meist maschinelle Systeme, in denen das eingebettete System mit Interaktionen durch einen Benutzer oder auch vollautomatisch agiert.
Die eingebetteten Systeme übernehmen komplexe Steuerungs-, Regelungs- und Datenverarbeitungsaufgaben für oder innerhalb der technischen/maschinellen Systeme.
Damit lässt sich ein Embedded System kurz wie folgt zusammenfassen:

\paragraph{Embedded System:}
Ein eingebettetes System (eng. embedded system) ist ein in sich geschlossenes System, dass für eine spezifische Aufgabe gemacht ist.

\vspace{6 mm}

Die Entwicklung von Software für diese Systeme steht vor zusätzlichen und anderen Hürden als die Entwicklung von Lösungen, die beispielsweise im Bereich des Webs oder der Applikationen benötigt werden.

So muss hier darauf geachtet werden, dass die richtigen Treiber der \gls{mcu}s mit eingebunden werden. 
Die Funktionen, die die Treiber beinhalten sind so implementiert, dass damit die Register in den \gls{mcu}s angesteuert werden.
Diese Register steuern das Verhalten der Hardware und bestimmt wie einzelne Bits gesetzt werden müssen, damit ein Signal gelesen/empfangen z.B. das Drücken eines Tasters oder geschrieben/gesendet z.B. die Ausgabe eines Textes auf dem Bildschirm, werden kann.



\section{Begriffe und Definitionen}

%\begin{itemize}
%	\item namespace
%	\item make/cmake
%\end{itemize}


\subsection*{Microcontroller Unit (MCU)}

\subsection*{Architektur}

\subsection*{Hardware Abstraction Layer (HAL)}

\subsection*{Bus/Bussysteme}

\subsection*{Echtzeit und echzeitfähige Betriebssysteme (\gls{rtos})}

\subsection*{Peripherie}
\begin{itemize}
	\item GPIO
	\item SPI
	\item UART
	\item CAN
	\item Register % erstmal nicht vergessen; mal schauen wie/ob das mit rein kommt.
\end{itemize}

\subsection*{Compiler}

\subsection*{Build}

\section{Hintergrundwissen}
C++

Assembler



