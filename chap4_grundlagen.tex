
Die Informatik umfasst eine Vielzahl unterschiedlicher Fachgebiete mit teils stark variierenden Schwerpunkten. 
Dazu zählen unter anderem die Web- und Anwendungsentwicklung sowie der Bereich der IT-Sicherheit und viele weitere Disziplinen. 
Im Rahmen dieser Arbeit liegt der Fokus auf dem speziellen Teilbereich der Embedded-Softwareentwicklung.

In diesem Kapitel werden die grundlegenden fachlichen und technischen Konzepte vermittelt, die zum Verständnis der weiteren Inhalte erforderlich sind
% Ablauf des Kapitels
Zu Beginn wird eine Einführung in das Themenfeld der Embedded-Softwareentwicklung gegeben, um ein klares Verständnis dafür zu schaffen, welche Unterschiede diesen Bereich kennzeichnen und wie er sich von anderen Teilgebieten der Informatik unterscheidet.
Darauffolgend werden zentrale Begriffe und Konzepte erläutert, die in der Embedded-Entwicklung eine signifikante Rolle spielen, wie beispielsweise Register, Ports, Peripherieansteuerung und hardwarenahe Programmierung.
Darüber hinaus wird technisches Hintergrundwissen vermittelt, das für das Verständnis der späteren Implementierungsschritte und der Architekturentscheidungen von Relevanz ist.

Das Ziel dieses Kapitels besteht darin, eine solide Wissensbasis zu schaffen, auf der die Analyse bestehender Lösungen sowie die Entwicklung einer eigenen Treiber-API aufbauen können.

\section{Embedded Systems}
Bevor auf die Entwicklung eingebetteter Systeme eingegangen werden kann, ist zunächst zu klären, worum es sich bei diesen Systemen handelt.
Der Begriff \emph{''Embedded System''} (deutsch: eingebettetes System) bezeichnet ein Computersystem, das aus Hardware und Software besteht und fest in einen übergeordneten technischen Kontext integriert ist. 
Typischerweise handelt es sich dabei um Maschinen, Geräte oder Anlagen, in denen das eingebettete System spezifische Steuerungs-, Regelungs- oder Datenverarbeitungsaufgaben übernimmt.
Ein wesentliches Merkmal eingebetteter Systeme besteht darin, dass sie nicht als eigenständige Recheneinheiten agieren, sondern als integraler Bestandteil eines übergeordneten Gesamtsystems dienen.
In der Regel operieren sie im Hintergrund und sind nicht direkt mit den Benutzern verbunden. In einigen Fällen erfolgt die Interaktion automatisch, in anderen durch Eingaben des Nutzers.

\paragraph{Definition:}
Ein Embedded System ist ein spezialisiertes, in sich geschlossenes Computersystem, das für eine klar definierte Aufgabe innerhalb eines übergeordneten technischen Systems konzipiert wurde.

\vspace{6 mm}

Die Entwicklung von Software für eingebettete Systeme ist mit besonderen Anforderungen verbunden, die sich signifikant von denen unterscheiden, die etwa in der Web- oder Anwendungsentwicklung üblich sind.
Es ist von besonderer Bedeutung, hardwarenahe Aspekte zu berücksichtigen, da die Software unmittelbar mit der zugrunde liegenden Mikrocontroller-Hardware interagiert.
Ein zentraler Aspekt dabei ist die Integration geeigneter Treiber für die jeweilige Mikrocontroller-Architektur.
Die betreffenden Treiber beinhalten Funktionen, welche den Zugriff auf die Hardware mittels sogenannter Register erlauben.
Register sind spezifische Speicherbereiche innerhalb des Mikrocontrollers, welche eine unmittelbare Manipulation des Hardware-Verhaltens ermöglichen.
Durch das gezielte Setzen oder Auslesen einzelner Bits in diesen Registern ist es möglich, beispielsweise Sensorwerte zu erfassen (z. B. das Drücken eines Tasters) oder Ausgaben zu erzeugen (z. B. das Anzeigen eines Textes auf einem Display).



\section{Begriffe und Definitionen}

%\begin{itemize}
%	\item namespace
%	\item make/cmake
%\end{itemize}


\subsection*{Microcontroller Unit (MCU)}

\subsection*{Hardware Architektur}

\subsection*{Hardware Abstraction Layer (HAL)}

\subsection*{Bus/Bussysteme}

\subsection*{Echtzeit und echzeitfähige Betriebssysteme (\gls{rtos})}

\subsection*{Peripherie}
\begin{itemize}
	\item GPIO
	\item SPI
	\item UART
	\item CAN
	\item Register % erstmal nicht vergessen; mal schauen wie/ob das mit rein kommt.
\end{itemize}

\subsection*{Compiler}

\subsection*{Build}

\section{Hintergrundwissen}
C++

Assembler



