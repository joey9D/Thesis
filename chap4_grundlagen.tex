
Die Informatik umfasst eine Vielzahl unterschiedlicher Fachgebiete mit teils stark variierenden Schwerpunkten. 
Dazu zählen unter anderem die Web- und Anwendungsentwicklung sowie der Bereich der IT-Sicherheit und viele weitere Disziplinen. 
Im Rahmen dieser Arbeit liegt der Fokus auf dem speziellen Teilbereich der Embedded-Softwareentwicklung.

In diesem Kapitel werden die grundlegenden fachlichen und technischen Konzepte vermittelt, die zum Verständnis der weiteren Inhalte erforderlich sind
% Ablauf des Kapitels
Zu Beginn wird eine Einführung in das Themenfeld der Embedded-Softwareentwicklung gegeben, um ein klares Verständnis dafür zu schaffen, welche Unterschiede diesen Bereich kennzeichnen und wie er sich von anderen Teilgebieten der Informatik unterscheidet.
Darauffolgend werden zentrale Begriffe und Konzepte erläutert, die in der Embedded-Entwicklung eine signifikante Rolle spielen, wie beispielsweise Register, Ports, Peripherieansteuerung und hardwarenahe Programmierung.
Darüber hinaus wird technisches Hintergrundwissen vermittelt, das für das Verständnis der späteren Implementierungsschritte und der Architekturentscheidungen von Relevanz ist.

Das Ziel dieses Kapitels besteht darin, eine solide Wissensbasis zu schaffen, auf der die Analyse bestehender Lösungen sowie die Entwicklung einer eigenen Treiber-API aufbauen können.

\section{Embedded Systems}
Bevor auf die Entwicklung eingebetteter Systeme eingegangen werden kann, ist zunächst zu klären, worum es sich bei diesen Systemen handelt.
Der Begriff \emph{''Embedded System''} (deutsch: eingebettetes System) bezeichnet ein Computersystem, das aus Hardware und Software besteht und fest in einen übergeordneten technischen Kontext integriert ist. 
Typischerweise handelt es sich dabei um Maschinen, Geräte oder Anlagen, in denen das eingebettete System spezifische Steuerungs-, Regelungs- oder Datenverarbeitungsaufgaben übernimmt.
Ein wesentliches Merkmal eingebetteter Systeme besteht darin, dass sie nicht als eigenständige Recheneinheiten agieren, sondern als integraler Bestandteil eines übergeordneten Gesamtsystems dienen.
In der Regel operieren sie im Hintergrund und sind nicht direkt mit den Benutzern verbunden. In einigen Fällen erfolgt die Interaktion automatisch, in anderen durch Eingaben des Nutzers.

\paragraph{Definition:}
Ein Embedded System ist ein spezialisiertes, in sich geschlossenes Computersystem, das für eine klar definierte Aufgabe innerhalb eines übergeordneten technischen Systems konzipiert wurde.

\vspace{6 mm}

Die Entwicklung von Software für eingebettete Systeme ist mit besonderen Anforderungen verbunden, die sich signifikant von denen unterscheiden, die etwa in der Web- oder Anwendungsentwicklung üblich sind.
Es ist von besonderer Bedeutung, hardwarenahe Aspekte zu berücksichtigen, da die Software unmittelbar mit der zugrunde liegenden Mikrocontroller-Hardware interagiert.
Ein zentraler Aspekt dabei ist die Integration geeigneter Treiber für die jeweilige Mikrocontroller-Architektur.
Die betreffenden Treiber beinhalten Funktionen, welche den Zugriff auf die Hardware mittels sogenannter Register erlauben.
Register sind spezifische Speicherbereiche innerhalb des Mikrocontrollers, welche eine unmittelbare Manipulation des Hardware-Verhaltens ermöglichen.
Durch das gezielte Setzen oder Auslesen einzelner Bits in diesen Registern ist es möglich, beispielsweise Sensorwerte zu erfassen (z. B. das Drücken eines Tasters) oder Ausgaben zu erzeugen (z. B. das Anzeigen eines Textes auf einem Display).



\section{Begriffe und Definitionen}

\subsection{Grundbegriffe}
\subsubsection*{Microprozessor Unit (MPU)}
Ein Mikroprozessor ist ein vollständig auf einem einzigen integrierten Schaltkreis (Chip) realisierter Prozessor.
Der Prozessor ist die zentrale Recheneinheit eines Computersystems.
Seine Funktion umfasst die Ausführung von Befehlen sowie die Steuerung des Datenflusses innerhalb des Systems. 
Ein Mikroprozessor beinhaltet in der Regel Komponenten wie das Rechenwerk (ALU), Register, Steuerwerk und gegebenenfalls Caches, jedoch keine Peripheriefunktionen wie Speicher oder Schnittstellen. 
Diese müssen extern angebunden werden.
Der Begriff "Mikrocomputer" wird verwendet, um ein auf Basis eines Mikroprozessors aufgebautes Gesamtsystem zu definieren. 
Derartige Systeme sind in klassischen Personal Computern, Laptops oder Servern häufig anzutreffen.
In diesen Geräten wird der Mikroprozessor mit externem RAM, ROM, I/O-Komponenten und weiteren Funktionseinheiten kombiniert.

Demgegenüber ist der Mikrocontroller für spezifische Steuerungsaufgaben mit integrierten Peripheriefunktionen konzipiert. 
Der Mikroprozessor findet dagegen meist in leistungsfähigen, aber nicht auf eine konkrete Aufgabe spezialisierten Systemen Anwendung. 
Insbesondere für allgemeine Rechenaufgaben, komplexe Betriebssysteme sowie Anwendungen mit hohem Ressourcenbedarf erweist sich dieser Prozessor als geeignet.

\subsubsection*{Microcontroller Unit (MCU)}
Ein Mikrocontroller ist ein vollständig auf einem einzigen Chip realisierter Mikrocomputer, der neben dem eigentlichen Prozessor (CPU) auch sämtliche für den Betrieb notwendigen Komponenten integriert. 
Zu den Komponenten eines solchen Systems zählen in der Regel Programmspeicher (Flash), Datenspeicher (RAM), digitale Ein- und Ausgänge (GPIO), Timer, Kommunikationsschnittstellen (wie UART, SPI, I2C, CAN) sowie in vielen Fällen analoge Peripheriekomponenten wie A/D-Wandler oder PWM-Einheiten.

Mikrocontroller werden für spezifische Steuerungs- und Regelungsaufgaben konzipiert und finden typischerweise Anwendung in eingebetteten Systemen, wie beispielsweise Haushaltsgeräten, Fahrzeugsteuerungen, Industrieanlagen oder IoT-Geräten. 
Die Geräte zeichnen sich durch einen geringen Energieverbrauch, eine kompakte Bauform, niedrige Kosten und eine direkte Hardwareansteuerung aus. 
Im Vergleich zu Mikroprozessoren sind für den Grundbetrieb von Mikrocontrollern keine externen Komponenten erforderlich, was besonders kompakte und zuverlässige Systemlösungen ermöglicht.

%\subsubsection*{Firmware}
%Der Begriff "Firmware" bezeichnet eine spezielle Form von Software, die dauerhaft auf einem Gerät gespeichert ist und dort grundlegende Steuerungs- und Betriebsfunktionen übernimmt. Im Gegensatz zu klassischer Anwendungssoftware, die typischerweise eine größere Distanz zur Hardware aufweist, ist Firmware in der Regel sehr nah an der Hardware angesiedelt und direkt auf die jeweilige Plattform zugeschnitten. Sie stellt somit die funktionale Verbindung zwischen der Hardware und höherliegenden Softwareschichten oder der Anwendungslogik her.
%
%In Embedded-Systemen ist die Firmware typischerweise in einem nichtflüchtigen Speicher des Mikrocontrollers abgelegt und wird unmittelbar nach dem Einschalten des Systems ausgeführt. Der Quelltext beinhaltet eine Reihe von Komponenten, darunter Initialisierungsroutinen, Treiber für die Peripherie, Kommunikationsschnittstellen sowie zeitkritische Steuerungsalgorithmen. Änderungen an der Firmware werden in der Regel durch gezielte Updates initiiert, die entweder über ein externes Programmiergerät oder über eine serielle Schnittstelle implementiert werden.
%
%Da Firmware eine zentrale Rolle im zuverlässigen Betrieb von eingebetteten Systemen spielt, wird bei ihrer Entwicklung besonderer Wert auf Stabilität, Effizienz und Robustheit gelegt. In Bereichen, die sich durch eine hohe Relevanz für die Sicherheit charakterisieren lassen (wie beispielsweise die Automobilindustrie oder die Medizintechnik), unterliegt sie strengen Qualitäts- und Testanforderungen.

\subsubsection*{Echtzeit und echzeitfähige Betriebssysteme (\gls{rtos})}
In eingebetteten Systemen ist es von zentraler Bedeutung, dass Aufgaben nicht nur korrekt, sondern auch innerhalb einer genau definierten Zeit ausgeführt werden.
Derartige Systeme unterliegen den sogenannten Echtzeitanforderungen.
Ein System wird als in Echtzeit arbeitend bezeichnet, wenn es die Fähigkeit besitzt, innerhalb einer garantierten Zeitspanne auf Ereignisse zu reagieren.
In diesem Kontext ist nicht die absolute Geschwindigkeit von Bedeutung, sondern die Vorhersagbarkeit des zeitlichen Verhaltens.
In Abhängigkeit von den jeweiligen Anforderungen wird eine Unterscheidung zwischen harter Echtzeit und weicher Echtzeit vorgenommen. \\
% weiche Echtzeit
Bei letzterem Fall, der eine geringere Priorität auf die Einhaltung der Zeitvorgaben legt, werden gelegentliche Überschreitungen der vorgegebenen Zeit toleriert, solange das Gesamtsystem funktionsfähig bleibt.
Dies ist beispielsweise bei Audio- oder Videostreaming der Fall. \\
% harte Echtzeit
Demgegenüber steht harter Echtzeit, bei der das Einhalten der Zeitvorgaben zwingend erforderlich ist. 
Sie findet Anwendung in Bereichen, in denen die zeitnahe Ausführung von entscheidender Bedeutung ist, wie beispielsweise im Fall von Airbagsystemen oder medizinischer Notfalltechnik.

Ein echtzeitfähiges Betriebssystem, auch bekannt als Real-Time Operating System (RTOS), unterstützt die genannten Anforderungen, indem es deterministische Ablaufgarantien bietet. 
Dies impliziert, dass ein spezifischer Input stets zu einem präzise definierten Output innerhalb einer festgelegten Zeit führt. 
Späte oder unerwartete Reaktionen sind in Echtzeitsystemen inakzeptabel, beispielsweise die bereits genannte Steuerung von Airbagsystemen oder medizinischer Notfalltechnik.
Prozesse sind derart organisiert, dass zeitkritische Tasks priorisiert und fristgerecht abgeschlossen werden können. 
Dies wird durch ein planbares Multitasking mit klar definierten Reaktionszeiten und synchronisierter Ressourcenzuteilung ermöglicht. 
% Redundanz
In sicherheitskritischen Anwendungen werden Systeme so konstruiert, dass sie auch im Falle von Teilfehlern weiterhin funktionsfähig bleiben.
Die Realisierung erfolgt durch eine redundante Implementierung, wobei zentrale Komponenten doppelt vorhanden sind oder im Fehlerfall durch andere Einheiten ersetzt werden können, um einen vollständigen Systemausfall zu verhindern.

\subsection{Hardware und Architektur}
\subsubsection*{Hardware Architektur}
- CISC vs. RISC

- Cortex Familie

\subsubsection*{Bus/Bussysteme}
- Kommunikation einzelner Bauteile via Datenleitungen

\subsection*{Register}
- Speicherzellen mit kleiner Zugriffszeit

- innerhalb des Prozessors untergebracht

- oft an bestimmte Zwecke gebunden

\subsubsection*{Peripherie}
\begin{itemize}
	\item GPIO
	\item SPI
	\item UART
	\item CAN
\end{itemize}

\subsection{Abstraktion- und Schnittstellenkonzepte}
\subsubsection*{Hardware Abstraction Layer (HAL)}

- abstrahiert den Anwendungscode auf Hardwarenahen Code.

- beinhaltet Funktionen um die Register des Spezifischen MCU anzusprechen/zu steuern.

- für STM32 Hardwareboards

\subsubsection*{Common Microcontroller Software Interface Standard (CMSIS)}

\subsection{Software-Werkzeuge und Buildprozess}
\subsubsection*{Compiler}
- Präprozessor

- Lexikalische Analyse

- Syntaktische Analyse

- semantische Analyse

- Zwischencode Erzeugung

- Code Optimierung

- Code Generierung

\subsubsection*{Build/Toolchain}

\subsubsection*{Linker}

\subsubsection*{Debugger}

\subsubsection*{CMake/Make}


\subsection{Speicher und Ressourcen}
\subsubsection*{Flash/RAM/ROM}

\subsubsection*{Ressourcenbeschränkungen bei MCUs}

\section{Hintergrundwissen}
C++

Assembler



