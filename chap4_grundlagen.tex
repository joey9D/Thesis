
Der Bereich der Informatik umfasst  sehr viele Teilgebiete.
% TODO: Beschreibung der einzelnen Teilgebiete hinzufügen
So gibt es unteranderem die Web- und die Anwendungsentwicklung; den Bereich der IT-Security, der sich mit der Sicherheit beschäftigt.
Im Falle dieser Arbeit befindet man sich in der Embedded Bereich.
Was genau es damit auf sich hat, welche Fachbegriffe dabei mit auftauchen und was diese bedeuten, wird in den jeweiligen Abschnitten in diesem Kapitel erklärt.
% Ablauf des Kapitels
Angefangen damit, was Embedded-Softwareentwicklung ist,
gefolgt von Definitionen und Erklärung wichtiger Begriffe aus diesem Bereich,
und zusätzlichem Hintergrundwissen rund um die Implementierung und Entwicklung von Hardware.

\section{Embedded Entwicklung}
Bevor man sich um die Entwicklung von Embedded Systems Gedanken machen kann, muss geklärt werden, um was es sich bei diesen Systemen handelt.
Sog. \emph{eingebettete Systeme} sind Computersysteme, die aus Hardware und Software bestehen und in komplexe technische Umgebungen eingebettet sind.
Diese Umgebungen sind meist maschinelle Systeme, in denen das eingebettete System mit Interaktionen durch einen Benutzer oder auch vollautomatisch agiert.
Die Systeme übernehmen komplexe Steuerungs-, Regelungs- und Datenverarbeitungsaufgaben für oder innerhalb der technischen Systemen.

Hier beschäftigt man sich (unteranderem) mit der Hardwareprogrammierung.


\section{Begriffe und Definitionen}
Begriffe bezüglich Softwareentwicklung allgemein:
\begin{itemize}
	\item namespace
	\item Compiler
	\item build
	\item make/cmake
	\item IDE
	\item 
\end{itemize}


\subsection*{Microcontroller Unit (MCU)}

\subsection*{Architektur}

\subsection*{Hardware Abstraction Layer (HAL)}

\subsection*{Bus/Bussysteme}

\subsection*{Echtzeit und echzeitfähige Betriebssysteme (\gls{rtos})}

\subsection*{Peripherie}
\begin{itemize}
	\item GPIO
	\item SPI
	\item UART
	\item CAN
	\item Register % erstmal nicht vergessen; mal schauen wie/ob das mit rein kommt.
\end{itemize}


\section{Hintergrundwissen}
C++

Assembler



